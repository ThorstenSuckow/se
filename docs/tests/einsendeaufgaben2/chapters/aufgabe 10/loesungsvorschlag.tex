Beurteilen Sie folgendes Code-Fragment aus Sicht des Software Engineerings.

\begin{minted}[mathescape,
    linenos,
    numbersep=5pt,
    gobble=2,
    fontsize=\small]{java}
    public double sum ( Bestellung bestellung ) {
        Vector items=bestellung.items;
        double sum=0;
        for (int i=0; i<items.size(); i++) {
        sum+=((Item)items.get(i)).price;;
        }
        return sum;
    }
\end{minted}

\section*{Antwort}

\subsection*{Beobachtungen}
\begin{itemize}
    \item Durch die Übergabe eines Referenz-Parameters liegt \textbf{Stamp}-Kopplung vor.
    \item Eine Bestellung ist ein \textit{Aggregat} und besteht aus mehreren \textit{Items} (Artikeln / Bestell-Positionen):\\
    Es wird auf eine \textit{Collection} des übergebenen Objektes zugegriffen, über die iteriert wird, ohne, dass ein \textbf{Iterator} verwendet wird (der Aufrufer \textit{bricht}, wenn die Datenstruktur (\code{Vector}) gegen eine andere ausgetauscht wird).
    \item Innerhalb der Schleife muss die aufrufende Methode über die \textbf{interne Struktur} der einzelnen Teile der Collection Kenntnisse haben.
    \item Prinzip des \textbf{Information Hiding} / der \textbf{Datenkapselung} verletzt, da auf das Attribut \code{price} eines Objekts der KLasse \code{Item} direkt zugegriffen werden kann.
    \item Laufzeitfehler durch eine mögliche \code{ClassCastException} verursachen höchstwahrscheinlich einen Absturz der Software.
\end{itemize}

\noindent
Insgesamt wird festgestellt, dass der \textbf{Kopplungsgrad} unerwünscht \textbf{hoch} ist.\\

\noindent
Es bietet sich an, die Berechnung der Summe in die Klasse \code{Bestellung} zu verlagern.\\
Aufrufende Methoden müssen dann keine Kenntnisse über die interne Struktur einer Bestellung besitzen.