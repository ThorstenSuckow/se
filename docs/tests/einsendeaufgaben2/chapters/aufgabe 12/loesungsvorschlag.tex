Welche Arten von Kopplung gibt es zwischen Klassen in einem Entwurf?

\section*{Antwort}

Im Folgenden von $1=$ \textit{sehr starke Kopplung} - $5 =$ \textit{schwache Kopplung}\footnote{
\cite[33 ff.]{Mye75}
}:

\begin{enumerate}
    \item \textbf{Content-Kopplung}: Zugriff auf die interne Struktur eines Objektes.
    \item \textbf{Common-Kopplung}: Zugriff und Änderung gemeinsam genutzter globaler Variablen.
    \item \textbf{Stamp-Kopplung}: Übergabe von Objekten als Parameter an eine Methode.
    \item \textbf{Data-Kopplung}: Übergabe einfacher Datentypen an eine Methode, je länger die Parameterliste, desto stärker ist die Kopplung (vgl.~\cite[75]{Wed09b}).
    \item \textbf{Routine Call-Kopplung}: Eine Methode einer Klasse/eines Objekts ruft Methoden einer anderen Klasse/ eines anderen Objektes auf.
\end{enumerate}