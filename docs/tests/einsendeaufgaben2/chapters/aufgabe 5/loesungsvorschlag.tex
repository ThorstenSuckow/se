Gegen welche Grundsätze der Ergonomie wurde bei folgenden Dialogen eines Wizards wo verstoßen?


Unter der Annahme, dass der Wizard nur aus den beiden Dialogen besteht:\\

\subsection*{Dialog 1}
\begin{itemize}
    \item es besteht keine Möglichkeit zum Abbrechen (\textbf{Steuerbarkeit})
\end{itemize}


\subsection*{Dialog 2}
\begin{itemize}
    \item Keine klare Info ersichtlich, ob durch den Button \textit{OK} die Registrierung beendet ist, bzw. ob es sich überhaupt um einen aus mehreren Schritten bestehenden Wizard handelt (\textbf{Selbstbeschreibungsfähigkeit})
    \item es besteht keine Möglichkeit zum Abbrechen oder zum Zurückblättern (\textbf{Steuerbarkeit})\footnote{
    sollte der Wizard aus mehr als nur den gezeigten 2 Dialogen bestehen, gilt dies auch für Dialog 1
    }
\end{itemize}

\noindent
Man kann zusätzlich argumentieren, dass durch fehlende Tastaturkürzel (\textit{Mnemonics}\footnote{
    s. \url{https://en.wikipedia.org/wiki/Mnemonics_(keyboard)}, abgerufen 26.04.2024
}) die (\textbf{Lernförderlichkeit}) nicht erfüllt ist (vgl. \cite[147]{Rau07f}).