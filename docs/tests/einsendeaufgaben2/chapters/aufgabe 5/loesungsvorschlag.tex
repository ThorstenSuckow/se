Gegen welche Grundsätze der Ergonomie wurde bei folgenden Dialogen eines Wizards wo verstoßen?


Dialog 1:

\begin{itemize}
    \item Keine klare Info ersichtlich, in welchem Registrierungsschritt sich der Anwender befindet (\textbf{Selbstbeschreibungsfähigkeit})\footnote{
    bzw. geht für (Zertifikats)-Studenten aus der Aufgabe nicht hervor, ob der Wizard nur aus den beiden Dialogen besteht: ``Ein Wizard besteht aus den folgenden zwei Dialogen...``
    }
    \item es besteht keine Möglichkeit zum Abbrechen bzw. zum Zurückblättern (falls nicht erster Schritt - s. Selbstbeschreibungsfähigkeit)(\textbf{Steuerbarkeit})
\end{itemize}


Dialog 2:
\begin{itemize}
    \item Keine klare Info ersichtlich, ob durch den Button \textit{OK} die Registrierung beendet ist (\textbf{Selbstbeschreibungsfähigkeit})
    \item es besteht keine Möglichkeit zum Abbrechen oder zum Zurückblättern (\textbf{Steuerbarkeit})
\end{itemize}