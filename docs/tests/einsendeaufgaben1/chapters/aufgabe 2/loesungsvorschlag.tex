Ordnen Sie folgendes Projekt nach den in der Kurseinheit in den Abschnitten 1.5.1 – 1.5.7 genannten Aspekten ein:\\

\noindent
Ein größeres Team von Ingenieuren entwickelt die Bordsoftware der Steuerung eines neuen Satelliten.\\
Dieses Team gehört zum Entwicklungsteam des Satelliten.


\section*{Antwort}

\begin{itemize}
    \item es wird eine \textbf{Individualsoftware} entwickelt
    \item es darf davon ausgegangen werden, dass die Bordsoftware \textbf{häufig} genutzt wird.
    \item mit Sicherheit wird die Bordsoftware ausschließlich von \textbf{professionellen, geschulten Anwendern} benutzt
    \item die \textbf{Größe / Komplexität} des Projektes ist als \textbf{groß} einzuschätzen, da ein \textit{größeres Team von Ingenieuren} die Entwicklung durchführen soll
    \item die Fachlichkeit ist aufgrund der Interdisziplinarität der Domäne\footnote{Steuerungssysteme, Physik, Übertragungsprotokolle \ldots} \textbf{komplex}.
    Die Entwickler sind als Teil des Entwicklungsteams bereits mit der Fachlichkeit ``Satellit`` vertraut, in der Hinsicht sind sie also gleichzeitig auch \textit{Fachexperten}.
    \item wahrscheinlich kann davon ausgegangen werden, dass mit allen vorhandenen Technologien (bspw. GPS, Rechnerhardware \ldots) \textbf{keine Näherungslösungen} verwendet werden müssen, aber eine absolute Aussage kann mit den geg. Informationen nicht gegeben werden
    \item da in einem Satelliten als Empfänger von Steuerungssignale begrenzte Resourcen (bspw. hinsichtlich Energieverbrauch) zur Verfügung stehen, sollte \textbf{Effizienz} als kritisch betrachtet werden
    \item \textbf{Verlässlichkeit} ist als sehr kritisch zu erachten, da über den Satelliten jederzeit die Kontrolle behalten werden muss.
    Das Auftreten von Fehlverhalten oder das Risiko des Ausfalls kritischer Systeme sollte also so gering wie möglich gehalten werden.
\end{itemize}
