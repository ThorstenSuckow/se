Welche Vorgehensweise schlagen Sie für folgendes Projekt vor und warum:\\

\noindent
[...]

\section*{Antwort}
Es sind zwei Aufgaben von drei erfahrenen Personen zu realisieren:

\begin{enumerate}
    \item einen Treiber zur Marktreife entwickeln
    \item eine existierende Software zu einer marktfähigen Software zur Benutzung der Karte entwickeln
\end{enumerate}


\noindent
Es würde sich das \textbf{Wasserfallmodell} anbieten, denn:

\begin{itemize}
    \item das Team ist erfahren und besteht aus nicht zu vielen Personen
    \item die Anforderungen an die Funktionsweise des Treibers sind klar definiert
    \item[] \textit{und}
    \item für die zu erstellende Anwendung stehen die Anforderungen in Form eines ausgereiften Entwurfs des GUI zur Verfügung
    \item[] $\rightarrow$ die Realisierungsphase kann zielgerichtet in voller Breite durchgeführt werden
    \item es ist ausreichend Zeit vorhanden ($+2$ Wochen Puffer für initiale Aufwandsabschätzung)
\end{itemize}

\section*{Anmerkung}
Auch ein \textbf{nebenläufiges Vorgehen}  wäre denkbar, wenn die Aufgabe \textit{Treiberentwicklung} unabhängig von der Aufgabe \textit{Anwendungsentwicklung} gesehen werden darf: Da allerdings die Wahrscheinlichkeit besteht, dass die Anwendung von in der Entwurfsphase herausgearbeiteten Schnittstellen der Karte / des Treibers abhängig ist, besteht hier keine klare Abgrenzung der beiden Aufgaben.\\
Nichtsdestotrotz könnte man das Team \textit{Treiberentwicklung} als Vorgängerteam (ggf. zwei Mann aufgrund des fachlichen Anspruchs) im nebenläufigen Modell sehen, und das Team \textit{Anwendungsentwicklung} als Nachfolger``team`` (eine Person), das sofort beginnt, wenn die ersten Schnittstellen herausgearbeitet wurden (ggfl. parallel mit der Implementierung der verschiedenen Layouts, Interaktionselemente usw.).\\
Bei diesem Vorgehen müßten dann in der Umsetzungsphase Rückkoppelungen und Überarbeitungen in Kauf genommen werden, wobei
\textit{Balzert} in~\cite[522]{Bal08} feststellt, dass die Gesamtdauer i.d.R. dennoch kürzer als beim sequentiellen Modell ist.
Genauso führt er dort allerdings auf, dass das Modell gut geeignet ist, wenn die ``Anforderungen beim Projektstart noch nicht alle bekannt sind`` (\cite[524]{Bal08}) - im Gegensatz zu den in der Aufgabe beschriebenen Gegebenheiten, die in Summe für das Wasserfallmodell sprechen.
