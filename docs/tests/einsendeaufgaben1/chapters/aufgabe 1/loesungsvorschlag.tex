Worauf basiert ingenieurmäßiges Vorgehen?


\section*{Antwort}
\textbf{Software Engineering} besitzt den Anspruch, durch \textbf{ingenieurmäßiges Vorgehen} eine \textit{wirtschaftliche Entwicklung} von Software zu gewährleisten.
Dabei bedeutet ``ingenieurmäßiges Vorgehen``, dass Software Engineering auf \textit{wissenschaftlicher Basis} und \textit{kodifizierter Erfahrung}\footnote{
    unter \textit{kodifizieren} versteht man das Zusammenfassen und Festlegen von Regeln und Prinzipien in ein systematisches Format.
} beruht.\\

\noindent
\textbf{Ingenieurmäßiges Vorgehen} selber basiert dabei auf \textit{Normen}, \textit{Standards} und \textit{Regeln}, die auf jahrelangen Erfahrungen in ihrer (erfolgreichen) Anwendung gründen.
Dabei können diese sowohl \textbf{methodischer} als auch \textbf{technologischer} Art sein: Bei den Methoden geht es dabei bspw. um Vorgehensweise, Dokumentation und Organisation in einem Projekt.
Bei der Technologie hingegen u.a. um Werkzeuge und Architekturmuster, die sich bewährt haben\footnote{vgl. \cite[2]{Wed09}}.
