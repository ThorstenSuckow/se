Bei einer Individualentwicklung für eine Versicherung soll der Vorgesetzte von etwa
zwanzig Sachbearbeitern als Stellvertreter dieser Endbenutzer die Anforderungen
bestimmen. Was sagen Sie dazu?


\section*{Antwort}

Stellvertreter der Endbenutzer sind dazu ausgewählt, um deren Bedürfnisse und Wünsche in der Anforderungsphase gegenüber dem Auftragnehmer zu vertreten.\\
Hierdurch soll sichergestellt werden, dass möglichst alle Anforderungen der identifizierten Endanwender für das zu erstellende Projekt berücksichtigt werden.\\
Für eine Individualentwicklung soll nun eine Person die Anforderungen von 20 Sachbearbeitern vertreten.\\
Hierbei besteht die Gefahr, dass der Stellvertreter mit der Aufgabe überlastet ist und in diesem Zug nicht alle Bedürfnisse und Wünsche der Sachbearbeiter gegenüber dem Anforderungsteam kennt oder nicht äußern kann.\\
Evtl. kann der Stellvertreter als Entscheidungsträger auch dazu verleitet sein, die Fülle an Anforderungen nach eigenem Ermessen zu priorisieren und dabei andere Anforderungen zu vernachlässigen.\\

\noindent
Die Anforderungen der einzelnen Sachbearbeiter können darüber hinaus recht speziell sein, vor allem, wenn die Sachbearbeiter nochmals in unterschiedliche Fachabteilungen (Hausrat, Brandschutz, KFZ...) der Versicherung eingesetzt werden.\\
Es könnte angemessen sein, die 20 Sacharbeiter nochmals zu unterteilen, die dann durch eine Person jeweils vertreten werden.