\subsection*{a)}


Bei \textbf{PMD}\footnote{
    \url{https://pmd.github.io}, abgerufen 1.05.2024
} handelt es sich um ein Tool zur Durchführung \textbf{werkzeuggestützter Analyse} von Java-Quellcode.
Mit solch einem \textit{Mess Detector} lassen sich \textbf{typische Defekte} in Quellcode ausmachen, die u.a. in der Implementierungsphase entstehen.\\
Es handelt sich also um eine \textbf{analytische} Qualitätsmaßnahme, durch die die Codequalität und damit u.a. die \textbf{Wartbarkeit}\footnote{
ein Qualitätsmerkmal nach ISO 9126; daneben gibt es noch die Qualitätsmerkmale \textit{Benutzbarkeit}, \textit{Effizienz}, \textit{Funktionalität}, \textit{Zuverlässigkeit}, \textit{Portabilität} (vgl.~\cite[Abb. 1.2, 3]{Wed09c} sowie \cite[463 f.]{Bal08})
} des Codes verbessert wird.