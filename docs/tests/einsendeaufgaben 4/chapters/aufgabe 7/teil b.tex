\subsection*{b)}


Eine \textbf{Anweisungsüberdeckung} (\textit{statement coverage}) ist ein Kriterium bei strukturorientierten Testverfahren (\textit{White-Box Tests}) zur Angabe, wieviele der im Programm vorhandenen Anweisungen durch die Tests ausgeführt wurden.\\
Eine \textit{vollständige} Anweisungsüberdeckung wird erreicht, wenn jede Anweisung mindestens einmal ausgeführt wurde.\\

\noindent
Um in dem geg. Quellcode eine vollständige Anweisungsüberdeckung zu erreichen, reichen folgende minimale Testdaten:

\begin{itemize}
    \item $i=1$
\end{itemize}