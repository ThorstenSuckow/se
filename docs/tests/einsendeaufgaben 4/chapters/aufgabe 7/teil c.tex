\subsection*{c)}

\section*{Antwort}

Eine \textbf{Boundary-Interior Überdeckung} ist ein Kriterium bei strukturorientierten Testverfahren (\textit{White-Box Tests}) zur Angabe, wieviele der im Programm vorhandenen Anweisungen durch die Tests ausgeführt wurden: Allerdings wird im Gegensatz zu anderen Überdeckungskriterien wie bspw. der  \textit{Anweisungsüberdeckung} gefordert, dass vorhandene Schleifen durch entsprechende Testdaten durchlaufen und \textit{nicht} durchlaufen werden.\\

\noindent
Eine \textit{vollständige} Boundary-Interior Überdeckung wird erreicht, wenn die in dem zu testenden Programm enthaltenen Schleifen mindestens 2-mal\footnote{
nach \textit{Wedemann} die ``schlüssigere`` Variante als die Forderung nach genau einem Durchlauf (vgl.~\cite[52]{Wed09c})
} und  \textit{keinmal} (abweisender Fall)  durchlaufen worden sind.

\noindent
Um in dem geg. Quellcode eine  Boundary-Interior Überdeckung zu erreichen, können folgende minimale Testdaten genutzt werden:

\begin{itemize}
    \item $i=0$ (abweisender Fall)
    \item $i=2$ (Schleife wird 2-mal durchlaufen)
\end{itemize}