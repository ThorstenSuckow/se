\subsection*{a)}

Gruppierung in Äquivalenzklassen nach unterschiedlichen Kriterien der möglichen Eingabewerte (vgl.~\cite[44]{Wed09c}):

\begin{itemize}
    \item \textbf{Anzahl der Zeichen}
    \begin{itemize}
        \item Gültige Äquivalenzklassen:
        \item[] (1a) Zeichenlänge $=5$
        \item Ungültige Äquivalenzklassen
        \item[] (1b) Zeichenlänge $<5$
        \item[] (1c) Zeichenlänge $>5$
    \end{itemize}
    \item \textbf{Art der Zeichen}
    \begin{itemize}
        \item Gültige Äquivalenzklassen:
        \item[] (2a) nur Ziffern
        \item Ungültige Äquivalenzklassen
        \item[] (2b) mindestens ein Zeichen ist keine Ziffer
    \end{itemize}
\end{itemize}


\noindent
Um die Gutfälle bzw. Schlechtfälle zu ermitteln und daraus die Testdaten zu ermitteln, werden alle Gutfälle und die Gutfälle mit den Schlechtfällen kombiniert.\\
Schlechtfälle werden nicht miteinander kombiniert, da man davon ausgeht, dass man selten neue Fehler entdeckt, die Anzahl der Testfälle aber zu stark ansteigt (vgl.~\cite[44]{Wed09c}).

\begin{itemize}
    \item \textbf{Gutfälle}: \textit{(1a) (2a))}, Testdaten: $12345$
    \item \textbf{Schlechtfälle}:
    \begin{itemize}
        \item[] \textit{(1a) (2b)}, Testdaten: $1234a$
        \item[] \textit{(2a) (1b)}, Testdaten: $1234$
        \item[] \textit{(2a) (1c)}, Testdaten: $123456$
    \end{itemize}
\end{itemize}
