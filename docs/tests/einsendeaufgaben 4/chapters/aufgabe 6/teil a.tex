\subsection*{a)}

\section*{Antwort}

\begin{itemize}
    \item \textbf{Gültige Äquivalenzklassen}:
    \begin{itemize}
        \item \textit{1a)} Zeichenlänge = $5$
        \item \textit{1b)} Nur Ziffern $0-9$
    \end{itemize}
    \item \textbf{Ungültige Äquivalenzklassen}:
    \begin{itemize}
        \item \textit{2a)} Zeichenlänge $\neq 5$
        \item \textit{2b)} Mindestens ein Zeichen keine Ziffer
    \end{itemize}
\end{itemize}

\noindent
Um die Gutfälle bzw. Schlechtfälle zu ermitteln und daraus die Testdaten zu ermitteln, werden alle Gutfälle miteinander kombiniert und die Gutfälle mit den Schlechtfällen.\\
Schlechtfälle werden nicht miteinander kombiniert, da man davon ausgeht, dass man selten neue Fehler entdeckt, die Anzahl der Testfälle aber zu stark ansteigt (vgl.~\cite[44]{Wed09c}).

\begin{itemize}
    \item \textbf{Gutfälle}: \textit{1a) 1b)}, Testdaten: $12345$
    \item \textbf{Schlechtfälle}:
    \begin{itemize}
        \item[] \textit{1a) 2b)}, Testdaten: $1234a$
        \item[] \textit{1b) 2a)}, Testdaten: $1234$
    \end{itemize}
\end{itemize}
