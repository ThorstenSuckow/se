\subsection*{b)}


Da ein Testfall für einen \textbf{Gutfall} erstellt werden soll, werden die Schlechtfälle \textit{nicht} in dem Testfall aufgeführt.\\
Weiter ist gefordert, dass der Nutzer am System angemeldet ist, was in dem Testfall entsprechend aufgeführt werden muss.


\begin{itemize}
    \item[] \textbf{Testfall}: Gutfall Anzeige Orte für PLZ, \textbf{ID}: abc123
    \item[] \textbf{Autor}: [Vorname] [Nachname] [email-adresse]
    \item[] \textbf{Datum}: 01.01.1970
    \item[] \textbf{Setup}: Nutzer ist angemeldet; PLZ/Orte-Tabelle muss mit den Testdaten gefüllt sein
\end{itemize}

\textbf{Vorgehen}: (s. Tabelle~\ref{tab:testfall})

\begin{table}[]
    \centering
    \setlength{\tabcolsep}{0.5em}
    \def\arraystretch{1.5}
    \begin{tabular}{|c|p{6cm}|p{6cm}|}
        \hline
        \multicolumn{1}{|l|}{\textbf{Lfd. Nr.}} & \multicolumn{1}{c|}{\textbf{Schritt}}        & \textbf{erwartetes Ergebnis}                        \\ \hline
        1 & Anwender klickt ``PLZ Eingabe``-Button & Dialog ``Postleitzahl eingeben`` öffnet sich\\ \hline
        2 & Eingabe $12345$ in PLZ-Feld, Klick ``OK`` & Schliessen des ``Postleitzahl eingeben``-Dialogs und Anzeige der zugehörigen Orte zu der PLZ\\ \hline
    \end{tabular}
    \caption[]{}
    \label{tab:testfall}
\end{table}

\noindent

\begin{itemize}
    \item[]  \textbf{Zugrundeliegende Anforderungen}:
    \item[] [Verweis auf Anwendungsfall]
    \item[] [Verweis auf GUI-Entwurf]
    \item[] [Verweis auf Geschäftsregeln / Data Dictionary]
\end{itemize}