
\section*{Antwort}

Dem Mitarbeiter sollte ein \textbf{manuelles Prüfverfahren} in Form einer \textbf{Inspektion} vorgeschlagen werden, da Modelle (und Dokumente) nicht getestet werden können (\vgl.~\cite[10]{Wed09c}). Dies ist zwar sehr aufwändig, allerdings handelt es sich bei dem Datenbankschema um ein \textbf{Ergebnis aus der Analyse} und ist damit  Teil des umgesetzten \textbf{Fachkonzepts}. Der Datenbank kommt i.d.R. eine tragende Rolle zu, es handelt sich also um ein wichtige Komponente des Gesamtsystems, bei denen folglich eine \textbf{Inspektion} unerlässlich ist (vgl.~\cite[16]{Wed09c}).\\
Sollte das Domänenmodell als Datenbankschema fehlerhaft umgesetzt worden sein, können zu spät entdeckte Fehler hohen wirtschaftlichen Schaden verursachen, bspw. durch den anfallenden Aufwand einer Korrektur in einer späten Phase (vgl.~\cite[484 ff.]{Bal08}).