\subsection*{a)}

Ein \textbf{Qualitätssystem}\footnote{
    auch\textit{Qualitätsmodell} bei \cite[461 ff.]{Bal08}
} gibt eine Definition des Begriffs Qualität, seiner Merkmale und deren Teilmerkmale vor, damit in einem (Software-)Entwicklungsprozess alle (Projekt-)Beteiligten dasselbe unter dem Begriff ``Qualität`` - angewendet auf den Prozess und das dadurch entstehende Produkt - verstehen. Ein solches System wird auch als \textbf{FCM-Modell} bezeichnet (\textit{factor-criteria-metrics-model}) (vgl.~\cite[462]{Bal08}).\\
Qualitätssysteme erlauben es darüber hinaus, über Messvorschriften (\Textit{Metriken}) Qualität (objektiv) zu messen.\\

\noindent
Es gibt verschiedene Normen und Standards für Qualitätssysteme.\\
Der Kurs verwendet u.a. den Standard ISO 9126\footnote{
    Grundlage des Standards \textbf{ISO/IEC 25000:2014}, s. \url{https://www.iso.org/standard/64764.html}, abgerufen 11.06.2024
} zur Definition von Qualität in einem Softwareprodukt (vgl.~\cite[Abb. 1.2]{Wed09c}).