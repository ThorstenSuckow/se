
Wenn der Kunde dem Lieferanten voll vertraut darf davon ausgegangen werden, dass der Kunde dem Lieferanten auch in der Hinsicht vertraut, als dass das gelieferte Produkt die von dem Kunden gesetzten Qualitätsansprüche erfüllt.\\
Trotzdem sollte dem Kunden zu folgendem geraten werden:
\begin{itemize}
    \item die in der Systemtest-Phase entstandenen \textbf{Testprotokolle} und durchgeführten \textbf{Testfälle} sollten vom Kunden zumindest  durchgesehen werden\footnote{
        die \textbf{stichprobenartig Überprüfung} würde hingegen wieder zu einem \textbf{Abnahmetest} führen, der aufgrund des ``großen Vertrauens`` nicht angedacht ist
    }, bevor in den Wirkbetrieb übergegangen wird:\\
    So kann festgestellt werden, ob Tests für Anwendungsfälle fehlen, die für den Erfolg der Software kritisch sind.\\
    Auch sollten die durchgeführten \textbf{Massentests} überprüft werden, damit bekannt ist, ob die Software mit dem geforderten / erwarteten Daten- und Anfragevolumen zurechtkommt.
    \item In einem \textbf{Verbundtest} sollte - sofern erforderlich - durch den Kunden festgestellt werden, ob das Produkt mit der bestehenden Infrastruktur (Hardware und Software) zusammenarbeitet.
    \item Die \textbf{Akzeptanz} des neuen Systems durch die Anwender des Kunden sollte entweder in einem \textbf{Probebetrieb} festgestellt werden, oder die Anwender sollten in \textbf{Schulungen} an das neue System herangeführt werden.
\end{itemize}