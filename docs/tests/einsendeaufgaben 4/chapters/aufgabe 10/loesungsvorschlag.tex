


\begin{itemize}
    \item \textbf{a)} Die Frage, ob das System für das geforderte Volumen geeignet ist, wird im \textbf{Massentest} beantwortet, der ein Teil des \textbf{Systemtests} ist
    \item \textbf{b)} Was mit dem System passiert, wenn die spezifizierten Grenzen überschritten werden, wird im \textbf{Stresstest} überprüft, der ebenfalls Teil des \textbf{Systemtests} ist\footnote{
    ein weiterer Test, der die \textit{Performance} des Systems überprüft und Teil des Systemtests ist, ist der \textbf{Lasttest}, der das System \textit{an} den Grenzen der geforderten Spezifikationen testet - nicht nur bzgl. des Ressourcenverbrauchs, sondern auch um zu ermittteln, wie sich das System bspw. bei Ausfällen von Partnersystemen oder beim Import fehlerhafter Daten verhält (vgl.~\cite[63]{Wed09c})
    }.
    \item \textbf{c)} Ob die entwickelte Software korrekt mit anderer vom Kunden eingesetzter Software zusammenarbeitet und um gegenseitige negative Beeinflussungen im \textbf{Wirkbetrieb} zu vermeiden, wird der \textbf{Verbundtest} durchgeführt, der Teil des \textbf{Abnahmetests} ist und diese Frage beantwortet.
    Dies kann bedeuten, dass in speziellen Rechenzentrum des Kunden die Inbetriebnahme aller geänderten (und der dafür nötigen existierenden) Systeme stattfindet (vgl.~\cite[65]{Wed09c}); in der Automobilindustrie bedeuten Verbundtests aber auch, dass bei der Entwicklung von Kraftfahrzeugen eine Software (bspw. Assistenzsysteme) im Zusammenspiel mit der Hardware (bspw. Steuerungselektronik) getestet wird.
    In jedem dieser Fälle  soll eine Situation nachgestellt werden, die später dem Wirkbetrieb entspricht.
\end{itemize}