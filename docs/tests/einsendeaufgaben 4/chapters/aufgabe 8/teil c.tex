\subsection*{c)}



\textbf{Richtig}\\
Da es sich bei den überschriebenen Methoden um eine neue Implementierung handelt, muss diese auch getestet werden.\\

\noindent
Gleiches gilt im übrigen für die öffentlichen Methoden, die von der abgeleiteten Klasse geerbt werden: Auch für diese sollten Testfälle bestehen, damit das fehlerfreie Zusammenspiel abgeleiteter und überschriebener Methoden garantiert werden kann und auf Defekte, die durch Änderungen in der abgeleiteten Klasse entstehen können, sofort gefunden werden.\\
Im Folgenden Beispiel wird die Methode \code{id(x: int): int} in der abgeleiteten Klasse verwendet und kann deshalb als öffentliche Methode auch über Objekte von \code{B} aufgerufen werden.\\

\begin{minted}{java}

    class A {
        public int id(int x) {
            return x;
        }
    }

    class B extends A {
    }
\end{minted}

\noindent
Es ist davon auszugehen, dass entsprechende Tests in $A$ für die Methode geschrieben worden sind.\\
Da im Programm auch Objekte von $B$ diese Methode aufrufen, muss sichergestellt sein, dass diese Methode für Objekte von $B$ dasselbe Verhalten zeigen.
Sollten zwischenzeitlich Änderungen im Code entstehen, die mit dem getesteten Verhalten nicht übereinstimmen, können sonst Defekte entstehen. \textit{Binder} listet in \cite[Table 17.2, 843]{Bin99} einiger solcher Defekte auf.

