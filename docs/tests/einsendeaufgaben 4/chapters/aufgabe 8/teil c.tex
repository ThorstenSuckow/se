\subsection*{c)}



\textbf{Richtig}\\
Da es sich bei den überschriebenen Methoden um eine neue Implementierung handelt, muss diese auch getestet werden.\\

\noindent
Gleiches gilt im übrigen für die öffentlichen Methoden, die von der abgeleiteten Klasse geerbt werden: Auch für diese sollten Testfälle bestehen, damit das fehlerfreie Zusammenspiel abgeleiteter und überschriebener Methoden garantiert werden kann und auf Defekte, die durch Änderungen in der abgeleiteten Klasse entstehen können, sofort gefunden werden.\\
Im Folgenden Beispiel wird die Methode \code{y(): int} in der abgeleiteten Klasse überschrieben. \code{x()} und \code{z()} können als öffentliche Methoden über Objekte von \code{B} aufgerufen werden.\\

\begin{minted}{java}
class A {
    public int z() {
        return x() + y();
    }
    public int x() {
        return 1;
    }
    public int y() {
        return 2;
    }
}

class B extends A {
    @Override
    public int y() {
        return 3;
    }
}
\end{minted}

\noindent
Werden \code{x()} und \code{z()} nicht für Objekte der Klasse \code{B} getestet, wird unter Umständen nicht festgestellt, dass Aufrufe von \code{B::z()} nicht mehr \code{1+2} zurückliefern, sondern \code{1+3}. (vgl. ``\textit{naughty children}``,~\cite[Table 17.2, 843]{Bin99}).

