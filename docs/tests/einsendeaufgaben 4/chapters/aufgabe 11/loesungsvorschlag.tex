
\section*{Antwort}

\begin{itemize}
    \item \textbf{a)} \textbf{Falsch} - Testprotokolle dienen zur Protokollierung der durchgeführten Testfälle, die den Testablauf beschreiben.
    In den Protokollen werden die durchgeführten Testfälle und deren Ergebnisse vermerkt, aber nicht die schrittweise durchgeführten Aktionen.
    \item \textbf{b)} \textbf{Falsch} - bei dem QS-Management handelt es sich um einen komplexen Prozess.
Zur Unterstützung der Workflows werden zur Fehlerberichterstattung  spezielle Tools verwendet (\textit{Ticketsystem}, \textit{Bugtracker}), die bei der Erfassung der Fehler helfen, eine ausreichende Beschreibung der Situation zum Reproduzieren des Fehlers ermöglichen und über entsprechende Funktionen Fehlerberichte an die zuständigen Personen weiterleiten.
In größeren \textit{agilen} Projekten kann eine solche Person bspw. der \textit{Product Owner} sein, der über den weiteren Verlauf der Fehlerkorrektur entscheidet\footnote{
    der dann die durch \textit{Wedemann} beschriebene Rolle des \textit{Fehlermanagers} hat (vgl.~\cite[75]{Wed09c})
}, was bspw. über eine \textit{Priorität} ausgedrückt werden kann (vgl.~\cite[108]{Pic08}).

\end{itemize}