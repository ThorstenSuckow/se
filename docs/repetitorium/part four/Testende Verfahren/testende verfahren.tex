\section{Testende Verfahren}
Bei \textbf{testenden Verfahren} wird ein \textbf{Testgegenstand} in Betrieb genommen und überprüft, ob er gemäß den Spezifikationen korrekt funktioniert.\\

\noindent
Ein Test setzt ein \textbf{lauffähiges System} voraus, weshalb ein Test meistens auf Software beschränkt ist\footnote{
\textit{Wedemann} weist auf die Möglichkeit hin, auch Modelle zu testen: Dies passiert in Systemen, die vollständig formal modelliert werden, was bei der Entwicklung für Embeddd Systems vorkommt (vgl.~\cite[41]{Wed09c})
}.\\

\noindent
Tests sollen nach Möglichkeit so strukturiert sein, dass
\begin{itemize}
    \item mit möglichst wenig Aufwand möglichst viele Defekte identifiziert werden können
    \item möglichst \textit{vollständig} getestet wird
\end{itemize}


\subsection*{Vollständigkeit von Tests}
Es muss zunächst geklärt werden, was mit \textbf{Vollständigkeit von Tests} gemeint ist.\\
\textit{Wedemann} führt aus, dass die Vollständigkeit von Tests anhand von zwei Hauptkriterien beschrieben werden kann:

\begin{itemize}
    \item \textbf{Funktionsorientierter Test (Black-Box-Test)}: Ein Test kann als \textbf{vollständig} betrachtet werden, wenn die \textbf{gesamte Funktionalität} im Test erprobt worden ist.
    \item[] Diese Art von Test wird als \textbf{funktionsorientierter} bzw. \textbf{Black-Box-Test} bezeichnet.
    \item[] Bei dieser Art von Tests ist es möglich, dass nicht alle Codeteile ausgeführt werden, weil sie
    \begin{itemize}
        \item nicht erreichbar sind
        \item nicht spezifiziert sind, aber technisch notwendig
        \item fehlerhafterweise überflüssig sind
    \end{itemize}
    \noindent
    Es könnte kritisch sein, wenn diese Codeteile ungetestet bleiben.\\
    Auch Tests, die sich auf nicht-funktionale Anforderungen beziehen (etwa Performance), werden meistens ohne Berücksichtigung auf die \textit{innere Struktur} durchgeführt und deswegen in die Reihe der Black-Box-Tests eingereiht - auch, wenn sie nicht funktionsorientiert sind.
    \item \textbf{Strukturorientierter Test (White-Box-Test)}: Ein Test ist \textbf{strukturorientiert}, wenn seine Vollständigkeit anhand der Codeüberdeckung definiert wird.
\end{itemize}

\vspace{2mm}
\begin{tcolorbox}[title=Strukturorientierte und funktionsorientierte Tests]
    Ein Test ist \textbf{strukturorientiert}, wenn seine Vollständigkeit anhand der Codeüberdeckung definiert wird.  \textbf{Funktionsorientierte Tests} testen die gesamte Funktionalität ohne Berücksichtigung der inneren Struktur.
\end{tcolorbox}
\vspace{2mm}

\noindent
Alle Verfahren sind wichtig: Ein rein \textbf{strukturorientierter Test} wäre nicht ausreichend, da \textit{fehlende Funktionalität} nicht bemerkt werden würde.\\
In der Praxis wird meist \textbf{funktionsorientiert} getestet.\\

\noindent
Bei hohen Anforderungen an die Korrektheit der Tests werden beide Vollständigkeitskriterien angewandt (vgl.~\cite[42]{Wed09c}).
