\section{Strukturorientierte Testverfahren}
\begin{tcolorbox}
Bei \textbf{strukturorientierten Testverfahren}, auch \textbf{White-Box-Tests} genannt, wird zur \textbf{Konstruktion der Testfälle} als auch zur \textbf{Bestimmung der Vollständigkeit} der Tests der Quellcode herangezogen.
\end{tcolorbox}

\noindent
Dadurch soll sichergestellt werden, dass jede existierende Codezeile in Tests ausgeführt worden ist.\\
Wird Code in Tests nicht ausgeführt, ist nicht bekannt, ob dieser funktioniert oder nicht - was auch darauf hinweisen könnte, dass der Code nicht erreichbar und damit überflüssig ist, oder ob es sich um einen Defekt handelt.

\subsubsection*{Überdeckungskriterien}
Unter \textbf{vollständiger Überdeckung} des Quellcodes versteht man häufig, dass jede Zeile des Quellcodes einmal ausgeführt wird.\\
Allerdings kann die Vollständigkeit von Tests auch dadurch charakterisiert werden dass jede \textbf{Verzweigung} mindestens einmal durchlaufen werden muss.\\
Im Folgenden beschränkt sich \textit{Wedemann} auf Kriterien, die zur Definition den \textbf{Kontrollfluss} heranziehen (vgl.~\cite[49 f.]{Wed09c}); für diese existieren auch eine Vielzahl von Werkzeugen\footnote{
\textit{Wedemann} weist ebenda auf \textbf{datenflussorientierte Testverfahren}, die sich auf die vollständige Nutzung von Daten in Tests beziehen. Zum Zeitpunkt der Veröffentlichung von \cite{Wed09c} (2009) gibt er an, dass diese zwar vielversprechend sein, es aber kaum Werkzeuge dafür gäbe.
}.

\subsubsection*{Einsatz}
Strukturorientierte Testverfahren werden i.d.R. im \textbf{Modultest} eingesetzt (s. Abschnitt~\ref{sec:v-modell}), teilweise aber auch im Integrationstests.
Bei System- oder Abnahmetests werden eher \textbf{Black-Box-Tests} genutzt (vgl.~\cite[50]{Wed09c}).

\subsubsection*{Werkzeuge}
Für strukturorientierte Tests ist der Einsatz von Werkzeugen notwendig, da eine händische Kontrolle viel zu aufwändig wäre.
Für die praktische Einsetzbarkeit ist außerdem eine gute Integration in die Entwicklungsumgebung wichtig.

\subsection{Anweisungsüberdeckung}
Das einfachste Kriterium ist die \textbf{Anweisungsüberdeckung} (\textit{statement coverage}).

\begin{tcolorbox}[title=Anweisungsüberdeckung]
    Liegt eine \textbf{Anweisungsüberdeckung} vor, wurde im Test \textit{jede} Anweisung mindestens einmal ausgeführt.\\
    Eine \textbf{Anweisungsüberdeckung} liegt dann vor, wenn \textit{jeder} Knoten eines \textbf{Kontrollflussgraphen} mindestens einmal ausgeführt worden ist.
\end{tcolorbox}

\subsubsection*{Beispiel}
Als Beispiel sei der Kontrollflussgraph der Methode \code{countVowels()} gegeben (s. Abbildung~\ref{fig:kontrollflussgraph}).\\
Jeder Knoten wird mit den Testdaten \code{txt="A"} mindestens einmal ausgeführt.

\subsubsection*{Effektivität}
Untersuchungen, die für verschiedene fehlerhafte Methoden die Tests ausschließlich nach Überdeckungskriterien konstruiert haben, haben festgestellt, dass die Tests, die die \textbf{Anweisungsüberdeckung} erfüllen, ca. 15\% der Fehler finden\footnote{
\textit{Wedemann} gibt leider keine Quelle an, vgl.~\cite[51]{Wed09c}
}.

\subsubsection*{Werkzeugunterstützung}
Für die meisten imperativen Sprachen gibt es kommerzielle und freie Tools zur Messung der Anweisungsüberdeckung.

\subsubsection*{Einsatzgebiet}
\textit{Wedemann} gibt an, dass die Messung der Anweisungsüberdeckung das verbreiteste Verfahren darstellt, es aber alleine nicht ausreicht, und deshalb im Modultest gemeinsam mit \textbf{funktionsorientierten Testverfahren} (\textit{Grey-Box-Test}) eingesetzt wird (vgl.~\cite[51]{Wed09c}).