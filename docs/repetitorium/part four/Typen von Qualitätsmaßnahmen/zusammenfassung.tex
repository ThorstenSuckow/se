\section{Zusammenfassung}

\begin{itemize}
    \item Bei \textbf{qualitätssichernden Maßnahmen} können \textbf{konstruktive} und \textbf{analytische Maßnahmen} unterschieden werden
    \item \textbf{Konstruktive Maßnahmen} beinhalten \textbf{organisatorische Maßnahmen} (bspw. Standards)
    sowie \textbf{technische Maßnahmen} (bspw. Einsatz von Werkzeugen): Diese Maßnahmen sollen das \textit{Entstehen von Fehlern} verhindern
    \item \textbf{Analytische Maßnahmen} helfen dabei, entstandene Fehler zu finden: Hierzu gehören \textbf{Tests}, aber auch analysierende manuelle oder werkzeuggestützte Maßnahmen
    \item das \textbf{V-Modell} beschreibt qualitätssichernde Maßnahmen in ihrer Einordung in den Entwicklungsprozess: Klassen und Module werden im \textbf{Modultest} geprüft, das Zusammenspiel der Teile im \textbf{Integrationstest}, die \textit{gesamte Anwendung} im \textbf{Systemtest} und im Anschluss vom Kunden im \textbf{Abnahmetest}
\end{itemize}