\section{Zusammenfassung}

\begin{itemize}
    \item \textbf{Qualität} kann mit Hilfe von \textbf{Qualitätssystemen} definiert werden.
    \item \textbf{Qualitätssysteme} teilen \textbf{Qualität} in \textbf{Qualitätsmerkmale} ein, die in \textbf{Qualitätsteilmerkmale} eingeteilt werden.
    \item \textbf{Qualitätsindikatoren} definieren die \textbf{Qualitätsteilmerkmale}: Sie repräsentieren \textbf{Messvorschriften} (\textit{Metriken}), die eine Bestimmung von \textbf{Kennzahlen} und deren \textbf{Sollwerte} beschreiben.
    \item Als Hilfsmittel zur Beschreibung von Qualitätsteilmerkmalen dienen \textbf{Qualitäts-Szenarien}.
    \item \textbf{Vorlagen} für \textbf{Qualitätssysteme} finden sich in den Standards, bspw. in der Norm \textbf{ISO/EC 25000}.
    \item I.d.R. werden aus einem Qualitätssystem \textbf{Teilmerkmale} ausgesucht, geeignete Metriken gesucht und die \textit{relative Wichtigkeit} der Teilmerkmale definiert.
\end{itemize}