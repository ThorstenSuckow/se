\section{Zusammenfassung}

\begin{itemize}
    \item die \textbf{Qualitätssicherung} bei der Softwareentwicklung ist ein komplexer Prozess, der u.a. verschiedene Aufgaben wie Tests, Reviews und werkzeuggestützte Analysen beinhaltet
    \item diese Maßnahmen werden im \textbf{Qualitätssicherungsplan} und \textbf{Testplan} beschrieben
    \item je nach Art und Größe  der Aufgabe werden diese Maßnahmen durch die Entwickler, dedizierte Tester oder Testteams durchgeführt
    \item qualitätssichernde Maßnahmen müssen \textbf{dokumentiert} werden:
    \begin{itemize}
        \item \textbf{Testfälle} beschreiben die Durchführung der Tests
        \item \textbf{Testprotokolle} halten die Durchführung der Tests fest
        \item \textbf{Fehlerberichte} beschreiben die aufgetretenen Fehler
    \end{itemize}
    \item die Bearbeitung von Fehlermeldungen folgt einem komplexen \textbf{Workflow}, weshalb dafür spezielle Werkzeuge zum Einsatz kommen
    \item der gesamte Testprozess wird mit \textbf{Kennzahlen} verfolgt, wobei die wichtigsten Kennzahlen die \textbf{Anzahl der Fehlerberichte} und die \textbf{Abarbeitung der Testfälle} sind
\end{itemize}