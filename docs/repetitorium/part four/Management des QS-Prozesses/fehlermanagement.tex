\section{Fehlermanagement}
\textbf{Fehlerberichte} (\textit{Bug-Tickets}) werden von verschiedenen Projektbeteiligten genutzt:

\begin{itemize}
    \item Tester, Entwickler und Kunden finden Fehler und  möchten diese dem Entwicklungsteam mitteilen
    \item Entwickler benötigen eine möglichst präzise Beschreibung der Fehler, um Fehlersituationen nachvollziehen und den Defekt beheben zu können
    \item die Projektleitung will verhindern, dass Fehlermeldungen verloren gehen, die Bearbeitung von Fehlermeldungen steuern und den Überblick über Fehlerberichte und den Bearbeitungsstand behalten
\end{itemize}

\subsubsection*{Fehlerbericht}

Ein Fehlerbericht sollte mindestens folgende Informationen beinhalten (vgl.~\cite[75]{Wed09c}):

\begin{itemize}
    \item eindeutige Nummer
    \item Kurztitel
    \item Autor
    \item Priorität
    \item betroffenes System
    \item betroffene Version
    \item ausführliche Beschreibung
    \item Bearbeiter
    \item Bearbeitungsvermerke mit Datum
    \item aktueller Status der Bearbeitung (etwa: offen, zugewiesen, nicht nachvollziehbar, in Arbeit, im Test, Wiedervorlage, gelöst, fertig)
    \item Anlagen, bspw. Screenshots
\end{itemize}

\noindent
Solche Berichte werden üblicherweise mit Hilfe von \textbf{Fehlermanagementwerkzeugen} (\textit{Bugtracking-Tools}, bspw. \textbf{Jira}\footnote{
\url{https://www.atlassian.com/de/software/jira}, abgerufen 01.06.2024
}) gepflegt.

\subsubsection*{Workflow}
Die \textbf{Bearbeitung von Fehlerberichten} erfolgt i.d.R. durch viele Personen, wobei der Ablauf der Bearbeitungen bestimmten \textbf{Workflows} folgt, bspw. wie folgt:

\begin{enumerate}
    \item Ein Fehler tritt beim Kunden im Abnahmetest auf und wird in einem Fehlerbericht dokumentiert
    \item der Fehlerbericht wird vom Fehlermanager des Entwicklungsteams geprüft und einem Entwickler zur Prüfung weitergeleitet
    \item der Entwickler versucht, die Fehlermeldung nachzuvollziehen; im Erfolgsfall behebt er den Defekt und leitet dies an den Fehlermanager weiter, der den Fehlerbericht einem Tester zum Test gibt
    \item treten erneut Fehler auf, bekommt ein Entwickler den Fehlerbericht erneut, bis die Bearbeitung erfolgreich war
    \item der Fehlerbericht wird an den Liefermanager weitergegeben, damit er ihn bei der nächsten Lieferung für den Abnahmetest berücksichtigen kann
\end{enumerate}

\noindent
Der Ablauf solcher Workflows wird in größeren Projekten normalerweise im Managementplan genau definiert (vgl.~\cite[76]{Wed09c}).
