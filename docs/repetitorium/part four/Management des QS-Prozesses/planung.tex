\section{Planung}
Für jedes Software-Projekt muss festgelegt werden, welche qualitätssichernde Maßnahme \textbf{wie}, \textbf{wann} und \textbf{von wem} durchgeführt werden soll.\\

\noindent
Die verwendeten \textbf{Qualitätssicherungsmaßnahmen} und die Art ihrer \textbf{Dokumentation} werden in einem \textbf{Qualitätssicherungsplan} beschrieben.\\
Dazu gehören die \textbf{Tests} sowie \textbf{alle Maßnahmen} wie

\begin{itemize}
    \item Analysen wie Reviews
    \item werkzeuggestützte Analysen
    \item konstruktive Maßnahmen
\end{itemize}

\noindent
Als Vorlage für einen \textbf{Qualitätssicherungsplan} kann die  Norm \textbf{IEEE 730}\footnote{
\url{https://standards.ieee.org/ieee/730/5284/}, abgerufen 01.06.2024
} oder das \textbf{V-Modell XT} herangezogen werden.\\
Das \textbf{Wie}, also das Vorgehen beim Test, wird detailliert in einem \textbf{Testplan} dargestellt (bspw. \textbf{IEEE 829}\footnote{
\url{https://de.wikipedia.org/wiki/Software_Test_Documentation}, abgerufen 01.06.2024
}).\\
Das \textbf{konkrete Vorgehen} beim Test wird in \textbf{Testfällen definiert}.\\
\textbf{Von wem} welche der Maßnahmen durchgeführt wird, wird in einem getrennten Zeitplan oder im generellen Projektplan  geplant.

\subsubsection*{Pläne in konkreten Projekten}
In vielen Unternehmen ist die Art der qualitätssichernden Maßnahmen in unternehmensweit geltenden Standards geregelt.\\
In diesem Fall kann auf diese Standards im Qualitätssicherungsplan bzw. im Tesplan verwiesen werden.\\
Trotzdem muss festgelegt werden, \textbf{welche} der Maßnahmen durchgeführt werden sollen.\\
In vielen kleineren Projekten soll oftmals mit möglichst wenig Dokumentation gearbeitet werden - auch in diesem Fall müssen die Fragen nach dem \textit{was}, \textit{wie}, \textit{wann}, und \textit{von wem} geregelt werden, damit die Beteiligten dasselbe unter einer Maßnahme verstehen.\\
Solche Festlegungen können als Teil des Managementplans erfolgen oder im einfachsten Fall als Punkteliste.\\
Wichtig ist auch, dass andere überprüfen können, ob ergriffene Maßnahmen vollständig und zweckmäßig sind.