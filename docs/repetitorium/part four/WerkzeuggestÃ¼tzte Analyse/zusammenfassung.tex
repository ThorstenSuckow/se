\section{Zusammenfassung}

\begin{itemize}
    \item Werkzeuge zur \textbf{statischen Analyse von Sourcecode} werden eingesetzt, um
    \begin{itemize}
        \item die Einhaltung von \textbf{Programmierrichtlinien} zu überprüfen
        \item \textbf{typische Defekte} anhand vorgegebener Muster zu finden
        \item Ablauffehler mit Hilfe der \textbf{Datenflussanomalieanalyse} und der \textbf{abstrakten Interpretation} zu erkennen
        \item die Wartbarkeit mit Metriken zu beurteilen
    \end{itemize}
    \item In bestimmten Bereichen ist der Einsatz solcher Werkzeuge zum Industriestandard geworden, bspw. im Bereich Automotive
    \item mittels der werkzeuggestützten Analyse können Defekte entdeckt werden, aber eine Aussage über die \textbf{Lauffähigkeit} oder \textbf{Brauchbarkeit} kann nicht angegeben werden, da die \textbf{Funktionalität} \textit{nicht} analysiert wird
    \item der Einsatz der Werkzeuge muss sorgfältig vorbereitet werden
    \item die Regelsätze der Werkzeuge müssen an die Gegebenheiten angepasst werden
    \item der Entwicklungsprozess muss den Einsatz der Werkzeuge berücksichtigen
    \item der Einsatz solcher Werkzeuge kann eine sehr kostengünstige und effektive Maßnahme sein
\end{itemize}