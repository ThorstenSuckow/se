\section{Typische Defekte}\label{sec:typische-defekte}

Bestimmte Defekte lassen sich durch genaues Ansehen des Quellcodes entdecken. Dazu gehören bspw.

\begin{minted}{java}
    public void doSomething() {
        try {
            FileInputStream fis = new FileInputStream("/dev/null");
        } catch(IOException e) {
            // leerer catch-Block
        }
    }

    void foo(int x) {
        if (x == 0){
            // leerer block
        }
    }

    public class Bar() {
        public int hashcode() { // eher "hashCode"?
            ...
        }
    }

    for (int i = 0; i < 10; i++) {
        for (int j = 0; j < 10; i++) { // j++?
            ...
        }
    }


    if (x = 0) { // Zuweisung? Vergleich?
        ...
    }
\end{minted}

\subsection*{Potentielle Fehler und schlechte Wartbarkeit}
Die o.a. Beispiele können zu korrektem Laufzeitverhalten führen, meistens handelt es sich aber um Defekte.\\
Selbst bei korrektem Laufzeitverhalten handelt es sich um potentielle Schwachpunkte, die die Wartbarkeit erschweren - folglich muss solcher Code als Defekt kategorisiert werden.

\subsection*{Werkzeuge}
Für Java gibt es das Tool \textit{PMD}\footnote{
    \url{https://pmd.github.io}, abgerufen 24.05.2024
}, für PHP den \textit{PHP Mess Detector}\footnote{
    \url{https://phpmd.org}, abgerufen 24.05.2024
} (\textit{PHPMD}).

\subsection*{Einsatz}
Für den Einsatz dieser Werkzeuge gelten die gleichen Aussagen wie für den Einsatz von Werkzeugen zur Einhaltung von Programmierrichtlinien (s. Abschnitt~\ref{sec:programmierrichtlinien}
).\\
\textit{Wedemann} weist darauf hin, dass es bei diesen Werkzeugen noch wichtiger ist, die Regeln anzupassen, da die ``Standardeinstellungen häufig zu strenge und zum Teil sogar widersprüchliche Regeln beinhalten`` (\cite[31]{Wed09c}).

\subsection*{Pro und Contra}
Der Vorteil ist, dass die Werkzeuge leicht anzuwenden sind und viele Defekte gefunden werden können.
Außerdem hat sich in der Praxis gezeigt, dass auch erfahrene Mitarbeiter bei dem Einsatz der Werkzeuge noch dazulernen können.\\
Bei schlechter Anpassung der Werkzeuge besteht die Gefahr, dass Probleme an Stellen angezeigt werden, an denen keine Defekte vorliegen.
Mitarbeiter neigen dann eher dazu, die Werkzeuge abzulehnen.\\
Für Embedded-Applikationen ist der Einsatz solcher Werkzeuge bewährter Industriestandard (vgl.~\cite[31]{Wed09c}).
