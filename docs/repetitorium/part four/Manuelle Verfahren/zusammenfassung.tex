\section{Zusammenfassung}

\begin{itemize}
    \item \textbf{Manuelle Verfahren} sind wichtige Werkzeuge zur \textbf{Qualitätssicherung} von \textbf{Artefakten} der Softwareentwicklung
    \item sie sind bei \textbf{Dokumenten} und \textbf{Code} unabdingbar - Dokumente lassen sich zudem besser manuell vollständig prüfen
    \item bei \textbf{Programmcode} lassen sich manche Defekte einfacher und damit kostengünstiger finden
    \item je nach Verfahren untersuchen \textit{einzelne} oder \textit{mehrere} \textbf{Gutachter} oder \textit{zusammen in Gruppen}
    \item dabei unterscheiden sich die Verfahren in Umfang und dem Grad an Formalisierung: Je nach Wichtigkeit des \textbf{Prüfgegenstandes} ist eine andere \textbf{Prüfart} angemessen
    \item bei formalisierten Verfahren ist es wichtig, sich an die Abläufe zu halten, damit die Verfahren erfolgreich durchgeführt werden
    \item hilfreiche Werkzeuge sind \textbf{Checklisten}, um \textbf{systematisch} nach Defekten in Prüfgegenständen
    zu suchen
\end{itemize}