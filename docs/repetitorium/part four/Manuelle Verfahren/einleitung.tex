\section{Einleitung}

Viele Produkte der Softwareentwicklung (\textit{Artefakte}) sind nicht testbar, dazu gehören bspw. Dokumente, die Anwendungsfälle beschreiben, das DataDictionary\footnote{
hält fest, in welchem Format welche Daten verarbeitet werden; s. Abschnitt~\ref{fig:sec:datadictionary-und-mengengerust}
} aber auch Klassendiagramme aus Analyse und Entwurf.\\
Wenn sich die Qualität von Artefakten nicht oder nicht ausschließlich durch Testen sicherstellen lässt, kommen \textbf{manuelle Verfahren} zum Einsatz: Mitarbeiter untersuchen den \textbf{Prüfgegenstand} durch \textbf{systematisches Korrekturlesen}.\\

\noindent
Es ist wichtig, Defekte früh im Entwicklungsprozess zu finden.\\
Bei einem falschen Entwurf oder einer falschen Spezifikation fällt der Fehler u.U. erst in einer späten Phase bei den \textbf{Systemtests} durch spezialisierte Tester\footnote{
s. Abschnitt~\ref{sec:v-modell}
} oder bei dem \textbtf{Abnahmetest} durch den Kunden auf.\\
Fehlerhafter Code kann also auch in frühen Phasen durch das \textbf{manuelle Prüfen} von Dokumenten und Modellen verhindert werden, indem andere Personen Prüfgegenstände \textit{systematisch lesen}.\\
Hierbei kommen auch \textbf{Checklisten} zum Einsatz.\\
Gefundene Defekte und Verbesserungsvorschläge werden dem \textbf{Autor} vom \textbf{Prüfer} mitgeteilt.\\

\noindent
Die Funktionsfähigkeit von Software wird i.d.R. durch Tests sichergestellt, aber auch eine manuelle Prüfung kann sinnvoll sein: Manche Defekttypen werden so effektiver gefunden\footnote{
das \textit{Vieraugenprinzip} gilt als effektive präventive Kontrolle, s. \url{https://de.wikipedia.org/wiki/Vier-Augen-Prinzip}, abgerufen 22.05.2024
}.
Außerdem kann nur bei einer Analyse des Quellcodes überprüft werden, ob ein Entwickler den Quellcode auf eine \textbf{wartbare} Art und Weise geschrieben hat, also ob der Code \textbf{lesbar} und \textbf{einfach zu ändern} ist\footnote{
dass die Einhaltung von Coding-Standards im gesamten Quellcode besser automatisierte durch Werkzeuge stattfindet, bleibt hiervon unberührt
}.\\

\noindent
Ein wichtiger Nebeneffekt manueller Prüfverfahren ist, dass sich der Prüfer \textit{intensiv} mit dem Prüfgegenstand \textit{auseinandersetzt}.
Dies wirkt sich positiv bei einem personellen Ausfall oder einer Umstrukturierung aus.\\

\noindent
Manuelle Prüfungen sind recht aufwändig, was sich auch in den Kosten widerspiegelt.
Aus diesem Grund schrecken viele Projektverantwortliche davor zurück. \textit{Wedemann} führt an, dass im Zuge einer \textit{objektiven Argumentation} die Kosten miteinander verglichen werden sollten, die bei der Beseitigung der Defekte in bestimmten Phasen entstehen (vgl.~\cite[16]{Wed09c}).
Auch \textit{Brooks} weist auf die Notwendigkeit hin, Defekte in den relevanten Dokumenten möglichst früh zu erkennen:

\blockquote[{\cite[142]{Bro95}}]{
The most pernicious and subtle bugs are system bugs arising from mismatched assumptions made by the authors or various components. [\ldots] Long before any code exists, the specification must be handed to an outside testing group to be scrutinized for completeness and clarity.
}

\noindent
\textit{Wedemann} schließt aus verschiedenen Quellen (\cite{Rad01}, \cite{Wie02}, \cite{GG93}), dass eine Reduktion der Kosten zur Behebung von Defekten erreicht werden kann, wenn diese Defekte bei einer manuellen Überprüfung in einer frühen Phase gefunden werden.
Weiter stellt er fest,  dass ``bei einer Inspektion ca. 60{\%} aller Defekte gefunden werden`` (~\cite[16]{Wed09c}).\\

\begin{tcolorbox}
Prinzipiell sollten alle Dokumente und Modelle mit manuellen Verfahren geprüft werden, da diese Verfahren das einzig praktikable Verfahren sind (\cite[16]{Wed09c}).\\
Manuell geprüft werden sollten insb.

\begin{itemize}
    \item durch eine \textbf{Inspektion}\footnote{
    oder mindestens einem \textbf{Review} (vgl.~\cite[Aufgabe 3.8, 83]{Wed09c})
    }: wichtige Codeteile, die als \textbf{Vorlage} in Projekten dienen, bzw. denen eine \textbf{entscheidende Rolle} in der Software zukommt
    \item durch angemessene Prüfverfahren: Code, der von \textbf{neuen} oder \textbf{unerfahrenen Mitarbeitern} geschrieben wird
\end{itemize}
\end{tcolorbox}
\vspace{2mm}

\noindent
Der Einsatz von manuellen Verfahren ist unkompliziert, allerdings sollte vor Beginn der Entwicklung überlegt werden, welche manuellen Verfahren durchgeführt werden sollen.
Diese Abläufe sollten im Management-Plan schriftlich fixiert werden, um das Verständnis aller Mitarbeiter zu schärfen.
Es sollte außerdem ausreichend Zeit zur Durchführung der Maßnahmen eingeplant werden und die Durchführung dementsprechend verfolgt und die Ergebnisse protokolliert werden.\\
Wenn Autoren die Maßnahmen als unangenehm empfinden, sollte bei der Durchführung sorgsam mit ihnen umgegangen werden (vgl. \cite[17]{Wed09c}).