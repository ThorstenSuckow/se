\section{Inspektion}
Die \textbf{Inspektion} ist das komplizierteste, aufwändigste, fromalisierteste und \textit{effektivste Verfahren}, um Defekte manuell aufzuspüren (vgl~\cite[18]{Wed09c}).\\
Alle anderen Verfahren basieren auf dem Verfahren der Inspektion.\\

\noindent
Aufgrund ihres Aufwandes kommen Inspektionen in der Praxis i.d.R. nur bei \textit{sehr wichtigen} Prüfgegenständen zum Einsatz.\\

\noindent
Bei der Inspektion wird von \textbf{Gutachtern} die \textbf{Erfüllung der Spezifikation} und die \textbf{Einhaltung von Standards} geprüft.\\
Damit nichts Wichtiges übersehen wird, werden Checklisten verwendet, die nicht unbedingt umfangreich sein müssen.\\
Die Inspektion findet zunächst alleine durch die jeweiligen Gutachter statt.
Die Ergebnisse werden in einer anschließenden \textbf{Inspektionssitzung} besprochen.

\begin{tcolorbox}[colback=white]
    Die Erarbeitung von Möglichkeiten zur Behebung von Defekten ist kein Teil der Inspektion.
\end{tcolorbox}

\subsection{Inspektionsteam}

\begin{itemize}
    \item \textbf{Moderator}: organisiert die Vorgänge der Inspektion, moderiert die Inspektionssitzung. Hat im besten Fall bereits Erfahungen mit Inspektionen gemacht und hat Kenntnisse in Gesprächsmoderation und strukturierter Arbeit mit Gruppen.
    \item \textbf{Autor(en)}: stehen während der Inspektion zur Beantwortung von Fragen zur Verfügung.
    \item \textbf{Inspektoren}: untersuchen den Prüfgegenstand, müssen also zu einer fachlichen Beurteilung in der Lage sein. Sind keine Vorgesetzten der Autoren.
    \item \textbf{Leser}: Einer der Inspektoren nimmt \textbf{während der Inspektionssitzung} die Rolle des \textbf{Lesers} ein, der durch den Prüfgegenstand führt.
\end{itemize}

\noindent
Wegen der Rollenverteilung \textbf{Leser}/\textbf{Inspektor} sollten mindestens zwei Inspektoren beteiligt sein.\\

\noindent
Das \textbf{Protokoll} führt der Moderator bei kleineren Inspektionsgruppen.\\
Bei größeren Gruppen (mehrere Inspektoren bzw. mehrere Autoren) übernimmt ein \textbf{dedizierter Protokollführer} diese Aufgabe.