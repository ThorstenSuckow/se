\section{Vergleich und Einsatz der Verfahren}
\textit{Wedemann} gibt an, dass bei der Inspektion ca. 60\% der Defekte gefunden werden, und dass die Inspektion das \textbf{effizienteste Verfahren} darstellt: Die Inspektion ist doppelt so effizient wie andere manuelle Verfahren (vgl.~\cite[25]{Wed09c}).\\

\noindent
Dabei sind Inspektionen aber auch sehr zeitaufwändig: Bei umfangreichen Dokumenten ist eine Inspektion nicht durchführbar, weshalb man in solchen Fällen auf das \textbf{Review} zurückgreift.\\
\textbf{Walkthroughs} sind nur sehr eingeschränkt nutzbar und eignen sich vor allem, um Ergebnisse einem größeren Publikum vorzustellen (bspw. GUI-Entwürfe).\\

\noindent
Zur \textbf{manuellen Prüfung von Code} ist ebenfalls die Inspektion das Mittel der Wahl, aber auch hier wird in der Praxis meist das \textbf{Review in Kommentartechnik} eingesetzt.
Die \textbf{Solo-Inspektion} wird seltener eingesetzt, obwohl bei unwesentlich größerem Aufwand der Nutzen deutlich höher ist: Bei einem \textbf{Review in Kommentartechnik} muss recht viel Aufwand zur Klärung von Missverständnissen und zur Übermittlung gefundener Defekte aufgewendet werden.\\

\noindent
\textbf{Pair Programming} wird überwiegend in \textbf{Extreme Programming} (\cite{BA04}) eingesetzt. Die Wirksamkeit ist umstritten\footnote{
\textit{Wedemann} verweist hier auf die Ergebnisse von \cite{MPT05}
} und eher davon abhängig, ob sich Programmierer gut verstehen (vgl.~\cite[25]{Wed09c}). Allerdings erweist sich Pair Programming als nützlich, wenn Aufgaben nicht weiter aufgeteilt werden können oder zwei Kollegen in zwei unterschiedlichen Bereichen Wissen besitzen, das zur Erledigung der Aufgabe gemeinsam benötigt wird.\\

\noindent
\textit{Wedemann} stellt darüber hinaus fest, dass der Einsatz von manuellen Verfahren schwierig ist, wenn eine Änderung aus vielen kleinen Änderungen in existierenden Artefakten besteht\footnote{
wie etwa bei Wartungsprojekten
}: Dann ist es hilfreich, wenn Gutachter über ein Versionierungssystem zwischen zwei verschiedenen Versionen vergleichen können.