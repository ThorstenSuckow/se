\section{Andere Verfahren}\label{sec:andere-verfahren}

\begin{itemize}
    \item \textbf{Review}:
    \begin{itemize}
        \item es wird ähnlich wie bei der Inspektion vorgegangen, aber weniger formalisiert
        \item der Prüfgegenstand kann auf die Reviewer aufgeteilt werden\footnote{
            bei der Inspektion untersucht jeder Gutachter den gleichen Prüfgegenstand
        }.\\
        \item in der Review-Sitzung gibt es keinen \testbf{Leser}, der Moderator führt informell durch den Prüfgegenstand
        \item i.d.R. werden keine Metriken ermittelt
    \end{itemize}
    \item \textbf{Walkthrough}:
    \begin{itemize}
        \item es gibt keine Vorbereitung und keine Rollenverteilung
        \item der \textbf{Autor} ruft die Gutachter zusammen und trägt den Prüfgegenstand vor, die ihre Meinung dazu äußern
    \end{itemize}
    \item \textbf{Solo-Inspektion ($1:1$-Inspektion)}:
    \begin{itemize}
        \item wird \textit{genauso} durchgeführt wie eine \textbf{Inspektion}
        \item der Moderator ist Inspektor, Leser, Protokollführer
    \end{itemize}
    \item \textbf{Review in Kommentartechnik}:
    \begin{itemize}
        \item der Autor übermittelt einem Reviewer den Prüfgegenstand
        \item der Reviewer sendet Anmerkungen zu dem Prüfgegenstand zurück
    \end{itemize}
    \item \textbf{Pair-Programming}:
    \begin{itemize}
        \item kann als spezielles Verfahren der manuellen Prüfung angesehen werden
        \item zwei Personen arbeiten gleichzeitig an einem Arbeitsplatz.
        \textit{Beck und Andres} charakterisieren die Aufgaben beteiligter Programmierer in \cite[42]{BA04}:
        \begin{itemize}
            \item Keep each other on task.
            \item Brainstorm refinements to the system.
            \item Clarify ideas.
            \item Take initiative wen their partner is stuck, thus lowering frustration.
            \item Hold each other accountable to the team's practice.
        \end{itemize}
    \end{itemize}
\end{itemize}