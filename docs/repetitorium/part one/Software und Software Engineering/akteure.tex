\section{Wer an Software Engineering beteiligt ist}

\begin{itemize}
    \item \textbf{Kunden} investieren Geld in die Entwicklung.
    \item \textbf{Benutzer} arbeiten mit dem System.
    Dazu gehören auch Administratoren, die für den Betrieb der Computer und der darauf laufenden Anwendungen zuständig sind
    \item \textbf{Entwickler} sind an der Erstellung der Software beteiligt, dazu zählen auch Tester, sowie Personen, die für die Dokumentation oder der Erfassung von Anforderungen verantwortlich sind.
    \item \textbf{Manager} treffen organisatorische Entscheidungen, die größere Projekte betreffen können.
     Hierzu zählen auch die Manager auf Kundenseite, wie z.b. die Vorgesetzten der Endanwender
    \item außerdem: Personen, die zu der Entwicklung hinzugezogen werden, wie Rechtsanwälte, Datenschützer\ldots
\end{itemize}\\

\noindent
$\rightarrow$ egal, wer wie an der Entwicklung beteiligt ist, die \textbf{Prinzipien} und \textbf{Methoden} des SE helfen dabei, Vorgänge besser zu verstehen und damit produktiver und zielgerichteter zu arbeiten.

