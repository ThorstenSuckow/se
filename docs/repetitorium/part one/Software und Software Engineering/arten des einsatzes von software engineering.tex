\section{Arten des Einsatzes von Software Engineering}
Aufgrund der unterschiedlichen Art von Software sind gewisse Verfahren nur in einem bestimmten Bereich nützlich.

\subsection{Nähe und Anzahl der Kunden}
Zu unterscheiden ist hier zwischen Individual- und Standardsoftware:

\begin{itemize}
    \item \textbf{Individualsoftware} wird von einem einzelnen Kunden beauftragt, durch den die Anforderungen (i.d.R.) sehr genau bestimmt werden.
    \item \textbf{Standardsoftware} entwickeln bedeutet, einen großen Kreis von Kunden mit unterschiedlichen Anforderungen / Bedürfnissen\footnote{Geduld, Lernbereitschaft, Geld, Wissen, ...} zu bedienen (bspw. Software, die bei der Steuererklärung hilft, Textverarbeitung, Grafikprogramme).
    Die Herausforderung hierbei ist, die Software so zu entwickeln, dass möglichst viele Wünsche / Bedürfnisse ohne direkte Befragung abgedeckt / vorgeahnt sind
\end{itemize}

\subsection{Art der Benutzer / der Benutzung}
Bei \textbf{seltener Benutzung} einer Software sollte darauf geachtet werden, dass die Software robust und intuitiv zu bedienen ist.\\

\noindent
\textbf{Professionelle, geschulte Anwender}, die \textit{häufig} mit einer Software arbeiten, benötigen effektive Werkzeuge, die die Produktivität und Effizienz erhöhen.\\

\noindent
Müssen beide Benutzergruppen abgedeckt werden, wird es schwieriger.\\

\noindent
$\rightarrow$ \textit{Usability Engineering} als Zweig des SE beschäftigt sich u.a. mit diesen Fragen.

\subsection{Größe / Komplexität der Software und des Projektes}
In einem kleineren Projekt mit wenigen Entwicklern sind viele Methoden des SE nicht notwendig und würden den Aufwand unnötig in die Höhe treiben.\\

\noindent
Dennoch dürfen die \textbf{notwendigen Methoden}\footnote{und Werkzeuge} nicht vernachlässigt werden, um die Produktivität nicht zu gefährden.\\

\noindent
In großen Projekten mit vielen Entwicklern sind Methoden und ``Zeremonien`` unabdingbar, damit alle produktiv zusammenarbeiten können.

\subsection{Wie kritisch ist nicht-technisches Domänenwissen?}
Entwickler arbeiten oft in Projekten, in denen sie \textit{keine} fachlichen Experten sind (Flugsicherheit, Medizintechnik,\ldots).\\
Hierbei sind sie auf \textbf{Fachexperten} angewiesen, wobei diese oft garnicht wissen, welche Informationen die Entwickler benötigen $\rightarrow$ es besteht die Gefahr, dass beide Seiten aneinander vorbeireden\footnote{
womit sich schon seit jeher die Fachliteratur beschäftigt, bspw. \textit{Brooks}' 1975 in \textit{Why did the Tower of Babel fail?} (\cite{Bro95}). 2003 greift \textit{Evans} in \cite{Eva03} die Idee einer \textit{Ubiquitous Language} als einheitliches Vokabular für alle Projektbeteiligten auf (``Getting all team members to speak the same language``)
}.\\
Liegt eine komplexe Fachlichkeit vor, ist es notwendig, Methoden des SE zu benutzen, die zur Lösung solcher Probleme entwickelt wurden (Anforderungsmanagement, Analyse; Wahl eines Vorgehensmodells, um Fachexperten in die Entwicklung einzubinden\footnote{
man denke hier bspw. an \textit{Cross-Functional Teams}; s. \url{https://en.wikipedia.org/wiki/Cross-functional_team} - abgerufen 24.03.2024
}).\\

\subsection{Müssen Näherungslösungen verwendet werden?}
Wenn Anforderungen so beschaffen sind, dass Aufgaben nicht vollständig gelöst werden können (bspw. wegen Komplexität), müssen die Probleme mit \textbf{Näherungsverfahren} gelöst werden.\\

\subsection{Wie kritisch ist Effizienz?}
In bestimmten Umgebungen muss bei der Implementierung auch auf Effizienz hinsichtlich Laufzeit und Speicherverbrauch geachtet werden, z.B. bei \textit{embedded systems}.\\
Hier kann sich eine sub-optimale Implementierung negativ auf den Energieverbrauch auswirken.\\
Müssen hohe Stückzahlen von Hardware günstig produziert werden, wird meist auf ein gewisses Maß an Leistung verzichtet (Prozessor, Speicher), weshalb auch hier die Implementierung eine effiziente Nutzung der Ressourcen anstreben sollte.\\

\subsection{Wir kritisch ist Verlässlichkeit?}
Unter \textbf{Verlässlichkeit} versteht man die Summe aus
\begin{itemize}
    \item \textbf{Zuverlässigkeit}
    \item \textbf{Sicherheit}
    \item \textbf{Verfügbarkeit}
    \item \textbf{Schutz vor unberechtigtem Zugriff}
\end{itemize}\\

\noindent
Einige Programmfehler können bei bestimmten Anwendungen ohne weitere Konsequenzen sein, in manchen können sie hingegen fatale Auswirkungen haben (Flugsicherheit, Medizintechnik, selbstfahrende Autos\ldots), weshalb die Gewichtung von Verlässlichkeit nicht außer Acht gelassen werden darf.

