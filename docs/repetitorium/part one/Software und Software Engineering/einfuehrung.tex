\section{Einführung}\label{sec:einfuhrung}

\begin{tcolorbox}[title={Software Engineering}]
    \textbf{Software Engineering} beschäftigt sich mit der \textbf{systematischen} und \textbf{methodischen} Entwicklung von Software durch \textbf{ingenieurmäßiges Vorgehen}.
\end{tcolorbox}

\noindent
Unter dem Begriff \textbf{Software} wird nicht allein der Programmcode verstanden, sondern alle \textit{Gegenstände}, die zur Softwareentwicklung gehören, wie bspw. auch die \textit{Dokumentation} des Codes für die Entwickler und die eigentliche \textit{Applikation} für den Anwender sowie die \textit{Daten}, die die Anwendung benötigt.\\
$\rightarrow$ \textbf{Software = Code + Dokumente}\\


\noindent
\textbf{Software Engineering} (i.F. \textit{SE}) will durch \textbf{ingenieurmäßiges Vorgehen} (i.F. \textit{IV}) eine wirtschaftlichere Entwicklung von Software gewährleisten.\\
$\rightarrow$ \textbf{Grundlage für IV: finanzielle Interessen stehen hinter der Software Entwicklung (i.F. \textit{SD})}\\

\noindent
\textbf{IV} bedeutet, dass \textbf{SE} auf wissenschaftlicher Basis und kodifizierter\footnote{
\textit{kodifizieren}: Regeln und Prinzipien in einem systematischen Format zusammenfassen und festlegen.
} Erfahrung beruht.\\

\noindent
IV beruht dabei auf Normen, Standards und Regeln:

\begin{itemize}
    \item\textbf{technische Ebene}: bspw. Vorlagen, wie Anforderungen an Software erfasst werden; Normen für Entwürfe von Software.
    \item \textbf{methodische Ebene}: bspw. festgeschriebene Reihenfolge von Tätigkeiten und zu erstellende Produkte für die Entwicklungsarbeit; Kriterien für den Einsatz technischer oder organisatorische Maßnahmen
\end{itemize}\\

\noindent
\textbf{Menschliche Aspekte} sind für den Prozess der SD genauso wichtig wie \textit{technische}: Kommunikation zwischen Kunde und Entwickler ist wichtig, genauso wie die Berücksichtigung menschlicher Bedürfnisse der Kunden und Endanwender, bspw. hinsichtlich intuitiver Bedienung der Software (s.a. Abschnitt~\ref{sec:anforderungen}).