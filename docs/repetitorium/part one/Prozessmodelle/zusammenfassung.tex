\section{Zusammenfassung}

\begin{itemize}
    \item Anforderungen lassen sich bei Projekten vorab nicht genügend festlegen.
    \item Fehlt im Team die Erfahrung oder werden neue Technologien eingesetzt, ist ein ausgereifter Entwurf nicht machbar.
    \item Bei großen Projekten würde man unter diesen Voraussetzungen mit dem Wasserfallmodell nicht flexibel genug sein,
    um auf Änderungen reagieren zu können.
    \item Aus diesem Grund gibt es alternative Modelle, die eingesetzt werden können:
        \begin{itemize}
            \item \textbf{inkrementell}: Aufteilung der Anforderungen, so dass Teilsysteme umgesetzt und an den Kunden ausgeliefert werden können.
            Die Teilsysteme werden sequentiell bearbeitet.
            \item \textbf{iterativ}: In Iterationen wird das Projekt in Zeitabschnitte unterteilt, in denen die Anforderungen umgesetzt werden; entsprechend dem Spiralmodell zunächst die risikoreichsten.
            Vorhergehende Ergebnisse werden in darauffolgenden Iterationen weiterbearbeitet.
            Ein bekanntes iteratives Modell ist \textit{RUP}, bei dem einzelne Phasen Ergebnis-Artefakte liefern, wie Anwendungsfälle oder Klassendiagramme.
            \item \textbf{nebenläufig}: Die Aufgaben werden aufgeteilt in parallel (oder nacheinander) bearbeitbare Aufgaben, die dann von den Mitarbeitern umgesetzt werden.
        \end{itemize}
    \item Die genannten Modelle werden häufig nicht isoliert betrachtet, sondern je nach Projekt auch kombiniert, insb. bei dem \textbf{agilen Vorgehen}.
\end{itemize}