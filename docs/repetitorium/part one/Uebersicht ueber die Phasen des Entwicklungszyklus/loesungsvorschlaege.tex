\section{Lösungsvorschläge}

\subsection{Aufgabe 2.2}

\begin{enumerate}
    \item  \textbf{Anforderungen}
        \begin{itemize}
            \item \textbf{Aktivitäten}: Einarbeitung in die Fachlichkeit, Lernen der fachlichen Sprache, Aufgaben der Software ermitteln
            \item \textbf{Artefakte}: Lastenheft (\textit{Vision \& Scope}), funkt. / nicht funkt. Anfordungen, Umfang des ersten Release
        \end{itemize}

    \item  \textbf{Analyse}
    \begin{itemize}
        \item \textbf{Aktivitäten}: Formalisierung der Geschäftsprozesse\footnote{ohne auf technische Details in den Modellen einzugehen}, Anforderungen systematisch konkretisieren/ spezifizieren
        \item \textbf{Artefakte}: Fachkonzept, Dömanenmodell, Geschäftsregeln, Schnittstellenbeschreibungen (GUI, externe Systeme \ldots)
    \end{itemize}

    \item  \textbf{Entwurf}
    \begin{itemize}
        \item \textbf{Aktivitäten}: Erstellung eines DV-Konzeptes auf Basis des Fachkonzeptes, Planung der Architektur und Struktur sowie des Zusammenspiels der Komponenten / Module untereinander
        \item \textbf{Artefakte}: DV-Konzept, Architektur, Klassendiagramme, Spezifikation von Klassen und Methoden
    \end{itemize}

    \item  \textbf{Realisierung}
    \begin{itemize}
        \item \textbf{Aktivitäten}: Umsetzung des DV-Konzeptes, Programmierung
        \item \textbf{Artefakte}: Code, Datenbankschemen, Dokumentation
    \end{itemize}

    \item  \textbf{Tests}
    \begin{itemize}
        \item \textbf{Aktivitäten}: Testen d. Integration in das Gesamtsystem
        \item \textbf{Artefakte}: Testfälle, Testprotokolle
    \end{itemize}

    \item  \textbf{Betrieb}
    \begin{itemize}
        \item \textbf{Aktivitäten}: Überführung in den Wirkbetrieb, Fehlerbehebung und kleinere Anpassungen
        \item \textbf{Artefakte}: -
    \end{itemize}
\end{enumerate}

