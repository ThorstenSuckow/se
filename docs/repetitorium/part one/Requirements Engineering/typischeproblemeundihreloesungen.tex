\section{Typische Probleme und ihre Lösungen}

Eine \textbf{ausführliche} und \textbf{verlässliche Berechnung} ist bei großen Projekten wegen der Auftragssumme Grundvoraussetzung für die Beauftragung einer Entwicklung.\\

\noindent
Deshalb wird i.d.R. zuerst ein \textbf{Vision \& Scope} erstellt und die \textbf{Rentabilität} sorgfältig geprüft.\\

\noindent
Wesentliche Fragen sollten aber auch bei kleineren Projekten geklärt werden, indem auch dort ein Lastenheft erstellt wird - ggfl. reicht hier auch nur eine stichpunkthaltige Zusammenfassung der wesentlichsten Punkte.\\

\subsubsection*{Erstellung von Vision \& Scope klärt Anforderungen}
Es sollte bei der Erstellung von \textbf{Vision \& Scope} in jedem Fall darauf geachtet werden, die \textbf{Stakeholder miteinzubeziehen}, damit keine wichtigen Anforderungen übersehen werden.\\

\subsubsection*{Verschiedene Menschen = verschiedene Ziele}
In diesem Zusammenhang zeigt sich auch immer wieder, dass verschiedene Mitarbeiter der Kunden sehr unterschiedliche Sichten auf die Aufgaben haben können, was zu \textbf{gegensätzlichen} bzw. \textbf{sich widersprechenden Anforderungen} führen kann.\\
\textit{Bewertung}, \textit{Wichtigkeit} und \textit{Zweckmäßigkeit} einzelner Anforderungen können aufgrund unterschiedlicher Ziele oder Aufgaben divergieren.\\
Hierbei hilft Vision \& Scope, die Anforderungen \textbf{gemeinsam zu erarbeiten} und eine \textbf{gemeinsame Lösung} zu definieren.
Ansonsten könnten Mitarbeiter, die ihre Interessen verletzt sehen, gegen das System \textbf{opponieren} oder es torpedieren (bspw. indem sie keine weiteren Informationen oder nur Fehlinformationen beitragen).\\

\subsubsection*{Neue Software verändert Kunden}
Wird eine neue Software zur Änderung und / oder Optimierung von Prozessen eingesetzt, kann dies ebenfalls dazu führen, dass \textbf{Mitarbeiter befürchten}, ihre Arbeit nicht mehr in dem Rahmen durchführen zu dürfen, wie sie das bisher getan haben\footnote{oder gar ihren Job zu verlieren}.\\
Um die \textbf{spätere Akzeptanz} zu gewährleisten, sollten auch solche Punkte im Lastenheft mit aufgeführt sein.\\
Spezialisierte \textbf{Organisationsentwickler} können dabei helfen, in einem Diagnose-Prozess zusammen mit den Beteiligten die aktuelle Situation zu untersuchen und bestehende Probleme zu identifizieren, um ein \textbf{gemeinsames Bewusstsein für das Problem} zu schaffen.\\
Sie begleiten außerdem den Soll-Entwurfsprozess, in dem von allen Beteiligten eine gewünschte Zukunft entworfen wird.\\
Diese Prozesse ziehen sich meist durch die gesamte Softwareentwicklung (vgl.~\cite[53]{Wed09}).\\

\subsubsection*{Vision \& Scope muss im Laufe des Projekts aktualisiert werden}
Die Arbeit an Anforderungen hilft auch dem Kunden, eigene Bedürfnisse besser zu verstehen, weshalb sich die \textbf{Sicht auf ein Projekt} im Laufe der Entwicklung \textbf{ändern} kann.\\
Das \textbf{Management} sollte dabei die \textbf{Abweichungen verfolgen} und bei Abweichungen zum Vision \& Scope gemeinsam mit den \textbf{Stakeholdern} dieses \textbf{nochmal überprüfen}.\\
Ansonsten kann die entwickelte Lösung auf Ablehnung stoßen, was Nacharbeiten oder ein Scheitern des Projektes bedeutet.\\
Kommen neue Anforderungen hinzu die außerhalb des Vision \& Scope liegen, kann das Lastenheft erweitert werden, oder es wird eine alternative Lösung gefunden, die preiswerter und sinnvoller ist\footnote{Beispiel bei \cite[53]{Wed09}}.