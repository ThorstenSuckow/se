\section{Lastenheft: Vision \& Scope}\label{sec:lastenheft-vision-scope}

Software-Ingenieure helfen Kunden dabei, bei der Entwicklung oder Anpassung von Software einen wirtschaftlichen Nutzen zu erreichen, bspw. eine \textit{erhöhte Produktivität} oder die Einhaltung gesetzlicher Vorschriften.\\
Der Aufwand für die Entwicklung der Softwarelösung muss geringer sein als der voraussichtliche Nutzen, was man unter \textbf{Business Case} zusammenfasst:

\blockquote[{\cite[11]{Brug09}}]{
    Ein Business Case ist ein Szenario zur betriebswirtschaftlichen Beurteilung einer Investition.
    Auch ein Projekt stellt eine Investition dar und
    muss deshalb seine Vorteilhaftigkeit gegenüber der Geschäftsleitung unter
    Beweis stellen. [...] Ein IT-Projekt ist mit Ausgaben verbunden. Um den Mitteleinsatz zu rechtfertigen, muss dem Management
    aufgezeigt werden, welchen Gegenwert („Return“) es von dem Projekt erwarten kann. Hierzu müssen Annahmen hinsichtlich der voraussichtlichen
    Kosten des Projektes und der mit seinen Ergebnissen erwarteten Ertragsauswirkungen und Kosteneinsparungen getroffen werden.
}

\subsubsection*{Definition der Ziele und des Umfangs}
Die Ziele des Kunden (\textbf{Vision}) dienen als Voraussetzung für eine Auswahl der Funktionalität für die zu entwickelnde Lösung.\\
Der Projektumfang (\textbf{Scope}) grenzt den Projektumfang ein und gibt Auskunft darüber, was entwickelt werden soll und was nicht.\\
Da der Nutzen für den Kunden im Vordergrund steht, muss es sich bei \textbf{Vision \& Scope} um die \textit{Kundensicht} handeln.\\

\subsubsection*{Beteiligte entwickeln gemeinsam eine Richtung}
Die \textbf{Stakeholder} - also alle vom Projekt betroffenen Personen - erarbeiten dabei eine gemeinsame Sicht auf die \textit{Zielstellung} und den \textit{Nutzen} des Projektes.\\
Wird auf diesen Schritt verzichtet und keine gemeinsame Sicht erarbeitet, werden mit hoher Wahrscheinlichkeit einige Personen(gruppen) ihre Interessen bei der Anforderungsanalyse oder  Entwicklung nicht vertreten sehen, woraus (tiefgreifende) Anpassungen oder auch die Ablehnung der entwickelten Lösung resultieren kann.\\
Aus diesem Grund ist es eine Frage der Effektivität und der Risikovermeidung, Vision \& Scope \textit{am Anfang} der Entwicklung zu klären (vgl.~\cite[44]{Wed09}).

\noindent
Der Kunde sollte seine Ziele in einem solchen Lastenheft festhalten. \\
Entwickler/Berater können hier bei der Projektdefinition helfen und auch die Machbarkeit aus technischer Sicht klären.\\

\noindent
Wichtig ist die Klärung von \textit{Kernfragen} in einem solchen Dokument:

\begin{itemize}
    \item Was ist die Vision?
    \item Was soll gemacht werden und was nicht?
    \item Welchen konkreten Nutzen erwartet der Kunde hiervon?
\end{itemize}

\noindent
Der Kurs nutzt hierzu eine Vorlage von \textit{Wiegers}, die wie folgt aufgebaut ist (vgl. \cite[81 ff. sowie 576 ff.]{WJ13}):

\begin{tcolorbox}[colback=white]
    \begin{enumerate}
        \item Geschäftsanforderungen
        \begin{enumerate}[label*=\arabic*.]
            \item Hintergrund
            \item Geschäftsmöglichkeit
            \item Geschäftsziele und Erfolgskriterien
            \item Erfordernisse von Kunde oder Markt
            \item Geschäftsrisiken
        \end{enumerate}
        \item Vision der Lösung
        \begin{enumerate}[label*=\arabic*.]
            \item Vision
            \item Wichtigste Features
            \item Annahmen und Abhängigkeiten
        \end{enumerate}
        \item Fokus und Grenzen
        \begin{enumerate}[label*=\arabic*.]
            \item Umfang des ersten Release
            \item Umfang der folgenden Releases
            \item Begrenzung und Ausschlüsse
        \end{enumerate}
        \item Geschäftskontext
        \begin{enumerate}[label*=\arabic*.]
            \item Stakeholder
            \item Projektprioritäten
            \item Technische Anwendungsumgebung
        \end{enumerate}
    \end{enumerate}
\end{tcolorbox}

\noindent
Im Folgenden wird auf die einzelnen Abschnitte kurz eingegangen:

\subsection*{1. Geschäftsanforderungen}
Dieser Abschnitt stellt dar, \textbf{wie} und \textbf{für wen} umgesetzt werden soll.

\subsubsection*{1.1 Hintergrund}
Enthält eine Zusammenfassung über die Hintergründe, die zu der Entscheidung geführt haben, die Software zu entwickeln.

\subsubsection*{1.2 Geschäftsmöglichkeit}
Für \textbf{kommerzielle Systeme} werden in diesem Abschnitt \textit{Marktlücke} und \textit{Markt} beschrieben, bei \textbf{Invividualentwicklungen} das \textit{Geschäftsproblem} und die zu \textit{verbessernden Geschäftsprozesse}.

\subsubsection*{1.3 Geschäftsziele und Erfolgskriterien}
Die \textbf{wichtigsten Nutzen} des Projektes werden in diesem Abschnitt auf \textit{messbare} Art und Weise zusammengefasst, wie
\begin{itemize}
    \item \textit{Marktanteile}
    \item \textit{Umsatzziele}
    \item \textit{Gewinnziele}
    \item \textit{Einhaltung gesetzlicher Vorgaben}
    \item \textit{Ablösung von nicht mehr unterstützter Software} und / oder \textit{Hardware}
\end{itemize}

\subsubsection*{1.4 Erfordernisse von Kunde oder Markt}
Die \textbf{Bedürfnisse} des \textbf{Kunden} oder eines \textbf{Nutzers} des Marktsegments werden in diesem Abschnitt beschrieben.

\subsubsection*{1.5 Geschäftsrisiken}
\textbf{Geschäftsrisiken} können sich sowohl für eine Entwicklung als auch für eine Nichtentwicklung ergeben.
Entsprechend werden diese in diesem Abschnitt aufgeführt.

\subsection*{2. Vision der Lösung}
In diesem Abschnitt ist die \textbf{strategische Vision} für die Entwicklung festgehalten, wobei hier weder auf  detaillierte Anforderungen noch auf die Projektplanung eingegangen wird.
Folgende Vorlage kann zur Beschreibung der Vision verwendet werden:
\begin{itemize}
    \item \textbf{Für} <Kunden>
    \item \textbf{die} <Nutzen, Gelegenheit>
    \item ist \textbf{der / die das} <Produktname>
    \item \textbf{ist} <Produktkategorie>
    \item \textbf{das} <Hauptnutzen, überzeugender Grund zu kaufen oder zu nutzen>
    \item \textbf{anders als} <Alternativen, aktuelles System oder Prozess>
    \item \textbf{Unser Produkt} <Unterschiede und Vorteile des neuen Produkts>
\end{itemize}

\subsubsection*{2.1 Vision}
Enthält eine \textbf{prägnante Beschreibung der Vision}, die den \textbf{langfristigen Nutzen} und das \textbf{Ziel des neuen Produkts} beschreibt.

\subsubsection*{2.2 Wichtigste Features}
\textbf{Wesentliche Features} werden zusammen mit \textbf{eindeutigen Bezeichnern} aufgeführt.

\subsubsection*{2.3 Annahmen und Abhängigkeiten}
Alle \textbf{Annahmen} der \textbf{Stakeholder}, die bei der Erarbeitung von \textbf{Vision \& Scope} geäußert wurden, sind hier festgehalten.\\
Andere Stakeholder werden diesen Annahmen u.U. nicht zustimmen.

\subsubsection*{3. Fokus und Grenzen}

\subsubsection*{3.1 Umfang des ersten Release}
Die \textbf{wichtigsten Features} des \textbf{ersten Release} werden hier aufgeführt, außerdem die \textbf{Qualitätsmerkmale} des ersten Releases.\\
Manche Qualitätsmerkmale wie Performance oder GUI-Design können u.U. erst für ein späteres Release wichtig sein.

\subsubsection*{3.2 Umfang der folgenden Releases}
Die \textbf{Features} der \textbf{nachfolgenden Releases} werden hier aufgeführt.

\subsubsection*{3.3 Begrenzung und Ausschlüsse}
Der Abschnitt definiert, was \textbf{zu dem Produkt gehört} und was nicht.\\
Auch Funktionen, die von  solch einer Lösung erwartet werden würden, die aber nicht geplant sind, werden hier aufgeführt.

\subsubsection*{4. Geschäftskontext}

\subsubsection*{4.1 Stakeholder}
Die \textbf{wichtigsten Stakeholder} werden hier aufgeführt, sowie den Nutzen, den diese von dem Projekt haben, ihre Einstellung zu dem Projekt, für sie wesentliche Features sowie Einschränkungen, denen die Stakeholder unterliegen\footnote{
\textit{Einschränkungen} bei \textit{Wiegers}: \textit{constraints}. Ein \textit{constraint} für Vertriebsmitarbeiter könnte bspw. sein, dass diese die Anwendung fast ausschließlich mobil nutzen, oder dass andere Stakeholder (bspw. nicht technisch-affine Mitarbeiter) Schulungen benötigen
} (s. Tabelle~\ref{tab:stakeholder}).


\begin{table}
[htbp]
    \centering
    \begin{tabular}{|l|l|l|l|l|}
        \hline
        \textbf{Stakeholder} & \textbf{Nutzen} & \textbf{Einstellung} & \textbf{Hauptinteresse} & \textbf{Randbedingungen}   \\
        \hline
        Marketingabteilung & \makecell{Einfacher Zugriff\\auf geeignete\\aktuelle Daten} & \makecell{begeistert\\von der\\Möglichkeit} & \makecell{optimales\\Erreichen der\\richtigen Kunden} & \makecell{muss einfach\\bedienbar sein} \\
        \hline
    \end{tabular}
    \caption{Beispiel für eine Auflistung von Stakeholdern. (Quelle: in Anlehnung an \cite[85]{Wed09})}\label{tab:stakeholder}
\end{table}




\subsubsection*{4.2 Projektprioritäten}
Führt die \textbf{Prioritäten des Projektes} auf, die \textbf{von allen Stakeholdern} unterstützt werden müssen.\\
Nach Möglichkeit sollen hierzu $5$ Kategorien unterschieden werden: \textit{Feature}, \textit{Qualität}, \textit{Termin},
\textit{Kosten}, \textit{Mitarbeiter}. \\
Es kann zudem unterschieden werden zwischen \textit{festen}, \textit{anpassbaren} und \textit{gewählten} Prioritäten.\\

\subsubsection*{4.3 Technische Anwendungsumgebung}
Beschreibt die \textbf{technische Umgebung}, in der die Anwendung laufen soll, also \textit{Hardware}, \textit{Betriebssysteme}, \textit{Netzwerk} usw.\\
Anforderungen an \textbf{Verfügbarkeit} und \textbf{Performance} und \textbf{Sicherheit} haben hier genauso ihren Platz.\\
Außerdem wird festgehalten, ob \textbf{Nutzer} oder \textbf{Hardware} \textbf{räumlich verteilt} sind.