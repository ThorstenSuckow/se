\chapter{Anmerkungen Präsenzphase}

\section{Black-Box-Verfahren}

Funktionsorientierte Tests werden auch \textbf{Black-Box-Tests} genannt und testen das Verhalten eines Programmteils, eines Systems oder auch Subsystems.\\
Ziel ist der Test der vollständigen Funktionalität ohne Berücksichtigung der inneren Struktur.\\
Testfälle werden hierbei aus den Spezifikationen abgeleitet: Grundlage aller Black-Box ist stets die Spezifikation.\\

\noindent
Zu den \textbf{Nachteilen} aller Black-Box Verfahren gehört
\begin{itemize}
    \item das nicht systematisch geprüft wird, ob das Programm mehr kann, als in der Spezifikation gefordert
    \item Fehler in der Spezifikation werden nicht entdeckt\footnote{
    hierfür eignen sich manuelle Prüfverfahren wie die (formale) Inspektion oder Reviews
    }.
\end{itemize}.

\section{Äquivalenzklassen}

\begin{itemize}
    \item Black-Box-Verfahren dienen zur sinnvollen Auswahl von Testmengen aus der großen Zahl der Möglichkeiten: bspw. \textbf{Äquivalenzklassen}
    \item Idee: Üblicherweise verhält sich ein Programm nicht für jeden Werst eines Eingabeparameters unterschiedlich
    \item[] $\rightarrow$ Eingabewerte werden zu Äquivalenzklassen zusammengefasst, und es wird angenommen
    \begin{enumerate}
        \item Ist der Test mit einem Repräsentanten einer Äquivalenzklasse fehlerfrei, ist er für andere Eingabedaten der Klasse auch fehlerfrei
        \item Erzeugt der Test mit einem Repräsentanten einer Äquivalenzklasse einen Fehler, treten auch Fehler mit anderen Eingabedaten dieser Klasse auf
    \end{enumerate}
\end{itemize}

\vspace{5mm}
\begin{tcolorbox}[title={Ideale Testmenge}]
    Repräsentanten von Äquivalenzklassen unterstellt man, dass sie eine ideale Testmenge darstellen.
\end{tcolorbox}

\subsection*{Bestimmung der Äquivalenzklassen}

Grundsätzlich wird der Definitionsbereich einer Eingabe eines Programmes betrachtet und daraus eine (oder mehrere\footnote{
``oder mehrere``: bezieht sich auf eine Granulierung der Äquivalenzklassen durch weitere Unterteilungen, falls nötig
}) \textbf{gültige} Äquivalenzklassen gebildet ($gK$); für ungültige Werte wird eine (oder mehrere) \textbf{ungültige} Äquivalenzklassen gebildet ($uK$).\\

\subsubsection*{Richtlinien zur Bildung von Äquivalenzklassen}

\begin{itemize}
    \item \textbf{geordneter Definitionsbereich}: eine $gK$ für Werte innerhalb des Definitionsbereichs, je eine $uK$ unterhalb und eine oberhalb des Definitionsbereichs
    \item \textbf{Aufzählung $K$}: für jedes Element in $K$ wird eine $gK$ gebildet, eine $uK$ für ein Element aus $\overline{K}$
    \item \textbf{zusammengesetzter Datentyp}: für jeden Untertyp mindestens eine $gK$ und eine $uK$
\end{itemize}

\subsection*{Bestimmung der Testmenge}

\begin{itemize}
    \item Bildung von \textbf{Gutfällen} ausschliesslich durch alle möglichen Kombinationen der vorhandenen Repräsentanten der $gK$
    \item Bildung von \textbf{Schlechtfällen} ausschliesslich durch genau einen Repräsentanten einer $uK$, der mit Repräsentanten der übrigen $gK$ kombiniert wird
\end{itemize}

\subsection*{Grenzwertanalyse}

Die Grundidee der Grenzwertanalyse lautet: Es werden die \textbf{Grenzen der Äquivalenzklassen} und deren benachbarte Werte betrachtet, da hier besonders oft ``fehlerträchtige Fallunterscheidungen`` (\textit{T\&V-Skript}) gemacht werden.

\vspace{5mm}
\begin{tcolorbox}[title={Vorraussetzungen für die Grenzwertanalyse}]
    \begin{itemize}
        \item die Äquivalenzklasse ist geordnet
        \item Grenzen lassen sich identifizieren
    \end{itemize}
\end{tcolorbox}

\section{White-Box-Verfahren}

Zur Konstruktion von Testfällen als auch zur Bestimmung der Vollständigkeit wird bei den strukturorientierten Testverfahren der Quellcode betrachtet.\\
Verschiedene Überdeckungskriterien sollen hierbei sicherstellen, das bspw. alle Anweisungen mindestens einmal ausgeführt worden sind.\\

\noindent
Als Hilfsmittel bei den White-Box-Verfahren wird der Programmablaufgraph bzw. der Kontrollflussgraph herangezogen:

\vspace{5mm}
\begin{tcolorbox}[title={Programmablaufgraph}]
    Ein $PAG$ (\textit{Programmablaufgraph}) ist ein gerichteter Graph mit einer endlichen Knotenmenge und gerichteten Kanten.\\

    \noindent
    Knoten können zu \textbf{Anweisungsblöcken} zusammengefasst werden.
    Bedingung ist hier, dass die erste Anweisung des Anweisungsblockes den Beginn des Blckes markiert, und dieser genau einmal vollständig durchlaufen wird.\\

    \noindent
    Ein \textbf{Pfad} ist ein \textbf{geschlossener Kantenzug} zwischen zwei Knoten.\\

    \noindent
    Ein \textbf{vollständiger Pfad} ist ein Pfad zwischen Start- und Endknoten des Graphs.
\end{tcolorbox}
\vspace{5mm}

\noindent
Die Auswahl der Testmenge für einen strukturorientierten Test richtet sich nach der Auswahl der Überdeckungskriterien.
Möglich sind u.a.

\begin{itemize}
    \item \textbf{Knotenüberdeckung}: Jeder Knoten (also jede Anweisung eines Programmes) wird mindestens einmal durchlaufen. Nachteil: Nicht alle Kanten werden verfolgt
    \item \textbf{Kantenüberdeckung}: Jede Kante (also jeder Zweig eines Programmes) wird mindestens einmal durchlaufen. Nachteil: Nicht alle Kombinationen von Kanten werden berückscihtigt
    \item \textbf{Bedingungsüberdeckung}: Verschiedene Auswertungsmöglichkeiten der \textbf{Verzweigungsbedingung} werden berücksichtigt, bspw.
    \begin{itemize}
        \item einfache Bedingungsüberdeckung: Jede atomare Bedingung in jeder Verzweigungsbedingung nimmt jeweils einmal den Wert $true$/$false$ an. Nachteil: Geringe Aussagekraft, insb. werden nicht immer zusammengesetzte Bedingungen genügend berücksichtigt. Messung der Überdeckung schwierig, wenn im Programm short-circuit evaluation verwendet wird.
        \item mehrfache Bedingungsüberdeckung: alle Kombinationen der Wahrheitswerte treten auf. Jede Verzweigungsbedingung mit $n$ atomaren Bedingungen nimmt somit alle $2^n$ verschiedenen Kombinationen von $true$/$false$ an. Nachteil: Hoher Aufwand durch exponentielles Wachstum mit der Anzahl der atomaren Bedingungen.
    \end{itemize}
    \item \textbf{Pfadüberdeckung}: Alle möglichen vollständigen Pfade werden mindestens einmal durchlaufen. Nachteil: ``Explosives Wachstum`` der Pfade bei Schleifen und Verzweigungen (\cite[406]{Bal97})
\end{itemize}

\vspace{5mm}
\begin{tcolorbox}[title={Verzweigungsbedingung}]
    Eine \textbf{Verzeigungsbedingung} ist ein Knoten mit mindestens zwei ausgehenden Kanten.
\end{tcolorbox}
\vspace{5mm}

\noindent
Wenn die Anzahl der Schleifendurchläufe von der Eingabe abhängig ist, lässt sich eine obere Schranke für die Anzahl der Iterationen of nur schwer finden.\\
Für eine vollständige Pfadüberdeckung bedeutet das, dass die Anzahl der möglichen zu durchlaufenden Pfade oft von exponentiellem Maß ist, und generell eine 100\%ige Pfadüberdeckung dadurch nur schwer erreicht werden kann (Anzahl der Testfälle, Dauer für die Tests).\\
Als ``gute Heuristik`` gibt das \textit{T\&V-Skript} die \textbf{vereinfachte Pfadüberdeckung} an, die aus folgenden Testfällen besteht:

\begin{itemize}
    \item kein Eintritt in die Schleife
    \item genau ein Durchlauf der Schleife, um \textit{Initialisierungsfehler} aufzudecken
    \item genau zwei Durchläufe der Schleife. Manche Initialisierungsfehler fallen nur so auf
    \item eine typische Anzahl von Iterationen als \textbf{Normalfall}
    \item die maximal mögliche Anzahl von Iterationen. Hierdurch sollen Fehler im \textit{Abbruchkriterium} gefunden werden.
\end{itemize}

\noindent
Zu den \textbf{Vorteilen} aller White-Box Verfahren gehört, dass die Testintensität an die Software angepasst werden kann, und dass eine große Menge an bereits fertigen Tools für den Einsatz der Verfahren gibt.\\
Zu den \textbf{Nachteilen} aller White-Box Verfahren gehört
\begin{itemize}
    \item es wird davon ausgegangen, dass immer alle Anweisungen erreichbar sind, was nicht immer der Fall sein muss: 100\% Überdeckung ist dann nicht möglich
    \item falls keine 100\%ige Überdeckung gemessen wird, kann algorithmisch nicht entschieden werden, ob dies an den Testfällen liegt, oder ob unerreichbare Anweisungen vorliegen
    \item selbst wenn 100\% Überdeckung erreicht wird, ist nicht garantiert, dass das getestete System die geforderten Spezifikationen erfüllt\footnote{hierfür eignen sich Grey-Box-Tests}
\end{itemize}.