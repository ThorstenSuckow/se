\section{Inhalt SE4}

\section*{1. Qualität}

\subsection*{Lernziele}

\begin{itemize}
    \item die Qualität eines Softwaresystems definieren können
    \item typische Qualitätsmerkmale kennen
    \item Qualitätskriterien messbar festlegen können
\end{itemize}

\subsection*{Zusammenfassung}

\begin{itemize}
    \item \textbf{Qualität} kann mit Hilfe von \textbf{Qualitätssystemen} definiert werden.
    \item \textbf{Qualitätssysteme} teilen \textbf{Qualität} in \textbf{Qualitätsmerkmale} ein, die in \textbf{Qualitätsteilmerkmale} eingeteilt werden.
    \item \textbf{Qualitätsindikatoren} definieren die \textbf{Qualitätsteilmerkmale}: Sie repräsentieren \textbf{Messvorschriften} (\textit{Metriken}), die eine Bestimmung von \textbf{Kennzahlen} und deren \textbf{Sollwerte} beschreiben.
    \item Als Hilfsmittel zur Beschreibung von Qualitätsteilmerkmalen dienen \textbf{Qualitäts-Szenarien}.
    \item \textbf{Vorlagen} für \textbf{Qualitätssysteme} finden sich in den Standards, bspw. in der Norm \textbf{ISO/EC 25000}.
    \item I.d.R. werden aus einem Qualitätssystem \textbf{Teilmerkmale} ausgesucht, geeignete Metriken gesucht und die \textit{relative Wichtigkeit} der Teilmerkmale definiert.
\end{itemize}


\section*{2. Typen von Qualitätsmaßnahmen}
\subsection*{Lernziele}
\begin{itemize}
    \item grundlegende Arten von qualitätssichernden Maßnahmen kennen
    \item grundlegende Arten von qualitätssichernden Maßnahmen nach verschiedenen Gesichtspunkten kategorisieren können
\end{itemize}

\subsection*{Zusammenfassung}
\begin{itemize}
    \item Bei \textbf{qualitätssichernden Maßnahmen} können \textbf{konstruktive} und \textbf{analytische Maßnahmen} unterschieden werden
    \item \textbf{Konstruktive Maßnahmen} beinhalten \textbf{organisatorische Maßnahmen} (bspw. Standards)
    sowie \textbf{technische Maßnahmen} (bspw. Einsatz von Werkzeugen): Diese Maßnahmen sollen das \textit{Entstehen von Fehlern} verhindern
    \item \textbf{Analytische Maßnahmen} hefen dabei, entstandene Fehler zu finden: Hierzu gehören \textbf{Tests}, aber auch analysierende manuelle oder werkzeuggestützte Maßnahmen
    \item das \textbf{V-Modell} beschreibt qualitätssichernde Maßnahmen in ihrer Einordung in den Entwicklungsprozess: Klassen und Module werden im \textbf{Modultest} geprüft, das Zusammenspiel der Teile im \textbf{Integrationstest}, die \textit{gesamte Anwendung} im \textbf{Systemtest} und im Anschluss vom Kunden im \textbf{Abnahmetest}
\end{itemize}

\section*{3. Manuelle Verfahren}
\subsection*{Lernziele}
\begin{itemize}
    \item wissen, was manuelle Verfahren sind
    \item die verschiedenen manuellen Verfahren kennen
    \item den sinnvollen Einsatz der verschiedenen manuellen Verfahren kennen
    \item manuelle Verfahren durchführen können
\end{itemize}

\subsection*{Zusammenfassung}
\begin{itemize}
    \item \textbf{Manuelle Verfahren} sind wichtige Werkzeuge zur \textbf{Qualitätssicherung} von \textbf{Artefakten} der Softwareentwicklung
    \item sie sind bei \textbf{Dokumenten} und \textbf{Code} unabdingbar - Dokumente lassen sich zudem besser manuell vollständig prüfen
    \item bei \textbf{Programmcode} lassen sich manche Defekte einfacher und damit kostengünstiger finden
    \item je nach Verfahren untersuchen \textit{einzelne} oder \textit{mehrere} \textbf{Gutachter} oder \textit{zusammen in Gruppen}
    \item dabei unterscheiden sich die Verfahren in Umfang und dem Grad an Formalisierung: Je nach Wichtigkeit des \textbf{Prüfgegenstandes} ist eine andere \textbf{Prüfart} angemessen
    \item bei formalisierten Verfahren ist es wichtig, sich an die Abläufe zu halten, damit die Verfahren erfolgreich durchgeführt werden
    \item hilfreiche Werkzeuge sind \textbf{Checklisten}, um \textbf{systematisch} nach Defekten in Prüfgegenständen
    zu suchen
\end{itemize}

\section*{4. Werkzeuggestützte Analyse}
\subsection*{Lernziele}
\begin{itemize}
    \item einen Überblick über die Typen von werkzeuggestützte Analyse besitzen
    \item die Leistungsfähigkeit von werkzeuggestützter Analyse beurteilen können
    \item beurteilen können, bei welcher Problemstellung welcher Typ sinnvollerweise eingesetzt wird
\end{itemize}

\subsection*{Zusammenfassung}
\begin{itemize}
    \item Werkzeuge zur \textbf{statischen Analyse von Sourcecode} werden eingesetzt, um
    \begin{itemize}
        \item die Einhaltung von \textbf{Programmierrichtlinien} zu überprüfen
        \item \textbf{typische Defekte} anhand vorgegebener Muster zu finden
        \item Ablauffehler mit Hilfe der \textbf{Datenflussanomalieanalyse} und der \textbf{abstrakten Interpretation} zu erkennen
        \item die Wartbarkeit mit Metriken zu beurteilen
    \end{itemize}
    \item In bestimmten Bereichen ist der Einsatz solcher Werkzeuge zum Industriestandard geworden, bspw. im Bereich Automotive
    \item mittels der werkzeuggestützten Analyse können Defekte entdeckt werden, aber eine Aussage über die \textbf{Lauffähigkeit} oder \textbf{Brauchbarkeit} kann nicht angegeben werden, da die \textbf{Funktionalität} \textit{nicht} analysiert wird
    \item der Einsatz der Werkzeuge muss sorgfältig vorbereitet werden
    \item die Regelsätze der Werkzeuge müssen an die Gegebenheiten angepasst werden
    \item der Entwicklungsprozess muss den Einsatz der Werkzeuge berücksichtigen
    \item der Einsatz solcher Werkzeuge kann eine sehr kostengünstige und effektive Maßnahme sein
\end{itemize}

\section*{5. Testende Verfahren}
\subsection*{Lernziele}
\begin{itemize}
    \item Kriterien für die Vollständigkeit von Tests kennen
    \item geeignete Testverfahren für die Aufgabe auswählen
    \item Tests effektiv gemäß den Zielen definieren und durchführen können
\end{itemize}

\subsection*{Zusammenfassung}
\begin{itemize}
    \item für die \textbf{Qualitätssicherung} einer Software sind \textbf{testende Verfahren} unverzichtbar
    \item es gibt verschiedene \textbf{systematische Verfahren}, die bei der Erstellung von \textbf{Testfällen} hilfreich sein können:
    \begin{itemize}
        \item Bei \textbf{funktionsorientierten Tests} (\textbf{Black-Box-Tests}) orientieren sich Tester an \textbf{spezifizierter Funktionalität}.
        \item[] Wichtige Hilfsmittel sind hierbei \textbf{Äquivalenzklassen} von \textbf{Eingabedaten}, bei denen sich Software \textbf{gleichartig} verhält.
        \item[] Außerdem helfen \textbf{zustandsbasierte Tests} dabei, Systeme in allen \textbf{Zuständen} und \textbf{Zustandsübergängen} zu testen.
        \item Bei \textbf{strukturorientierten Testverfahren} (\textbf{White-Box-Tests}) orientieren sich Tester an der \textbf{inneren Struktur} der Software.
        \item[] Der Umfang strukturorientierter Testverfahren hängt davon ab, ob
        \begin{itemize}
            \item alle Anweisungen (\textbf{Anweisungsüberdeckung})
            \item alle Verzweigungen (\textbf{Zweigüberdeckung})
            \item alle Schleifendurchläufe (\textbf{Boundary-Interior-Überdeckung})
            \item alle Bedingungen (\textbf{einfache Mehrfachbedingungsüberdeckung})
        \end{itemize}
        \noindent getestet werden sollen, wobei der tatsächliche Umfang durch Überdeckungskriterien  definiert wird
        \item[] Strukturorientierte Tests werden in der Regel nur im \textbf{Klassen-} bzw. \textbf{Komponententest} durchgeführt.
        \item eine \textbf{vollständige Überdeckung} garantiert keinen vollständigen Test, da hierbei nicht auf Funktionalität geachtet wird, umgekehrt werden wahrscheinlich nicht alle Codeteile getestet
        \item aus diesem Grund existiert der \textbf{Grey-Box-Test}, der funktionsorientiert testet und die Quellcode-Abdeckung mit Werkzeugen überprüft.
        \item \textbf{Grey-Box-Tests} werden häufig im \textbf{Klassentest} eingesetzt
    \end{itemize}
    \item für den Test von Systemen, die mit \textbf{objektorientierten Mitteln} implementiert worden sind, gibt es Techniken, die \textbf{Vererbung}, \textbf{Kapselung}  und \textbf{parametrisierte Klassen} berücksichtigen
    \item für die \textbf{Integration} und den \textbf{Integrationstest} gibt es verschiedene Vorgehensweisen, wobei sich in der Praxis der \textbf{Bottom-Up-Ansatz} bewährt hat
    \item im \textbf{Systemtest} wird neben der \textbf{Funktionalität} des Gesamtsystems auch die nicht-funktionalen Anforderungen getestet
    \item der Kunde prüft das System im \textbf{Abnahmetest} auf seine Einsetzbarkeit, wozu der \textbf{Verbundtest}, \textbf{Probebetrieb} oder der \textbf{Betatest} gehören kann
\end{itemize}

\section*{6. Management des QS-Prozesses}
\subsection*{Lernziele}
\begin{itemize}
    \item für kleinere Projekte qualitätssichernde Maßnahmen planen und verfolgen können
    \item Tests planen und dokumentieren können
    \item gefundene Fehler geeignet verwalten können
\end{itemize}


\subsection*{Zusammenfassung}
\begin{itemize}
    \item die \textbf{Qualitätssicherung} bei der Softwareentwicklung ist ein komplexer Prozess, der u.a. verschiedene Aufgaben wie Tests, Reviews und werkzeuggestützte Analysen beinhaltet
    \item diese Maßnahmen werden im \textbf{Qualitätssicherungsplan} und \textbf{Testplan} beschrieben
    \item je nach Art und Größe  der Aufgabe werden diese Maßnahmen durch die Entwickler, dedizierte Tester oder Testteams durchgeführt
    \item qualitätssichernde Maßnahmen müssen \textbf{dokumentiert} werden:
    \begin{itemize}
        \item \textbf{Testfälle} beschreiben die Durchführung der Tests
        \item \textbf{Testprotokolle} halten die Durchführung der Tests fest
        \item \textbf{Fehlerberichte} beschreiben die aufgetretenen Fehler
    \end{itemize}
    \item die Bearbeitung von Fehlermeldungen folgt einem komplexen \textbf{Workflow}, weshalb dafür spezielle Werkzeuge zum Einsatz kommen
    \item der gesamte Testprozess wird mit \textbf{Kennzahlen} verfolgt, wobei die wichtigsten Kennzahlen die \textbf{Anzahl der Fehlerberichte} und die \textbf{Abarbeitung der Testfälle} sind
\end{itemize}