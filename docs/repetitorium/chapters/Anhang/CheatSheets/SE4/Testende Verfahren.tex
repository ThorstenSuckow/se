\section{Testende Verfahren}

\begin{tcolorbox}[title=Strukturorientierte und funktionsorientierte Tests]
    Ein Test ist \textbf{strukturorientiert}, wenn seine Vollständigkeit anhand der Codeüberdeckung definiert wird.  \textbf{Funktionsorientierte Tests} testen die gesamte Funktionalität ohne Berücksichtigung der inneren Struktur.\\

    \begin{itemize}
        \item \textbf{Funktionsorientierter Test (Black-BoxTest)}: Ein Test kann als \textbf{vollständig} betrachtet werden, wenn die \textbf{gesamte Funktionalität} im Test erprobt worden ist.
        \item[] Diese Art von Test wird als \textbf{funktionsorientierter} bzw. \textbf{Black-Box-Test} bezeichnet.
        \item[] Bei dieser Art von Tests ist es möglich, dass nicht alle Codeteile ausgeführt werden, weil sie
        \begin{itemize}
            \item nicht erreichbar sind
            \item nicht spezifiziert sind, aber technisch notwendig
            \item fehlerhafterweise überflüssig sind
        \end{itemize}
        \noindent
        Es könnte kritisch sein, wenn diese Codeteile ungetestet bleiben.\\
        Auch Tests, die sich auf nicht-funktionale Anforderungen beziehen (etwa Performance), werden meistens ohne Berücksichtigung auf die \textit{innere Struktur} durchgeführt und deswegen in die Reihe der Black-Box-Tests eingereiht - auch, wenn sie nicht funktionsorientiert sind.
        \item \textbf{Strukturorientierter Test (White-Box-Test)}: Ein Test ist \textbf{strukturorientiert}, wenn seine Vollständigkeit anhand der Codeüberdeckung definiert wird.
    \end{itemize}
\end{tcolorbox}