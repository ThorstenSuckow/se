\section{Testfall}

\subsubsection*{Testfall MAR1-01: Kundenliste erzeugen für PLZ - Gutfall}

\begin{itemize}
    \item[] \textbf{Autor}: at
    \item[] \textbf{Kategorie}: Systemtest
    \item[] \textbf{Beschreibung}:
    \item[] \textbf{Setup}:
    \begin{itemize}
        \item Im System sind die Kundendaten TD01 eingespielt und nicht modifiziert worden
        \item Der Nutzer ist angemeldet und besitzt die Rolle ``Leitender Mitarbeiter``
    \end{itemize}
\end{itemize}

\noindent
\textbf{Vorgehen}:
\begin{table}[]
    \centering
    \setlength{\tabcolsep}{0.5em}
    \def\arraystretch{1.5}
    \begin{tabular}{|c|l|c|}
        \hline
        \multicolumn{1}{|l|}{\textbf{Lfd. Nr.}} & \multicolumn{1}{c|}{\textbf{Schritt}}        & \textbf{erwartetes Ergebnis}                        \\ \hline
        1                                       & Auswahl PLZ Combobox                         & ist ausgewählt                                      \\ \hline
        2                                       & Auswahl von-bis                              & ist ausgewählt                                      \\ \hline
        3                                       & Eingabe 18430 und 18435 in die Eingabefelder & Zahlen werden dargestellt                           \\ \hline
        4                              & Export anklicken                             & \multicolumn{1}{l|}{Dateiauswahldialog öffnet sich} \\ \hline
        5                              & Auswahl eines geeigneten Dateinamens         & \multicolumn{1}{l|}{Datei mit Inhalt der Daten}     \\ \hline
    \end{tabular}
\end{table}

\begin{itemize}
    \item[] \textbf{Zugrundeliegende Anforderungen}:
    \begin{itemize}
        \item Anwendungsfall MAR-1 vom 11.01.08
        \item GUI-Entwurf MAR-1 vom 26.03.08
        \item Liste der Geschäftsregeln (GSR) vom 18.01.08
    \end{itemize}
\end{itemize}