\section{Testdokumentation Kennzahlen}


\begin{tcolorbox}[title=Kennzahlen der Tesdokumentation]

    \begin{itemize}
        \item \textbf{Anzahl Fehlerberichte}
        \begin{itemize}
            \item Die Kennzahl \textbf{Anzahl der Fehlerberichte} kann nach dem Bearbeitungsstand und der Priorität aufgeschlüsselt werden.\\
            Übersteigt bspw. über längere Zeit die Anzahl der \textbf{neu gefundenen Defekte} die Anzahl der \textbf{behobenen Defekte}, ist das ein Hinweis darauf, dass die Entwickler mit der \textbf{Behebung der Fehler nicht nachkommen}
        \end{itemize}
        \item \textbf{Anzahl Testfälle}
        \begin{itemize}
            \item Die Kennzahl \textbf{Anzahl der Testfälle} wird aufgeschlüsselt nach ihrem Bearbeitungsstand (\textit{offen}, \textit{in Bearbeitung}, \textit{abgeschlossen}).\\
            \textbf{Stagniert} diese Zahlen, ist das ein Hinweis auf \textbf{Probleme im Test}.\\
            \textbf{Steigt} hingegen die Anzahl der Tests mit dem Status ``\textbf{in Arbeit}`` eher stark an, ohne, dass sich die Anzahl der \textbf{abgeschlossenen} Tests verändert, ist das ein Hinweis daraus, dass die \textbf{Qualität der Software} so \textbf{niedrig} ist, dass \textbf{kein Testfall abgeschlossen werden kann}.
        \end{itemize}
    \end{itemize}

\end{tcolorbox}