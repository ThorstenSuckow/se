\section{Einfache Mehrfachbedingungsüberdeckung}

\begin{tcolorbox}[title=Einfache Mehrfachbedingungsüberdeckung]
    Die \textbf{einfache Mehrfachbedingungsüberdeckung} wird erreicht, wenn \textit{jede} atomare Bedingung, also jeder \textit{direkte Vergleich}, und \textit{alle zusammengesetzten Bedingungen} einmal den Wahrheitswert \textit{wahr} und einmal den Wahrheitswert \textit{falsch} annehmen\footnote{
        \textit{Maxterm}: Für genau eine Variablenbelegung falsch (Disjunktion: $A \lor B \lor C$); bzw. \textit{Minterm}: für genau eine Variablenbelegung wahr (Konjunktion: $A \land B \land C$)(vgl.~\cite[92]{Hof22})
    }.\\
    Dadurch wird gleichzeitig eine \textbf{Zweigüberdeckung} erreicht
\end{tcolorbox}