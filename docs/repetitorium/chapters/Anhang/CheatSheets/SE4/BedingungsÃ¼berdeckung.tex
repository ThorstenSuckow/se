\section{Bedingungsüberdeckung}

\begin{tcolorbox}[title=Bedingungsüberdeckung]

    Der \textbf{Bedingungsüberdeckungstest} (\textit{condition coverage test}) beachtet die logische Struktur von Entscheidungen der zu testenden Software und kommt in unterschiedlichen Ausprägungen vor, wobei die schwächste der \textbf{einfache Bedingungsüberdeckungstest} (\textit{simple condition coverage test}) ist, der weder Anweisungsüberdeckungs- noch Zweigüberdeckungstest subsumiert (vgl.~\cite[93]{Lig09a}).
\end{tcolorbox}

\begin{tcolorbox}[title=Einfacher Bedingungsüberdeckungstest]
    \begin{itemize}
        \item \textbf{Einfacher Bedingungsüberdeckungstest} (\textit{simple condition coverage}): Fordert den Test aller \textbf{atomaren} Teilentscheidungen gegen \textit{wahr} oder \textit{falsch}, allerdings ist dadurch \textbf{nicht} garantiert, dass eine vollständige Zweigüberdeckung erreicht wird:

         \begin{minted}{java}
             if ((A || B) && (C || D)) {...
         \end{minted}

        \noindent
        Folgende zwei Testfälle genügen der einfachen Bedingungsüberdeckung, erreichen aber keine  Zweigüberdeckung:
        \begin{equation}\notag
            \begin{alignat}{3}
                &A: true, B: false, C: true, D: false && \text{ (Gesamtausdruck: $true$)}\\
                &A: false, B: true, C: false, D: true && \text{ (Gesamtausdruck: $true$)}\\
            \end{alignat}
        \end{equation}
        \noindent
        Die Forderung, dass jede atomare Bedingung einmal den Wahrheitswert \textit{wahr} und einmal den Wahrheitswert \texti{falsch} annimmt, ist durch beide Testfälle erfüllt. \\
        Folgende Testfälle erfüllen dieselbe Forderung, erreichen aber im Gegensatz zu den vorherigen Testfällen eine Zweigüberdeckung:
        \begin{equation}\notag
        \begin{alignat}{3}
            &A: true, B: true, C: true, D: true && \text{ (Gesamtausdruck: $true$)}\\
            &A: false, B: false, C: false, D: false && \text{ (Gesamtausdruck: $false$)}\\
        \end{alignat}
        \end{equation}
    \end{itemize}
\end{tcolorbox}

\begin{tcolorbox}[title={Bedingungs-/Entscheidungsüberdeckungstest}]
    \begin{itemize}
        \item \textbf{Bedingungs-/Entscheidungsüberdeckungstest} \textit{(condition / decision coverage)}:
        Garantiert ergänzend zu der einfachen Bedingungsüberdeckung einen vollständigen Zweigüberdeckungstest, allerdings kann dadurch die logische Gliederung kompliziert aufgebauter Entscheidungen ignoriert werden (vgl.~\cite[100]{Lig09a}). Für $(A || B) \&\& (C || D)$:
        \begin{equation}\notag
        \begin{alignat}{3}
            &A: false, B: true, C: false, D: false\\ &\qquad \text{ ($A || B:true, C || D: false$, Gesamtausdruck:  $false$)}\\
            &A: true, B: false, C: true, D: true\\ &\qquad \text{ ($A || B:true, C || D: true$, Gesamtausdruck: $true$)}\\
        \end{alignat}
        \end{equation}
        In dem Beispiel wird zwar eine Zweigüberdeckung erreicht, aber die Teilentscheidungen ($A||B, C||D$) evaluieren nicht zu jeweils unterschiedlichen Ergebnissen in den Testfällen.
    \end{itemize}

\end{tcolorbox}