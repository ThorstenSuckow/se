\section{Zweigüberdeckung}


\begin{tcolorbox}[title=Zweigüberdeckung]
    Eine \textbf{Zweigüberdeckung} liegt vor, wenn \textit{jede Kante} des Kontrollflussgraphen mindestens einmal ausgeführt worden ist.
    Es ist also nicht notwendig, alle Bedingungen auf einmal zu erfüllen, solange mit entsprechenden Testdaten jeweils ein Zweig durchlaufen werden kann, und am Ende alle Zweige durchlaufen worden sind.
    \begin{minted}[fontsize=\small]{java}
        // a = 1 -> erstes if
        // a = -2 -> zweites if
        if (a > 0) {...}
        if (a % 2 == 0) {...}
    \end{minted}
    Liegt eine Zweigüberdeckung vor, ist gleichzeitig auch die \textbf{Anweisungsüberdeckung} erfüllt.\\
    Die Zweigüberdeckung bietet schon eine ziemlich vollständige Überdeckung, berücksichtigt aber Schleifen nicht genügend.\\

    \noindent
    Die Zweigüberdeckung wird auch \textbf{Entscheidungsüberdeckung} genannt, da jede Entscheidung die Wahrheitswerte \code{true} und \code{false} mindestens einmal annehmen muss (vgl.~\cite[404]{Bal97}).
\end{tcolorbox}