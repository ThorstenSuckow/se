\section{Zweigüberdeckung}


\begin{tcolorbox}[title={Zweigüberdeckung ($C_1$-Test)}]
    Das Ziel des Zweigüberdeckungstest (\textit{branch coverage test}), auch $C_1$-Test genannt, ist die Ausführung aller Zweige des Prüflings  (vgl.~\cite[88]{Lig09a}).\\

    Eine \textbf{Zweigüberdeckung} liegt vor, wenn \textit{jede Kante} des Kontrollflussgraphen mindestens einmal ausgeführt worden ist.
    Es ist also nicht notwendig, alle Bedingungen auf einmal zu erfüllen, solange mit entsprechenden Testdaten jeweils ein Zweig durchlaufen werden kann, und am Ende alle Zweige durchlaufen worden sind.
    \begin{minted}[fontsize=\small]{java}
        // a = 1 -> erstes if
        // a = -2 -> zweites if
        if (a > 0) {...}
        if (a % 2 == 0) {...}
    \end{minted}
    Liegt eine Zweigüberdeckung vor, ist gleichzeitig auch die \textbf{Anweisungsüberdeckung} erfüllt: Der Zweigüberdeckungstest \textbf{subsumiert} den Anweisungsüberdeckungstest.\\

    \noindent
    Die Zweigüberdeckung bietet schon eine ziemlich vollständige Überdeckung, berücksichtigt aber Schleifen nicht genügend.
    Sie wird auch \textbf{Entscheidungsüberdeckung} genannt, da jede Entscheidung die Wahrheitswerte \code{true} und \code{false} mindestens einmal annehmen muss (vgl.~\cite[404]{Bal97}).\\


    \noindent
    Das Testmaß ist das Verhältnis der ausgeführten Zweige zu der im Prüfling vorkommenden Gesamtzahl an Zweigen (Zweig = Kante):

    \begin{equation}\notag
    C_{Zweig} = \frac{\text{Anzahl ausgeführte Zweige}}{\text{Anzahl vorhandene Zweige}}
    \end{equation}
\end{tcolorbox}