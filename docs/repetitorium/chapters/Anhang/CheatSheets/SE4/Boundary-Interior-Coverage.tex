\section{Boundary-Interior-Coverage}

\begin{tcolorbox}[title=Boundary-Interior-Coverage]
    Das Verhalten des Testgegenstandes beim abweisenden Fall oder bei mehreren Durchläufen einer Schleife führen zu der Forderung der Verallgemeinerung der Möglichkeiten: Im Test sollen alle \textit{Pfade}, die durch den Kontrolflussgraphen laufen, vorkommen.\\
    In diesem Fall spricht man von \textbf{Pfadüberdeckung} (\textit{path coverage} bzw. \textit{loop coverage}).
    Die \textbf{Boundary-Interior-Coverage} ist eine Form der \textbf{Pfadüberdeckung}, bei der bei Schleifen der abweisende Fall (\textit{boundary}) und mindestens zwei Durchläufe (\textit{interior}) genügen.\\
    Wird eine Boundary-Interior-Überdeckung erreicht, ist automatisch auch die \textbf{Zweigüberdeckung} erreicht.
\end{tcolorbox}