\section{Äquivalenzklassen}


\begin{tcolorbox}[title=Äquivalenzklassen]
    Die \textbf{funktionale Äquivalenzklassenbildung} gehört zu den \textbf{funktionsorientierten Testtechniken}, da Testfälle anhand der Spezifikation von Programmteilen erstellt werden. Unterschiedlichen Bereiche, in denen ein Testgegenstand \textbf{gleichartig reagiert}, werden \textbf{Äquivalenzklassen} genannt (vgl.~\cite[43]{Wed09c}).\\

    \noindent
    \textbf{Gültige Äquivalenzklassen} stellen hierbei zulässige Elemente dar, die vom Testgegenstand verarbeitet werden sollen, \textbf{ungültige Äquivalenzklassen} hingegen Elemente, die vom Testgegenstand abgelehnt werden sollen.\\

    \noindent
    Die Tests verwenden dann aus jeder Äquivalenzklasse einen \textbf{Stellvertreter}.
    Es wird unterschieden nach \textbf{Gutfällen} und \textbf{Schlechtfällen}.\\
    Es ist sinnvoll, an den \textbf{Grenzen von Äquivalenzklassen} zu testen, da hier erfahrungsgemäß Fehler auftreten\footnote{
    fällt unter die Kategorie ``Test spezieller Werte`` (vgl.~\cite[432]{Bal97})
    }.\\
    Hierzu werden für (un)gültige Äquivalenzklassen diejenigen Werte gewählt, die als Grenzwerte der möglichen Eingaben in Frage kommen.\\
    Für die gültigen Äquivalenzklassen gilt: Es entstehen nicht mehrere gültige Äquivalenzklassen durch die Auswahl der Grenzwerte, sondern nur Testdaten, die in den Testfällen vorkommen sollten (vgl.~\cite[Tabelle 2.2, Tabelle 2.3, 56 f.]{Lig09b}).
    Sind alle Grenzen getestet für eine Äquivalenzklasse, ist eine Auswahl von Testdaten aus der Mitte dieser Äquivalenzklasse sinnvoll (vgl.~\cite[431]{Bal97}).\\

    \noindent
    Bei der Ableitung von Testfällen wird i.d.R. so vorgegangen:
    \begin{enumerate}
        \item \textbf{Gutfälle}: Bilden von Testfällen die ausschliesslich gültige Äquivalenzklassen (\textit{gK})enthalten: So kann die Anwesenheit gültiger Repräsentanten über mögliche Werte(-bereiche) sichergestellt werden
        \item \textbf{Schlechtfälle}: Anschliessend Bilden von Schlechtfällen, wobei immer eine ungültige Äquivalenzklasse (\textit{uK}) mit ausschliesslich gültigen Äquivalenzklassen kombiniert wird.
        Dadurch wird für einen Testfall sichergestellt, dass der Fehler nur durch ein ungültiges Testdatum verursacht wird (vgl.~\cite[55 f.]{Lig09b}).
    \end{enumerate}

    \noindent
    \blockquote[{\cite[58]{Bal97}}]{
        Die funktionale Äquivalenzklassenbildung ist gut geeignet für den Test von Software, die nicht zustandsbasiert ist und bei der die Reaktionen nicht abhängig sind von komplizierten Eingabeverknüpfungen.
    }
\end{tcolorbox}

\begin{tcolorbox}[title=Beispiel,colback=white]
    Regeln für die Bildung von Äquivalenzklassen (vgl.~\cite[52 ff.]{Lig09b}):
    \begin{itemize}
        \item[] \textbf{Wertebereich}
        \begin{itemize}
            \item \textbf{1 gK} für einen Wert, der in dem Bereich liegt
            \item \textbf{2 uK} jeweils für Werte außerhalb des Bereiches
            \item \textbf{Beispiel}: $1 \leq n  \leq 100$, \textit{gK}: $1 \leq n  \leq 100$, \textit{uK}: $x < 1, x > 100$
        \end{itemize}
        \item[] \textbf{Aufzählung}
        \begin{itemize}
            \item \textbf{$n$ gK} für die $n$ Werte innerhalb der Aufzählung
            \item \textbf{1 uK} für einen Wert, der nicht Teil der Aufzählung ist
            \item \textbf{Beispiel}: \code{[A, D, F]}, \textit{$3$ gK}: \code{A, D, F}, \textit{1 uK}: alles anderes, bspw. \code{Z}
        \end{itemize}
        \item[] \textbf{Zwingend zu erfüllende Bedingung}
        \begin{itemize}
            \item \textbf{gK} für die Erfüllung der Bedingung
            \item \textbf{uK} für die Nichterfüllung der Bedingung
            \item \textbf{Beispiel}: erster Buchstabe ein \code{'W'}, \textit{gK}: \code{'W'}, \textit{uK}: alles andere, bspw. \code{'!'}
        \end{itemize}
    \end{itemize}
\end{tcolorbox}