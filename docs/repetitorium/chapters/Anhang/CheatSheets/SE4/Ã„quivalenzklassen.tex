\section{Äquivalenzklassen}


\begin{tcolorbox}[title=Äquivalenzklassen]
    \blockquote[{\cite[43]{Wed09c}}]{
        Die unterschiedlichen Bereiche, in denen ein Testgegenstand gleichartig reagiert, werden Äquivalenzklassen genannt.
    }\\
    \textbf{Gültige Äquivalenzklassen} stellen hierbei zulässige Elemente dar, die vom Testgegenstand verarbeitet werden sollen, \textbf{ungültige Äquivalenzklassen} hingegen Elemente, die vom Testgegenstand abgelehnt werden sollen.\\

    \noindent
    Die Tests verwenden dann aus jeder Äquivalenzklasse einen \textbf{Stellvertreter}.
    Es wird unterschieden nach

    \begin{itemize}
        \item \textbf{Gutfälle}: Tests, in denen nur Stellvertreter aus gültigen Äquivalenzklassen verwendet werden
        \item \textbf{Schlechtfälle}: Tests, in denen nur Stellvertreter aus ungültigen Äquivalenzklassen verwendet werden
    \end{itemize}

    Es ist sinnvoll, an den \textbf{Grenzen von Äquivalenzklassen} zu testen, da hier erfahrungsgemäß Fehler auftreten.

\end{tcolorbox}