\section{Einfache Mehrfachbedingungsüberdeckung}

\begin{tcolorbox}[title=Einfache Mehrfachbedingungsüberdeckung (minimale Mehrfachbedingungsüberdeckung)]
    Die \textbf{minimale Mehrfachbedingungsüberdeckung} wird erreicht, wenn \textit{jede} atomare Bedingung, also jeder \textit{direkte Vergleich}, und \textit{alle zusammengesetzten Bedingungen} einmal den Wahrheitswert \textit{wahr} und einmal den Wahrheitswert \textit{falsch} annehmen.\\
    Dadurch wird gleichzeitig eine \textbf{Zweigüberdeckung} erreicht.\\

    \noindent
    Bei zusammengesetzten Entscheidungen muss für eine \textbf{vollständige Evaluation} darauf geachtet werden, dass die Bedingungen einer Teilentscheidung so mit Werten versehen werden, dass die Gesamtentscheidung den Forderungen der minimalen Mehrfachbedingung entspricht:

    \begin{minted}{java}
        if (E) {
            if (A || B || C || D) { ...

    \end{minted}

    \noindent
    Der Gesamtausdruck  \code{(A || B || C || D)} wird zu \code{false} evaluiert, wenn \code{A}, \code{B}, \code{C}, \code{D} auf \code{false} gesetzt wird.\\


\end{tcolorbox}