\usepackage{lstmisc}\section{Anweisungsüberdeckung}

\begin{tcolorbox}[title=Anweisungsüberdeckung ($C_0$-Test)]

    Der Anweisungsüberdeckungstest (\textit{statement coverage test}), auch $C_0$-Test genannt, bietet die einfachste kontrollflussorientierte Testmethode (vgl.~\cite[85]{Lig09a}).\\

    \noindent
    Liegt eine \textbf{Anweisungsüberdeckung} vor, wurde im Test \textit{jede} Anweisung mindestens einmal ausgeführt: \textit{Jeder} Knoten eines \textbf{Kontrollflussgraphen} wurde dann mindestens einmal ausgeführt.\\

    \noindent
    Das Testmaß ist das Verhältnis der ausgeführten Anweisungen zu der im Prüfling vorkommenden Gesamtzahl an Anweisungen:

    \begin{equation}\notag
        C_{Anweisung} = \frac{\text{Anzahl ausgeführte Anweisungen}}{\text{Anzahl der Anweisungen}}
    \end{equation}


\end{tcolorbox}