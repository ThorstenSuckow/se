\section{Funktionsorientierte Verfahren}

\begin{tcolorbox}[title=Funktionsorientierte Verfahren]
   \textbf{Funktionsorientierte Tests} testen die gesamte Funktionalität ohne Berücksichtigung der inneren Struktur.\\
   Testfälle werden anhand der \textbf{Spezifikationen} abgeleitet.
   Funktionsorientierte Tests sind systematisch an der Überprüfung der \textbf{Soll-Funktionalität} ausgerichtet und gehören zu den \textbf{Black-Box-Tests}  (vgl.~\cite[50 f.]{Lig09b}).

    Ein Test kann als \textbf{vollständig} betrachtet werden, wenn die \textbf{gesamte Funktionalität} im Test erprobt worden ist.\\
        \item[] Bei dieser Art von Tests ist es möglich, dass nicht alle Codeteile ausgeführt werden, weil sie
        \begin{itemize}
            \item nicht erreichbar sind
            \item nicht spezifiziert sind, aber technisch notwendig
            \item fehlerhafterweise überflüssig sind
        \end{itemize}
        \noindent
        Es könnte kritisch sein, wenn diese Codeteile ungetestet bleiben.\\
        Auch Tests, die sich auf nicht-funktionale Anforderungen beziehen (etwa Performance), werden meistens ohne Berücksichtigung auf die \textit{innere Struktur} durchgeführt und deswegen in die Reihe der Black-Box-Tests eingereiht - auch, wenn sie nicht funktionsorientiert sind.
\end{tcolorbox}