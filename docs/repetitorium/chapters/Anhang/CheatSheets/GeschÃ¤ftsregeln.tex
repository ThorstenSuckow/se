\section{Geschäftsregeln}


\begin{tcolorbox}[title=Geschäftsregeln]
    \textbf{Geschäftsregeln} (\textit{Business Rules}), die in der Software implementiert werden müssen, verstehen sich als Repräsentanten fundamentaler Regeln von (Geschäfts-)Prozessen.\\
    Die Einhaltung dieser Regeln erhöht die Zuverlässigkeit der Anwendung und die Qualität der Daten.\\

    \noindent
    Geschäftsregeln können in folgende Typen eingeteilt werden:

    \begin{itemize}
        \item \textbf{Randbedingungen (Constraint)} beschränken, was Nutzer tun dürfen (Stichwörter \textit{muss}, \textit{darf nicht}, \textit{nur})
        \item \textbf{Aktionsauslöser (Action Enabler)} beschrieben, was nach Auslösung eines Ereignisses passiert (\textit{wenn \ldots, dann} [Aktion])
        \item \textbf{Wissenserzeuger (Inference)} erzeugen neues Wissen, wenn eine Bedingung zutrifft (\textit{wenn \ldots, dann} [neues Wissen])
        \item \textbf{Berechnung (Computation)}: Vorschriften zur Berechnung mit Formeln oder Algorithmen
    \end{itemize}

\end{tcolorbox}