\section{Ingenieurmäßiges Vorgehen}

\begin{tcolorbox}[title=Ingenieurmäßiges Vorgehen]
\textbf{Software Engineering} will durch \textbf{ingenieurmäßiges Vorgehen}eine wirtschaftlichere Entwicklung von Software gewährleisten.\\
Grundlage für Ingenieurmäßiges Vorgehen: finanzielle Interessen stehen hinter der Software Entwicklung.\\

\noindent
\textbf{Ingenieurmäßiges Vorgehen} bedeutet, dass \textbf{Software Engineering} auf wissenschaftlicher Basis und kodifizierter\footnote{
    \textit{kodifizieren}: Regeln und Prinzipien in einem systematischen Format zusammenfassen und festlegen.
} Erfahrung beruht.\\

\noindent
Ingenieurmäßiges Vorgehen beruht dabei auf Normen, Standards und Regeln:

\begin{itemize}
    \item\textbf{technische Ebene}: bspw. Vorlagen, wie Anforderungen an Software erfasst werden; Normen für Entwürfe von Software.
    \item \textbf{methodische Ebene}: bspw. festgeschriebene Reihenfolge von Tätigkeiten und zu erstellende Produkte für die Entwicklungsarbeit; Kriterien für den Einsatz technischer oder organisatorische Maßnahmen
\end{itemize}
\end{tcolorbox}