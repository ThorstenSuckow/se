\section{Strukturorientierte Testverfahren}

\begin{tcolorbox}
    Bei \textbf{strukturorientierten Testverfahren}, auch \textbf{White-Box-Tests} genannt, wird zur \textbf{Konstruktion der Testfälle} als auch zur \textbf{Bestimmung der Vollständigkeit} der Tests der Quellcode herangezogen.\\
    Dadurch soll sichergestellt werden, dass jede existierende Codezeile in Tests ausgeführt worden ist.
\end{tcolorbox}


\subsection*{Überdeckungsverfahren}

\begin{tcolorbox}[title=Anweisungsüberdeckung]
    Liegt eine \textbf{Anweisungsüberdeckung} vor, wurde im Test \textit{jede} Anweisung mindestens einmal ausgeführt: \textit{Jeder} Knoten eines \textbf{Kontrollflussgraphen} wurde dann mindestens einmal ausgeführt.
\end{tcolorbox}

\begin{tcolorbox}[title=Zweigüberdeckung]
    Eine \textbf{Zweigüberdeckung} liegt vor, wenn \textit{jede} Kante des Kontrollflussgraphen mindestens einmal ausgeführt worden ist.\\
    Liegt eine Zweigüberdeckung vor, ist gleichzeitig auch die \textbf{Anweisungsüberdeckung} erfüllt.\\
    Die Zweigüberdeckung bietet schon eine ziemlich vollständige Überdeckung, berücksichtigt aber Schleifen nicht genügend.
\end{tcolorbox}

\begin{tcolorbox}[title=Boundary-Interior-Coverage]
    Das Verhalten des Testgegenstandes beim abweisenden Fall oder bei mehreren Durchläufen einer Schleife führen zu der Forderung der Verallgemeinerung der Möglichkeiten: Im Test sollen alle \textit{Pfade}, die durch den Kontrolflussgraphen laufen, vorkommen.\\
    In diesem Fall spricht man von \textbf{Pfadüberdeckung} (\textit{path coverage} bzw. \textit{loop coverage}).
    Die \textbf{Boundary-Interior-Coverage} ist eine Form der \textbf{Pfadüberdeckung}, bei der bei Schleifen der abweisende Fall (\textit{boundary}) und mindestens zwei Durchläufe (\textit{interior}) genügen.\\
    Wird eine Boundary-Interior-Überdeckung erreicht, ist automatisch auch die \textbf{Zweigüberdeckung} erreicht.
\end{tcolorbox}

\begin{tcolorbox}[title=Einfache Mehrfachbedingungsüberdeckung]
    Die \textbf{einfache Mehrfachbedingungsüberdeckung} wird erreicht, wenn \textit{jede} atomare Bedingung, also jeder \textit{direkte Vergleich}, und \textit{alle zusammengesetzten Bedingungen} einmal den Wahrheitswert \textit{wahr} und einmal den Wahrheitswert \textit{falsch} annehmen\footnote{
        \textit{Maxterm}: Für genau eine Variablenbelegung falsch (Disjunktion: $A \lor B \lor C$); bzw. \textit{Minterm}: für genau eine Variablenbelegung wahr (Konjunktion: $A \land B \land C$)(vgl.~\cite[92]{Hof22})
    }.\\
    Dadurch wird gleichzeitig eine \textbf{Zweigüberdeckung} erreicht
\end{tcolorbox}