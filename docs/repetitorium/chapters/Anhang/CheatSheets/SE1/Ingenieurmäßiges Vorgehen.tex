\section{Ingenieurmäßiges Vorgehen}

\begin{tcolorbox}[title=Ingenieurmäßiges Vorgehen]
\textbf{Software Engineering} will durch \textbf{ingenieurmäßiges Vorgehen} eine wirtschaftlichere Entwicklung von Software gewährleisten.\\
Grundlage für Ingenieurmäßiges Vorgehen: finanzielle Interessen stehen hinter der Software Entwicklung.\\

\noindent
\textbf{Ingenieurmäßiges Vorgehen} bedeutet, dass \textbf{Software Engineering} auf wissenschaftlicher Basis und kodifizierter\footnote{
    \textit{kodifizieren}: Regeln und Prinzipien in einem systematischen Format zusammenfassen und festlegen.
} Erfahrung beruht.\\

\noindent
Ingenieurmäßiges Vorgehen beruht dabei auf Normen, Standards und Regeln:

\begin{itemize}
    \item\textbf{technische Ebene}: bspw. Vorlagen, wie Anforderungen an Software erfasst werden; Normen für Entwürfe von Software.
    \item \textbf{methodische Ebene}: bspw. festgeschriebene Reihenfolge von Tätigkeiten und zu erstellende Produkte für die Entwicklungsarbeit; Kriterien für den Einsatz technischer oder organisatorischer Maßnahmen
\end{itemize}
\end{tcolorbox}


\begin{tcolorbox}[title=Ziele des Ingenieurmäßigen Vorgehens]
Machbarkeit und maximalen Nutzen bei minimalen Kosten für Entwicklung, Wartung und Betrieb sicherstellen.\\

\blockquote[{\cite[2]{SR94}}]{
[\ldots] es wird versucht, den
Entwicklungsprozeß zu strukturieren und wiederholbar zu gestalten.
[\ldots]
Ziele, die durch ein ``ingenieurmäßiges`` Vorgehen bei der Softwareentwicklung erreicht werden sollen, sind unter anderem:
\begin{itemize}
    \item Korrektheit und Überprüfbarkeit,
    \item  Robustheit,
    \item  Erweiterbarkeit,
    \item  Wiederverwendbarkeit,
    \item  Effizienz,
    \item  Benutzerfreundlichkeit sowie
    \item  Wartungsfreundlichkeit des entstehenden Softwareprodukts.
\end{itemize}
}
\end{tcolorbox}