\section{Funktionale und nicht-funktionale Anforderungen}

\begin{tcolorbox}[title=Funktionale und nicht-funktionale Anforderungen]
    \textbf{Nicht-funktionale Anforderungen} werden in \textbf{Qualitätsanforderungen} und \textbf{Randbedingungen} unterschieden:

    \begin{itemize}
        \item \textbf{Qualitätsanforderungen} beschreiben die Qualität oder Eignung eines Systems (Performance, Benutzerfreundlichkeit, \ldots)
        \item \textbf{Randbedingungen} beschreiben technische Anforderungen (Programmiersprache, Framework, \ldots) oder organisatorische Anforderungen (Budget, Deadlines, \ldots)
    \end{itemize}\\

    \noindent
    \textbf{Funktionale Anforderungen} definieren, welche Funktionen eines zu entwickelnden Systems von Endanwendern oder anderen Systemen benutzt werden können (\cite[66]{Wed09}).
\end{tcolorbox}