\section{Datadictionary}

\begin{tcolorbox}[title=Dataditctionary]
    In einem \textbf{Datadictionary} wird festgehalten, in welchem Format welche Daten verarbeitet werden.\\
    Über solch ein zentrales Dokument kann erreicht werden, dass unterschiedliche Anwendungsteile einheitliche Datenformate nutzen.\\

    \noindent
    Zur Datendefinition gehören:
    \begin{itemize}
        \item Typ (\textit{String}, \textit{Integer}, \textit{Aufzählungen}, \ldots)
        \item Format (bspw. kalendarische Datumsformate)
        \item ggf. die Einheit
        \item andere \textit{Daten} bei zusammengesetzten Datentypen (bspw. bei Adressen)
    \end{itemize}

    \noindent
    Zu beachten ist, dass es sich bei dem Datadictionary um ein \textbf{vorläufiges Dokument} handelt: Es dient dazu, Informationen, die während der Anforderungsphase gesammelt wurden, systematisch zu sammeln. Die endgültige Datendefinition erfolgt in der Analysephase, weil dann genügend Informationen zusammengetragen wurden.    \\
    Das Datadictionary ist als strukturierte Merkhilfe zu verstehen.
\end{tcolorbox}
