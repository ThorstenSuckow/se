\section{Pflichtenheft}

\begin{tcolorbox}[title=Pflichenheft]
    Nachdem vom AG das \textbf{Lastenheft} erstellt wurde, erstellt der AN das \textbf{Pflichtenheft}:
    \blockquote[{\cite[306]{AABG14m}}]{
        Das Pflichtenheft ist nach DIN 69905 die Beschreibung der vom Auftragnehmer erarbeiteten Realisierungsvorgaben. Es sollte das vom Auftraggeber vorgegebene
        Lastenheft umsetzen. Es wird als Teil eines Angebots vom Auftragnehmer erstellt.
        Das Pflichtenheft enthält eine Zusammenfassung aller fachlichen Anforderungen,
        die das zu entwickelnde Produkt/Dienstleistung aus der Sicht des Auftragnehmers
        erfüllen muss. Dabei geht es um den Funktions-, Leistungs- und Qualitätsumfang
        des Produkts. Das Pflichtenheft muss so abgefasst sein, dass es als Basis eines juristischen Vertrags dienen kann.
    }
    \noindent
    Das Pflichtenheft dient dem \textit{Systemanalytiker} als Grundlage zur Erstellung des \textbf{OOA-Modells} (OOA=\textit{objektorientierte Analyse}).
\end{tcolorbox}
