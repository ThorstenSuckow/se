\section{Lastenheft}

\begin{tcolorbox}[title=Lastenheft]

    Dokument, in dem die Ziele des Kunden und der Projektumfang dokumentiert wird. Enthält die Anforderungen an das Produkt, damit es für den Kunden ein Erfolg wird.\\

    \blockquote[{\cite[305]{AABG14m}}]{
        Das Lastenheft wird vom Auftraggeber (AG) erarbeitet und hat
        nach DIN 69905 die Gesamtheit der Forderungen an die Lieferungen und Leistungen eines Auftragnehmers zum Inhalt. Der Zweck eines Lastenhefts ist die Einholung von Angeboten von potentiellen Auftragnehmern. Das Lastenheft beschreibt
        in der Regel also, was und wofür etwas gemacht werden soll. Der Adressat des
        Lastenhefts ist der (externe oder interne) Auftragnehmer.
    }
\end{tcolorbox}

\begin{tcolorbox}[title=Mögliche Gliederung]
    \begin{enumerate}
        \item Geschäftsanforderungen
        \begin{enumerate}[label*=\arabic*.]
            \item Hintergrund
            \item Geschäftsmöglichkeit
            \item Geschäftsziele und Erfolgskriterien
            \item Erfordernisse von Kunde oder Markt
            \item Geschäftsrisiken
        \end{enumerate}
        \item Vision der Lösung
        \begin{enumerate}[label*=\arabic*.]
            \item Vision
            \item Wichtigste Features
            \item Annahmen und Abhängigkeiten
        \end{enumerate}
        \item Fokus und Grenzen
        \begin{enumerate}[label*=\arabic*.]
            \item Umfang des ersten Release
            \item Umfang der folgenden Releases
            \item Begrenzung und Ausschlüsse
        \end{enumerate}
        \item Geschäftskontext
        \begin{enumerate}[label*=\arabic*.]
            \item Stakeholder
            \item Projektprioritäten
            \item Technische Anwendungsumgebung
        \end{enumerate}
    \end{enumerate}
\end{tcolorbox}