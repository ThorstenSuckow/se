\section{Mengengerüst}

\begin{tcolorbox}[title=Mengengerüst]
    Das \textbf{Mengengerüst} beschreibt die zu erwartende Anzahl der Daten\footnote{Kundendatensätze, Adressdatensätze, \ldots} als wichtiges Kriterium, da sich damit festgehaltene Kennzahlen direkt auf die Architektur bzw. Datenhaltung (Datenbanksysteme) auswirken kann.\\

    \noindent
    Hierbei handelt es sich streng genommen um eine \textit{nicht-funktionale} Anforderung, aber nicht-funktionale Anforderungen sind meist globaler gefasst als ein detailliertes Mengengerüst.\\

    \noindent
    \textbf{Mengengerüste} werden im Rahmen der \textbf{Anforderungsanalyse} (des \textit{Requirements Engineering}) gesammelt.

\end{tcolorbox}


