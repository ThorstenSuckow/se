\section{Alternative Entwicklungsmodelle}

\begin{tcolorbox}[title=Alternative Entwicklungsmodelle]
    \begin{itemize}
        \item \textbf{inkrementell}: Aufteilung der Anforderungen so, dass Teilsysteme an den Kunden ausgeliefert werden können; Teilsysteme werden \textbf{sequentiell} bearbeitet
        \item \textbf{iterativ}: das Projekt wird in Zeitabschnitte unterteilt, in denen die Anforderungen umgesetzt werden; das Spiralmodell  setzt zunächst die risikoreichsten um.\\
        Ergebnisse werden in darauffolgenden Iterationen weiterbearbeitet.\\
        Ein bekanntes iteratives Modell ist \textit{RUP}, bei dem einzelne Phasen Ergebnis-Artefakte liefern, z.B. Anwendungsfälle oder Klassendiagramme.
        \item \textbf{nebenläufig}: Aufgaben werden aufgeteilt in parallel bearbeitbare Aufgaben, die dann von den Mitarbeitern umgesetzt werden
    \end{itemize}

    \noindent
    Diese Modelle werden häufig nicht isoliert betrachtet, sondern je nach Projekt auch kombiniert, insb. bei \textbf{agilen Vorgehen}
\end{tcolorbox}
