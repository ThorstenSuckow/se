\section{Kriterien für Softwareentwicklung}

\begin{tcolorbox}[title=Kriterien für Softwareentwicklung]
    \begin{enumerate}
        \item \textbf{Nähe und Anzahl der Kunden}
        \item[] Individualentwicklung od. anonymer Markt?
        \item \textbf{Art der Benutzer, Art der Benutzung}
        \item[] Professionell, geschult? Lösung sollte effektives Werkzeug darstellen, um Produktivität zu erhöhen
        \item[] Seltene Benutzung? Lösung sollte robust sein und intuitiv zu bedienen
        \item \textbf{Größe/Komplexität der Software und des Projekts}
        \item[] Auswahl der Methoden maßgeblich davon abhängig, um Aufwand nicht unnötig in die Höhe zu treiben.
        ``Größe`` betrifft auch die Anzahl der involvierten Entwickler/Projektbeteiligter
        \item \textbf{Wie kritisch ist nichttechnisches Domänenwissen?}
        \item[] Fachexperten sollten bei komplexen Fragestellungen zur Verfügung stehen
        \item \textbf{Müssen Näherungslösungen verwendet werden?}
        \item[] Falls die Aufgaben mit der geg. Technik nicht vollständig gelöst werden können
        \item \textbf{Wie kritisch ist Effizienz?}
        \item[] Embedded Systems, wenig Hardware-Ressourcen: Kritisch!
        \item \textbf{Wie kritisch ist Verlässlichkeit?}
        \item[] Verlässlichkeit: Summe aus Zuverlässigkeit, Sicherheit, Verfügbarkeit und Schutz vor unberechtigtem Zugriff
    \end{enumerate}
\end{tcolorbox}