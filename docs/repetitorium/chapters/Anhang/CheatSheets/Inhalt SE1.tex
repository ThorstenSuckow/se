\section{Inhalt SE1}

\section*{1. Software und Software Engineering}
\subsection*{Lernziele}
\begin{itemize}
    \item Probleme bei der Softwareentwicklung benennen können, die zur Notwendigkeit des Software Engineerings führen
    \item Aufgaben und Ziele des Software Engineerings erklären / benennen können
    \item Probleme und Risiken des Software Engineerings kennen
\end{itemize}

\subsection*{Zusammenfassung}

\begin{itemize}
    \item \textbf{Software Engineering} beschäftigt sich mit dem \textbf{ingenieurmäßigen Vorgehen} bei der Entwicklung von Software
    \item dabei nutzt es ein \textbf{systematisches}, \textbf{diszipliniertes} und \textbf{messbares Vorgehen} auf \textbf{wissenschaftlicher Basis} und \textbf{kodifzierter Erfahrung}
    \item SE stellt \textbf{Techniken} und \textbf{Methoden} zur Verfügung, um Entwicklung möglich zu machen und Kosten bei Betrieb und Produktion zu sparen sowie den Nutzen zu maximieren
    \begin{itemize}
        \item \textbf{Techniken} und Werkzeuge sind u.a. \textbf{Normen} und \textbf{Standards} die bei der Entwicklung helfen
        \item \textbf{Methoden} und Verfahren beschreiben hingegen organisatorische Abläufe
    \end{itemize}
    \item die Wahl der Hilfsmittel hängt von unterschiedlichen Faktoren ab, u.a. der
    \begin{itemize}
        \item \textbf{Teamgröße}
        \item \textbf{Art des Produkts} (Individualsoftware, Standardsoftware)
    \end{itemize}
    \item ein wesentlicher Aspekt des SE ist die Berücksichtigung menschlicher Bedürfnisse der von der SD betroffenen Menschen (Kunden, Entwickler, Management, Benutzer\ldots) sowie der Organisation der Zusammenarbeit der Menschen
\end{itemize}

\section*{2. Uebersicht ueber die Phasen des Entwicklungszyklus}
\subsection*{Lernziele}
\begin{itemize}
    \item Bedeutung systematischen Vorgehens und Auswirkungen unsystematischen Vorgehens beurteilen können
    \item mit den Phasen der Softwareentwicklung vertraut sein
    \item die Inhalte der einzelnen Phasen kennen
\end{itemize}

\subsection*{Zusammenfassung}
\begin{itemize}
    \item Unsystematisches Vorgehen bei der Softwareentwicklung verursacht Verzögerungen und Probleme
    \item Systematisches Vorgehen, wie es das \textit{Wasserfallmodell} in seinen 6 Phasen definiert, kann helfen, eine höhere Qualität
    durch Planbarkeit zu erreichen.\\
    Die einzelnen Phasen mit ihren jeweiligen Ergebnissen lauten:\\
    \item[]
    \begin{enumerate}
        \item \textbf{Anforderungen}: Es entsteht ein \textbf{Lastenheft}.
        \item \textbf{Analyse}: Das Domänen-Modell wird erarbeitet, es entsteht ein \textbf{Fachkonzept}.
        \item \textbf{Entwurf}: Das Fachkonzept wird zu einem \textbf{DV-Konzept} ausgearbeitet.
        \item \textbf{Implementierung}: Es entstehen \textbf{Programmcode} und \textbf{Datenbankschemen}.
        \item \textbf{Test}: Die in der vorherigen Phase erstellten Systemkomponenten werden in das System integriert, es finden Systemtests und Integrationstests statt, das Ergebnis sind \textbf{Testfälle} und \textbf{-protokolle}.
        \item \textbf{Betrieb}: Das System geht in den Betrieb über und wird über Aktualisierungen und Fehlerbehebungen gepflegt.
    \end{enumerate}
\end{itemize}

\section*{Prozessmodelle}

\subsection*{Lernziele}
\begin{itemize}
    \item Grenzen des Wasserfallmodells einschätzen können
    \item Unterschiede zwischen iterativem und inkrementellem Vorgehen verstehen
    \item Anliegen und Grundprinzipien agilen Vorgehens kennen
    \item einige klassische und agile Vorgehensmodelle kennen (V-Modell XT, RUB, Scrum, XP)
    \item einschätzen können, welches Vorgehen wann zu bevorzugen ist
\end{itemize}

\subsection*{Zusammenfassung}

\begin{itemize}
    \item Anforderungen lassen sich bei Projekten vorab nicht genügend festlegen.
    \item Fehlt im Team die Erfahrung oder werden neue Technologien eingesetzt, ist ein ausgereifter Entwurf nicht machbar.
    \item Bei großen Projekten würde man unter diesen Voraussetzungen mit dem Wasserfallmodell nicht flexibel genug sein,
    um auf Änderungen reagieren zu können.
    \item Aus diesem Grund gibt es alternative Modelle, die eingesetzt werden können:
    \begin{itemize}
        \item \textbf{inkrementell}: Aufteilung der Anforderungen, so dass Teilsysteme umgesetzt und an den Kunden ausgeliefert werden können.
        Die Teilsysteme werden sequentiell bearbeitet.
        \item \textbf{iterativ}: In Iterationen wird das Projekt in Zeitabschnitte unterteilt, in denen die Anforderungen umgesetzt werden; entsprechend dem Spiralmodell zunächst die risikoreichsten.
        Vorhergehende Ergebnisse werden in darauffolgenden Iterationen weiterbearbeitet.
        Ein bekanntes iteratives Modell ist \textit{RUP}, bei dem einzelne Phasen Ergebnis-Artefakte liefern, wie Anwendungsfälle oder Klassendiagramme.
        \item \textbf{nebenläufig}: Die Aufgaben werden aufgeteilt in parallel (oder nacheinander) bearbeitbare Aufgaben, die dann von den Mitarbeitern umgesetzt werden.
    \end{itemize}
    \item Die genannten Modelle werden häufig nicht isoliert betrachtet, sondern je nach Projekt auch kombiniert, insb. bei dem \textbf{agilen Vorgehen}.
\end{itemize}

\section*{Requirements Engineering}

\subsection*{Lernziele}
\begin{itemize}
    \item wissen, wie man bei der Ermittlung von Anforderungen vorgeht
    \item wissen, auf welche typischen Probleme man bei der Ermittlung von Anforderungen stößt
    \item Unterschiede zwischen funktionalen und nicht-funktionalen Anforderungen verstehen
    \item funktionale Anforderungen als Anwendungsfälle dokumentieren können
    \item Bedeutung nicht-funktionaler Anforderungen einschätzen können, sowie diese erfassen und dokumentieren können
    \item Pflichten- und Lastenhefte erstellen können
\end{itemize}

\subsection*{Zusammenfassung}

\begin{itemize}
    \item \textbf{Anforderungen} sind Bedürfnisse der Stakeholder und Kunden an das Softwaresystem.
    \item Vor der Erarbeitung der Anforderung sollte eine \textbf{Domänenanalyse} erfolgen, um sich mit den Begriffen und Konzepten der Domäne vertraut zu machen.
    \item daraufhin wird der wirtschaftliche Nutzen, den sich der Kunde verspricht, als \textbf{Geschäftsanforderungen} in einem \textbf{Lastenheft} festgehalten.
    \item Anschließend werden die Anforderungen aller involvierten Personen (\textbf{Stakeholder}) ermittelt, insbesondere der \textbf{Endanwender}.
    \item Diese Anforderungen werden grob unterteilt in \textbf{funktionale} und \textbf{nicht-funktionale} Anforderungen.
    \item funktionale Anforderungen sind hierbei die Funktionen, die der Endanwender von dem System erwartet.
    \begin{itemize}
        \item funktionale Anforderungen werden als \textbf{Use Case} oder \textbf{User Story} erfaßt.
        \item Anwendungsfälle (Use Cases) können hierbei sehr detailliert und formell beschrieben werden.
        \item User Stories sind i.d.R. wenig detailliert und beschreiben nur kurz den Umgang des Endanwenders mit dem System.
        \item für die Planung von Steueranwendungen (Hardware) können auch \textbf{Ereignistabellen} verwendet werden.
    \end{itemize}
    \item nicht-funktionale Anforderungen sind technische und organisatorische Anforderungen, die üblicherweise in Form von Listen gesammelt werden
    \item \textbf{Regeln}, und \textbf{Formate von Daten} werden gesondert erfaßt, als \textbf{Business Rules} in Listen bzw. in \textbf{Datadictionaries} und der Angabe von \textbf{Mengengerüste}.
    \item Die Aufstellung all dieser Anforderung wird bei dem \textbf{Wasserfallmodell} oft in Form des \textbf{Pflichtenhefts} realisiert.
\end{itemize}
