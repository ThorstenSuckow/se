\section{Objektorientierter Entwurf}


\begin{tcolorbox}[title=Objektorientierter Entwurf]
    \noindent
    Der \textbf{objektorientierte Entwurf} beinhaltet
    \begin{itemize}
        \item das Erstellen von Klassen samt deren Attribute und Methode
        \item das Festlegen der Assoziationen und Vererbungsbeziehungen
        \item das Beschreiben des Zusammenspiels der Instanzen
    \end{itemize}
Der Entwurf geschieht auf Grundlage der Anforderungen und der Ergebnisse der Analyse und innerhalb der gewählten Architektur.\\
Aus den Anforderungen sind für den Entwurf die \textbf{funktionalen Anforderungen} wichtig - die \textbf{nicht-funktionalen Anforderungen} sind bereits in der Architektur berücksichtigt.\\
Die \textbf{Architektur} liefert die Bauteile für die Software, der \textbf{Entwurf} konkretisiert diese dann anhand der funktionalen Anforderungen.\\
Die Ergebnisse der \textbf{Analyse} in Form des \textbf{Domänenmodells} ist Grundlage für die \textit{Datenhaltung} in der Anwendung und für eventuelle \textit{Datenbankschemas}.\\
Die \textbf{Definition} der Schnittstellen, insb. der GUI, bilden die Basis für den Entwurf der technischen Umsetzung der Schnittstelle.
\end{tcolorbox}