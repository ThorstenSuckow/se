\section{Vorgehen beim Entwerfen}

\begin{tcolorbox}[title=Vorgehen beim Entwerfen]
    \begin{enumerate}
        \item Aufteilung der User Story in Anwendungsfälle
        \item \textbf{Analyseergebnis}:
        \begin{itemize}
            \item  Aus den funktionalen Anforderungen extrahiert er das \textbf{Domänenmodell}, das in ein (wenig detailliertes) UML-Klassendiagramm überführt wird
            \item  ein \textbf{fachlicher Dialogentwurf} zeigt eine grobe Skizze des UI mit einigen Interaktionselementen
        \end{itemize}
        \item \textbf{Entwurf}:
        \begin{itemize}
            \item ein \textbf{technischer Dialogentwurf} erweitert den fachlichen Dialogentwurf um konkrete Typen der einzelnen Interaktionselemente und verwendeten Komponenten
            \item das Klassendiagramm der Analyse wird erweitert, Klassendiagramme der Komponenten werden hinzugefügt und um technischen Details ergänzt
        \end{itemize}
    \end{enumerate}
\end{tcolorbox}