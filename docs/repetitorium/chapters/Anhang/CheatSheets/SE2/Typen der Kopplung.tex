\section{Typen der Kopplung}


\begin{tcolorbox}[title=Typen der Kopplung]
    \begin{itemize}
        \item \textbf{Content-Kopplung}
        \begin{itemize}
            \item ein Teil ändert interne Daten eines anderen Teils
            \item die ändernde Klasse ist von der Struktur der zu ändernden Klasse \textbf{abhängig}
        \end{itemize}
        \item \textbf{Common-Kopplung}
        \begin{itemize}
            \item bezeichnet Kopplung über die gemeinsame Verwendung globaler Variablen
        \end{itemize}
        \item \textbf{Stamp-Kopplung}
        \begin{itemize}
            \item ein Objekt wird als Argument einer Methode verwendet
            \item wird der öffentliche Teil der Klasse des Objektes geändert, muss auch die aufrufende Methode geändert werden
        \end{itemize}
        \item \textbf{Data-Kopplung}
        \begin{itemize}
            \item je mehr Argumente eine Methode hat, desto größer ist die Kopplung mit der benutzenden Komponente
        \end{itemize}
        \item \textbf{Routine Call-Kopplung}
        \begin{itemize}
            \item eine Methode ruft eine andere Methode auf
            \item werden sehr viele Methode eines anderen Teils aufgerufen, sollte man die Zerlegung überlegen
            \item werden Methoden immer in gleicher Sequenz aufgerufen, könnte man überlegen, die Sequenz auch als eigenständige Methode zusammenfassen
        \end{itemize}
    \end{itemize}
\end{tcolorbox}
