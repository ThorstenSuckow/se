\section{Liskovsches Substitutionsprinzip / Polymorphie}

\begin{tcolorbox}[title=Liskovsches Substitutionsprinzip / Polymorphie]
Eine Bedeutung der Vererbungsbeziehung ist, dass eine Instanz einer Klasse genauso verwendet werden kann wie eine Instanz seiner übergeordneten Klasse.\\
Allgemein fasst man dies unter \textbf{Polymorphie} zusammen.\\
Als Konsequenz folgt, dass immer dann, wenn ein bestimmter Typ gefordert wird, auch sein Untertyp erlaubt ist (vgl.~\cite[466]{Ull23}), was der Kern des \textbf{Liskovschen Substitutionsprinzips} ist.\\

\noindent
Für die Beziehung zwischen einer abgeleiteten Klasse und ihrer Superklasse muss gelten:

\begin{itemize}
    \item \textbf{kann verwendet werden als}
    \item \textbf{ist ein}
\end{itemize}

In \cite[174 ff.]{LG00} formulieren \textit{Liskov und Guttag} am Beispiel der Sprache Java drei Regeln, die abgeleitete Typen befolgen müssen, damit das \textbf{Ersetzbarkeitsprinzip} erfüllt werden kann:
\begin{itemize}
    \item \textbf{Signature Rule}: Die Subtypen müssen die Methoden des Supertyps implementieren, bei kompatibler Signatur
    \item \textbf{Methods Rule}: Das Verhalten der im Subtyp implementierten Methoden muß das gleiche Verhalten aufweisen, wie entsprechende Methoden des Supertypes
    \item \textbf{Properties Rule}: Der Subtyp muss die gleichen Eigenschaften aufweisen (bspw. \textit{Invarianten}) wie der Supertyp
\end{itemize}
\end{tcolorbox}
