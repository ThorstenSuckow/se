\section{Liskovsches Substitutionsprinzip / Polymorphie}

\begin{tcolorbox}[title=Liskovsches Substitutionsprinzip / Polymorphie]
Eine Bedeutung der Vererbungsbeziehung ist, dass eine Instanz einer Klasse genauso verwendet werden kann wie eine Instanz seiner übergeordneten Klasse.\\
Allgemein fasst man dies unter \textbf{Polymorphie} zusammen.\\
Als Konsequenz folgt, dass immer dann, wenn ein bestimmter Typ gefordert wird, auch sein Untertyp erlaubt ist (vgl.~\cite[466]{Ull23}), was der Kern des \textit{Liskovschen Substitutionsprinzips}\footnote{
    ``Wikipedia - Liskov substitution principle``: \url{https://en.wikipedia.org/wiki/Liskov_substitution_principle}, abgerufen 11.04.2024
}\footnote{
    s.a. ``Wikipedia - Covariance and Contravariance``: \url{https://en.wikipedia.org/wiki/Covariance_and_contravariance_(computer_science)}, , abgerufen 11.04.2024
} ist.\\

\noindent
Für die Beziehung zwischen einer abgeleiteten Klasse und ihrer Superklasse muss gelten \textbf{kann verwendet werden als} und \textbf{ist ein}.
\end{tcolorbox}
