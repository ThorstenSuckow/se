\section{Analyse- und Entwurfsphase}

\begin{tcolorbox}[title=Analysephase]
    Nach der \textbf{Anforderungsphase} wird in der \textbf{Analysephase} ein formales Modell sowohl des Problems in seinem Umfeld als auch der Lösung erstellt.\\
    Die Analyse beschreibt fachliche Zusammenhänge, keine technische.
\end{tcolorbox}

\begin{tcolorbox}[title=Entwurfsphase]

In der \textbf{Entwurfsphase} wird die \textbf{Architektur} der Anwendung bestimmt, unter Berücksichtung technischer Randbedingungen, die im Rahmen nicht-funktionaler Anforderungen in der Anforderungsphase gesammelt wurden und vom Kunden stammen.\\
Die Architektur schließt u.a. ein:

\begin{itemize}
    \item Programmiersprache
    \item Frameworks
    \item Einsatz von Datenbanken
    \item \ldots
\end{itemize}

\noindent
Meist bleiben im Anschluss technische Fragen offen, die im weiteren Verlauf dieser Phase beantwortet werden, z.B.:

\begin{itemize}
    \item Welche \textbf{Klassen} aus der \textbf{Analyse} können übernommen werden?
    \item Wie werden Klassen in einem relationalen Datenbank-Schema gespeichert?
    \item Welche Steuerungsklassen müssen für die GUI implementiert werden?
\end{itemize}
\end{tcolorbox}
