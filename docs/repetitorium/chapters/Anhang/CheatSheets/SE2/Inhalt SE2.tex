\section{Inhalt SE2}

\section*{1. Objektorientierte Analyse}

\subsection*{Lernziele}
\begin{itemize}
    \item verstehen, wie und wozu Modelle in Analyse und Entwurf verwendet werden
    \item wissen, wie Objekte und Klassen zur Analyse von Anforderungen der Software verwendet werden
    \item Klassendiagramme der Analyse auf systematische Art und Weise erstellen können
    \item Nutzen von Mustern beurteilen können
    \item den Aufbau von Mustern verstehen
    \item die wichtigsten Analysemuster kennen und verwenden können
    \item Konzepte zum systematischen Entwurf und zur ergonomischen Gestaltung von Benutzeroberflächen kennen
\end{itemize}

\subsection*{Zusammenfassung}
\begin{itemize}
    \item in der \textbf{Analyse} werden \textbf{Modelle} genutzt, um Anforderungen zu verstehen und Lösungen aus fachlicher Sicht zu entwickeln
    \item hierzu werden Modelle des Umfelds und der Lösung auf formalisierte Art erstellt
    \item diese dienen den Modellen des \textbf{Entwurfs}
    \item Modelle bestehen aus \textbf{statischen} und \textbf{dynamischen} Modellteilen, wobei statische Modellteile die Teile und ihre Beziehungen untereinander modellieren, und dynamische das Zusammenspiel der Teile beschreiben
    \item In der OO-Analyse wird statische Modellierung mit \textbf{Klassendiagrammen} realisiert.
    \item[] Dynamische Modelle, die Zustände und Übergänge von Zuständen beschreiben, werden mittels \textbf{Zustandsdiagrammen} modelliert.
    \item[] Beschreiben dynamische Modelle Abläufe von Aktivitäten, nutzt man \textbf{Aktivitätsdiagramme}.
    \item Bei der Analyse kommen außerdem Analysemuster zum Einsatz: Muster beschreiben Lösungskizzen verallgemeinerter Probleme
    \item[] Sie helfen, indem sie \textbf{bewährte Lösungen} beschreiben und ein \textbf{Kommunikationsmittel} darstellen
    \item Die wichtigsten Analysemuster sind \textbf{Exemplartyp}, \textbf{Wechselnde Rollen} und \textbf{Allgemeine Hierarchie}.
    \item Die Gestaltung der Benutzeroberfläche ist Bestandteil der Analyse, wobei logische Struktur, die Informationsarchitektur und die physische Gestalt entworfen werden.
    \item[] Ergonomie und Barrierefreiheit sind wichtige Punkte, die dabei beachtet werden müssen.
\end{itemize}

\section*{2. Architektur}

\subsection*{Lernziele}
\begin{itemize}
    \item verstehen, was Architektur ist und welchen Einflüssen sie unterliegt
    \item die wichtigsten Architekturmuster kennen und einordnen können
\end{itemize}

\subsection*{Zusammenfassung}

\begin{itemize}
    \item Der grundlegende Aufbau einer Software wird über die \textbf{Softwarearchitektur} beschrieben.
    \item Die Architektur sollte aufgrund ihrer grundsätzlichen Bedeutung von Anfang an berücksichtigt werden.
    \item Grundlage für eine Architektur sind Anforderungen und Randbedingungen, wobei \textbf{nicht-funktionale Anforderungen} und \textbf{Randbedingungen} entscheidenden Einfluss haben.
    \item Hilfsmittel für eine Softwarearchitektur sind \textbf{Taktiken}, \textbf{Muster} (bspw. Client-Server, Schichtenbildung) und \textbf{Referenzarchitekturen}.
    \item Die Dokumentation von Architekturen erfolgt in \textbf{Sichten}.
\end{itemize}

\section*{3. Objektorientierter Entwurf}

\subsection*{Lernziele}
\begin{itemize}
    \item verstehen, wie und wozu Modelle im Entwurf verwendet werden
    \item Zusammenhang zwischen Modellen der Analyse und des Entwurfs verstehen
    \item Klassendiagramme des Entwurfs auf systematische Art und Weise erstellen können
    \item bei der Erstellung von Entwürfen Entwurfsmuster verwenden können
    \item die Qualität von Entwürfen beurteilen und systematisch verbessern können
\end{itemize}

\subsection*{Zusammenfassung}

\begin{itemize}
    \item im \textbf{Entwurf} werden Klassen und Dateien eines zu entwerfenden Systems \textbf{geplant}
    \item dabei beruht der Entwurf auf den \textbf{Anforderungen}, den \textbf{Ergebnissen der Analyse} und der \textbf{Architektur}
    \item die Klassen des Entwurfs entsprechen den Klassen der verwendeten objektorientierten Sprache
    \item die \textbf{Struktur der Klassen} wird mit \textbf{UML-Klassendiagrammen} definiert
    \item \textbf{Abläufe} und das \textbf{Zusammenspiel} von Klassen wird mit \textbf{UML-Sequenzdiagrammen} modelliert
    \item der Entwurf sollte sinnvollerweise \textbf{schrittweise} (\texit{iterativ}) entwickelt werden
    \item hierbei helfen grundlegende Prinzipien wie ``\textit{Reise mit leichtem Gepäck}``
    \item Entwurfsmuster wie \textbf{Beobachter}, \textbf{MVC} und \textbf{Fassade} sind wichtige Hilfsmittel beim Entwurf
    \item anhand von \textbf{Prinzipien guten Entwurfs} kann die \textbf{Qualität eines Entwurfs} beurteilt werden
    \item das wichtigste Prinzip hierbei ist die Aufteilung der Gesamtaufgabe in \textbf{kleine Teile}, wobei die Teile lose untereinander gekoppelt sein sollen, was durch eine \textbf{hohe Kohäsion} (\textit{Zusammenhalt}) sowie \textbf{Vermeidung starker Kopplung} (Globale Variablen, lange formale Parameterlisten in den Methoden-Signaturen usw.) erreicht werden kann, außerdem durch die Beachtung guten Entwurfs wie Vermeidung zirkulärer Abhängigkeiten
    \item die Qualität eines Entwurfs bleibt gut, indem der Code durch \textbf{Refactorings} stetig verbessert wird
\end{itemize}
