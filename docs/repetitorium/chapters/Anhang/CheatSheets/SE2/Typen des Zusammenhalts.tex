\section{Typen des Zusammenhalts}


\begin{tcolorbox}[title=Typen des Zusammenhalts]

    Mit Absteigender Richtung wird der Zusammenhalt schwächer:

\begin{itemize}
    \item \textbf{Funktionaler Zusammenhalt}
    \begin{itemize}
        \item ähnliche Funktionalität wird in gleichen Softwareteilen implementiert
        \item die Funktionalität ist nach Möglichkeit nicht von anderer Funktionalität abhängig und besitzt keine Nebeneffekte
        \item Idealfall: Eine Operation liefert ein bestimmtes Ergebnis und ist unabhängig von vorausgegangen Aufrufen, internen Zuständen oder Teilen des Systems
        \item Forderung nach Unabhängigkeit allerdings meist nicht möglich, da in der Objetorientierung Objekte i.d.R. über Attribute interne Zustände besitzen (s. Abbildung~\ref{fig:adt}); außerdem benutzen Klassen andere Klassen, oder externe Systeme wie Datenbanken
    \end{itemize}

    \item \textbf{Schichten-Zusammenhalt}
    \begin{itemize}
        \item Teile, die ähnliche Services für andere Teile zur Verfügung stellen, werden in Schichten zusammengefügt (s. Abschnitt~\ref{sec:architekturmuster})
    \end{itemize}
    \item \textbf{kommunikativer Zusammenhalt}
    \begin{itemize}
        \item Teile, die auf gleichen Daten operieren, gehören zusammen
        \item der Zusammenhalt ist an der Stelle schwächer als der Schichten-Zusammenhalt, da man i.d.R. nicht Geschäftslogik und Zugriff auf Datenbanken in derselben Klasse implementiert, auch, wenn das Model dieselben Daten benötigt
    \end{itemize}
    \item \textbf{Utility-Zusammenhalt}
    \begin{itemize}
        \item unter ``\textit{Utility}`` werden hier Teile gemeint, die ähnliche Funktionalität bereitstellen, sich aber logisch nicht ordnen lassen (bspw. in Schichten)\footnote{s. bspw. Klassen aus \url{https://docs.oracle.com/en/java/javase/21/docs/api/java.base/java/util/package-summary.html}, abgerufen 19.04.2024}
    \end{itemize}
\end{itemize}
    \end{tcolorbox}