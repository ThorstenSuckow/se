\section{Dialoggestaltung}

\begin{tcolorbox}[title=Grundsätze der Dialoggestaltung]
    Für die Gestaltung von Benutzeroberflächen sind 7 Gestaltungsgrundsätze relevant (nach \textbf{DIN EN ISO 9241}\footnote{
        s. \url{https://de.wikipedia.org/wiki/ISO_9241}, abgerufen 13.04.2024
    }):

    \begin{itemize}
        \item \textbf{Aufgabenangemessenheit}: Ein Dialog ist den Aufgaben angemessen, wenn er die Benutzer unterstützt, die Aufgaben effektiv und effizient zu erledigen.
        \item \textbf{Selbstbeschreibungsfähigkeit}: Für die Benutzer ist jederzeit offensichtlich, im welchem Dialog und an welcher Stelle im Dialog sie sich befinden, welche Handlungen unternommen werden und wie diese ausgeführt werden können.
        \item \textbf{Steuerbarkeit}: Der Benutzer ist in der Lage, den Dialogablauf zu starten und seine Richtung und Geschwindigkeit zu ändern, bis das Ziel erreicht ist (bspw. Rückgängigmachfunktion, Vor-/Zurück-Buttons \ldots).
        \item \textbf{Erwartungskonformität}: Ein Dialog ist erwartungskonform, wenn er konsistent ist und den Merkmalen des Benutzers entspricht (Kenntnisse Arbeitsgebiet, Ausbildung, Erfahrung, allgemein anerkannte Konventionen, bspw. einheitliche Verwendung von Funktionscodes und Funktionstasten in allen Dialogen).
        \item \textbf{Fehlertoleranz}: Das beabsichtigte Arbeitsergebnis wird trotz erkennbar fehlerhafter Eingaben mit keinem oder \textbf{minimalen Korrekturaufwand} seitens des Benutzers erreicht.
        \item \textbf{Individualisierbarkeit}: Das Dialogsystem erlaubt Anpassungen an Erfordernisse der Arbeitsaufgabe sowie individuelle Fähigkeiten und Vorlieben des Benutzers.
        \item \textbf{Lernförderlichkeit}: Der Benutzer wird beim Erlernen des Dialogs unterstützt und angeleitet (bspw. über ``Shortcuts`` bzw. \textit{Mnemonics}\footnote{
            s. \url{https://en.wikipedia.org/wiki/Mnemonics_(keyboard)}, abgerufen 26.04.2024
        } (vgl.~\cite[147]{Rau07f})).

    \end{itemize}
\end{tcolorbox}