\section{Modell}



\vspace{2mm}
\begin{tcolorbox}[title=Arbeitsdefinition ``Modell``]
    Ein \textbf{Modell} ist ein Produkt des \textbf{Modellierungsvorgangs}.\\

    \noindent
    Es beschreibt tatsächliche oder gedachte Gegenstände oder Konzepte und deren Beziehungen.\\

    \noindent
    Ein Modell erfasst diese Konzepte nicht vollständig, sondern \textbf{abstrakt}, also verkürzt und vereinfacht (vgl. \cite[2]{Wed09b}).\\

    \noindent
    \textbf{Statische Modelle} modellieren die Bausteine eines Systems und ihre Beziehungen, \textbf{dynamische Modelle} das Zusammenspiel dieser Bausteine.\\
    Zu den statische Modellen gehören bspw.
    \begin{itemize}
        \item Klassen
        \item Assoziationen
        \item Attribute
        \item Vererbungs-/Implementierungsbeziehungen
    \end{itemize}
    \noindent
    Dynamische Modelle bestehen aus:
    \begin{itemize}
        \item Zuständen
        \item Zustandsübergänge
        \item Aktivitäten
    \end{itemize}
\end{tcolorbox}
\vspace{2mm}