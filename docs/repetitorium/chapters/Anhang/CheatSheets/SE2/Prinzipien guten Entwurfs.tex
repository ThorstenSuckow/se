\section{Prinzipien guten Entwurfs}

\begin{tcolorbox}[title=Prinzipien guten Entwurfs]

    \begin{itemize}
        \item \textbf{Generelle Prinzipien}
        \begin{itemize}
            \item Funktionsfähigkeit
            \item Änderbarkeit
        \end{itemize}
        \item \textbf{Grundlegende Prinzipien}
        \begin{itemize}
            \item Abstraktion:  das Herausheben von Wesentlichem;  ein Teil ist leichter zu verstehen, wenn nicht alle Details verstanden werden müssen, sondern nur das Wesentliche.
            \item Law of Demeter: ``Talk to friends, not to strangers.
            \item Vermeidung zirkulärer Abhängigkeiten
            \item Liskovsches Substitutionsprinzip: Objekte einer Klasse können durch andere Objekte davon abgeleiteter Klassen ersetzt werden; wenn ein bestimmter Typ gefordert ist, ist auch sein Untertyp erlaubt.
        \end{itemize}
        \item \textbf{Teile und Herrsche}
        \begin{itemize}
            \item Lösungen in Teillösungen aufteilen
            \item Systeme $\rightarrow$ Subsysteme $\rightarrow$ Pakete $\rightarrow$  Klassen $\rightarrow$ Methoden
        \end{itemize}
        \item \textbf{Hohe Kohäsion}
        \begin{itemize}
            \item Zusammenhang innerhalb der Teile hoch, Abhängigkeiten nach außen minimal
        \end{itemize}
        \item \textbf{Lose Kopplung}
        \begin{itemize}
            \item Kopplungen möglichst schwach halten, da die Teile der Software unabhängig voneinander sein sollen.
        \end{itemize}
    \end{itemize}

\end{tcolorbox}