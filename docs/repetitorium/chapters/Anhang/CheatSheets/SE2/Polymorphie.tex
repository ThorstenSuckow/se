\section{Polymorphie}

\begin{tcolorbox}[title=Polymorphie-Prinzip]
    \blockquote[{\cite[67]{Oes05}}]{
        Polymorphie heißt, dass eine Operation sich (in unterschiedlichen Klassen) unterschiedliche verhalten kann. Es gibt zwei Arten der Polymorphie: statische Polymorphie (Überladung) und dynamische Polymorphie.
    }\\
    \noindent
    Dynamische Polymorphie wird durch \textit{late binding} realisiert und bedeutet, dass die Zuordnung von Methodenaufrufen zur Laufzeit geschieht: Die aufzurufende Methode wird einer konkreten Instanz einer Klasse zugeordnet,
    die eine überschriebene Methode beinhalten kann, die sich von dem Verhalten der Methode der Elternklasse unterscheiden kann.
\end{tcolorbox}