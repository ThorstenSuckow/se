\section{Paketdiagramm}

\begin{tcolorbox}[title=Paketdiagramm]
    \textbf{Paketdiagramme} zeigen Pakete und ihre Abhängigkeiten und werden in frühen Designphasen genutzt, um Strukturen (großer Systeme) aufzuzeigen.\\

    \noindent
    Elemente sollten so in Paketen gruppiert werden, dass ihr \textbf{funktionaler Zusammenhalt} (\textit{Kohäsion}, s. Abschnitt~\ref{subsec:hohe-kohasion}) klar wird.\\
    Hierdurch kann vermieden werden, dass Änderungen einer Klasse eines Paketes auch Änderungen außerhalb des Paketes erfordern.\\
    Außerdem erhöht sich die Wahrscheinlichkeit, dass einzelne Pakete als solche in anderen Projekten wiederverwendet werden können.\\

    \noindent
    In Paketdiagrammen können Beziehungen über folgende Schlüsselwörter deutlich gemacht werden:

    \begin{itemize}
        \item \guillemotleft import\guillemotright
        \item[] $\rightarrow$ \textbf{public-Import} zwischen Quell- und Zielpaket; Quellpaket kann die öffentlichen Elemente des Zielpaketes unter Verwendung des unqualifizierten und des qualifizierten Namens verwenden; die importierten Elemente sind auch für Pakete sichtbar, die das Quellpaket importieren
        \item \guillemotleft access\guillemotright
        \item[] $\rightarrow$ \textbf{privater Import} zwischen Quell- und Zielpaket; Zugriff auf Elemente des importierten Paketes ohne qualifizierenden Namen möglich; der Import eines Paketes in \textbf{Java} entspricht der \guillemotleft access\guillemotright-Beziehung
        \item \guillemotleft merge\guillemotright
        \item[] $\rightarrow$ Bei einem \textbf{merge} werden Elemente eines Zielpaketes in das Quellpaket kopiert und können dann verändert werden
   \end{itemize}
\end{tcolorbox}