\section{Inhalt SE3}

\section*{1. Übersicht UML}

\subsection*{Lernziele}
\begin{itemize}
    \item einen Einblick in den Sprachaufbau der UML 2.0 gewinnen
    \item einen Überblick über die Diagramme der UML 2.0 und deren Zweck haben
\end{itemize}

\subsection*{Zusammenfassung}
\begin{itemize}
    \item Die \textbf{UML} ist eine weit verbreitete Modellierungssprache zur Modellierung objektorientierter Softwaresysteme.
    \item Die UML 2.0 Spezifikation besteht aus vier Teilen:
    \begin{enumerate}
        \item \textbf{Superstructure}: Diagrammtypen
        \item \textbf{Infrastructure}: Modellierungskonzepte
        \item \textbf{OCL}: formale Sprache zur Formulierung von Einschränkungen
        \item \textbf{Diagram Interchange}: Austauschformate
    \end{enumerate}
    \item Die UML kann im \textbf{Sketching} oder \textbf{Blueprint Modus} eingesetzt werden, oder als Grundlage für automatische Codegenerierung
    \item Die UML ist an keinen speziellen Entwicklungsprozess gebunden
\end{itemize}

\section*{2. Klassendiagramme}

\subsection*{Lernziele}
\begin{itemize}
    \item wissen, wie Klassendiagramme im Entwicklungsprozess eingesetzt werden
    \item Klassendiagramme für Analyse- und Designentscheidungen einsetzen können
    \item die wichtigsten Modellierungskonzepte kennen
    \item sinnvolle Klassendiagramme erstellen und Java-Implementierungen ableiten können
    \item Klassendiagramme aus Java Quellcode erzeugen können
\end{itemize}

\subsection*{Zusammenfassung}

\begin{itemize}
    \item \textbf{Klassendiagramme} dienen zur \textbf{strukturellen Darstellung} von Softwaresystemen
    \item hierbei werden Klassen und \textbf{Beziehungen} zwischen Klassen dargestellt
    \item Properties können durch die \textbf{Attribut}- oder \textbf{Assoziationsnotation} modelliert werden
    \item Operationen werden als separate Zeile in ihrem Compartment dargestellt
    \item im Klassendiagramm können auch \textbf{Generalisierungsbeziehungen} modelliert werden
    \item \textbf{Abhängigkeitsbeziehungen} drücken ein Verhältnis zwischen \textbf{Client} und \textbf{Supplier} aus
\end{itemize}

\section*{3. Objektorientierter Entwurf}

\subsection*{Sequenzdiagramme}
\begin{itemize}
    \item das Wesen einer Interaktion kennen
    \item wissen, wie Sequenzdiagramme im Entwicklungsprozess eingesetzt werden
    \item Sequenzdiagramme für Analyse- und Designentscheidungen sowie zur Darstellung von Kontrollflüssen einsetzen können
    \item die wichtigsten Modellierungskonzepte kennen
    \item sinnvolle Sequenzdiagramme erstellen und Java-Implementierungen ableiten können
    \item Sequenzdiagramme aus Java Quellcode erzeugen können
\end{itemize}

\subsection*{Zusammenfassung}

\begin{itemize}
    \item Sequenzdiagramme sind \textbf{Verhaltensdiagramme}
    und können im gesamten Entwicklungszyklus zur Modellierung von Interaktionen angewendet werden
    \item Interaktionen bestehen im Wesentlichen aus dem Nachrichtenaustausch zwischen verschiedenen Kommunikationspartnern
    \item in den meisten Fällen wird die Kommunikation von Objekten von Klassen modelliert, aber es können auch Teilnehmer auf anderen Ebenen modelliert werden
    \item gewöhnlich gibt ein Sequenzdiagramm einen \textbf{Anwendungsfall} (\textit{Szenario}) wieder
    \item sie dienen nicht dazu, um Zustandsänderungen darzustellen (hierfür werden Zustandsdiagramme verwendet)
    \item Sequenzdiagramme sind \textit{nicht} gut geeignet für die Darstellung von
    \begin{itemize}
        \item nebenläufigem Verhalten (\textit{asynchrone} Nachrichten, \textit{kombiniertes Fragment})
        \item Schleifen und alternativem Verhalten (\textit{kombinierte Fragmente})
        \item[] $\rightarrow$  für beide Fälle sind \textbf{Aktivitätsdiagramme} besser geeignet
    \end{itemize}
\end{itemize}

\section*{4. Klassendiagramme - Erweiterte Konzepte und Paketdiagramme}

\subsection*{Sequenzdiagramme}
\begin{itemize}
    \item erweiterte Modellierungskonzepte in Klassendiagrammen für den Detailentwurf von Systemen kennen
    \item die Konsequenzen dieser Konzepte in der Implementierung kennen
    \item wissen, wie Paketdiagramme im Entwicklungsprozess eingesetzt werden
    \item die Modellierungskonzepte in Paketdiagrammen kennen
    \item anhand von Paketdiagrammen Software-Architekturen gestalten und bewerten können
\end{itemize}

\subsection*{Zusammenfassung}

\begin{itemize}
    \item Die vorgestellten erweiterten Konzepte werden vor allem im \textbf{Feinentwurf} von Klassenstrukturen eingesetzt.
    \item Generell gilt, dass die Spezifikationen von Objekten und Properties nur so detailliert ausgeführt werden sollen, wie nötig: Sind Details für das Verständnis unnötig, sollte darauf verzichtet werden.
    \item \textbf{Generalisierungsbeziehungen} sollten grundsätzlich modelliert werden.
    \item Ist eine \textbf{Komposition} statt einer Generalisierung für den Sachverhalt möglich, sollte diese verwendet werden: ``Favor object composition over inheritance.`` (\cite[19 f.]{GHJV94})
    \item \textbf{Kompositionsbeziehungen} geben eindeutige Impulse für eine Implementierung, \textbf{Aggregationsbeziehungen} sind in ihrer semantischen Aussage nicht sehr eindeutig
    \item \textbf{Assoziationsklassen} sind in ihrer frühen Designphase empfehlenswert, die Realisierung (mit {bspw.} Java) erfordert aber eine Transformation in eine volle Klasse, wobei die Multiplizitäten angepasst werden müssen
    \item Elemente können in \textbf{Paketen} zusamengefasst und unter einem gemeinsamen Namensraum gruppiert werden
\end{itemize}

\section*{4. Anwendungsfalldiagramm (Use-Case-Diagramm)}

\subsection*{Sequenzdiagramme}
\begin{itemize}
    \item wissen, wie die Anwendungsfallanalyse im Rahmen der Anforderungsspezifikation eingesetzt wird
    \item Prozessschritte der Anwendungsfallanalyse beherrschen
    \item die wichtigsten Modellierungskonzepte in Anwendungsfalldiagrammen kennen
    \item den Aufbau einer Anwendungsfallbeschreibung kennen
\end{itemize}

\subsection*{Zusammenfassung}

\begin{itemize}
    \item \textbf{Anwendungsfalldiagramme} besitzen eine einfache Notation, um \textbf{Akteure} und \textbf{Anwendungsfälle} eines \textbf{Systems} grafisch darzustellen
    \item auf Basis eines Anwendungsfalldiagramms lässt sich ermitteln, ob die Anwendungsfälle gewünschtes Verhalten realisieren bzw. Beziehungen zu einem oder mehreren Akteuren haben
    \item Anwendungsfälle und Akteure \textit{müssen} zusätzlich textlich beschrieben werden
\end{itemize}

\section*{5. Aktivitätsdiagramme}

\subsection*{Sequenzdiagramme}
\begin{itemize}
    \item wissen, wie Aktivitätsdiagramme im Rahmen der Verhaltensanalyse in verschiedenen Prozessphasen eingesetzt werden können
    \item die wichtigsten Modellierungskonzepte in Aktivitätsdiagrammen kennen
    \item Aktivitätsdiagramme interpretieren, erstellen und implementieren können
\end{itemize}

\subsection*{Zusammenfassung}

\begin{itemize}
    \item \textbf{Aktivitätsdiagramme} gehören zu den \textbf{Verhaltensdiagrammen} und können im \textit{gesamten} \textbf{Entwicklungsprozess} eingesetzt werden
    \item durch ein Aktivitätsdiagramm kann gezeigt werden, wie \textbf{Verhalten} realisiert wird
    \item es können mehrere \textbf{Aktivitäten} dargestellt werden
    \item \textbf{Aktivitätskanten} und \textbf{Aktivitätsknoten} werden zur Beschreibung von Verhalten verwendet
\end{itemize}

\section*{5. Zustandsautomaten}

\subsection*{Sequenzdiagramme}
\begin{itemize}
    \item wissen, wie Zustandsdiagramme im Rahmen der Verhaltensanalyse einzelner Objekte eingesetzt werden können
    \item die wichtigsten Modellierungskonzepte in Zustandsdiagrammen kennen
    \item Zustandsdiagramme interpretieren, erstellen und implementieren können
\end{itemize}

\subsection*{Zusammenfassung}

\begin{itemize}
    \item Das \textbf{Zustandsdiagramm} gehört wie das \textbf{Aktivitätsdiagramm} zu den \textbf{Verhaltensdiagrammen}.
    \item Ein Zustandsdiagramm zeigt den \textbf{Lebensweg} eines \textbf{Objektes}.
    \item \textbf{Attributwerte} bestimmen den Zustand eines Objektes.
    \item \textbf{Transitionen} zwischen Zuständen werden durch \textbf{Guards} und \textbf{Events} gesteuert.
    \item Während der Transitionen können \textbf{Effekte} (\textit{Aktivitäten}) realisiert werden.
    \item Die Modellierung \textbf{zusammengesetzter Zustände} ist ebenfalls möglich.
    \item Für die Darstellung \textbf{paralleler} oder  \textbf{konkurrierender} Abläufe können \textbf{Regionen} modelliert werden.
\end{itemize}