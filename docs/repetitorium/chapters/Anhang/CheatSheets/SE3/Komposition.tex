\section{Komposition}

\begin{tcolorbox}[title=Komposition]
    Bei einer \textbf{Komposition} handelt es sich um eine \textit{Ganzes}-\textit{Teile}-Beziehung, bei der das \textit{Ganze} verantwortlich für die Erstellung und Beseitigung der \textit{Teile} ist.
    Außerdem sind die \textit{Teile} an die \textbf{Existenz} des \textit{Ganzen} gebunden.\\
    \textit{Teile} gehören immer zu \textit{genau einem} \textit{Ganzen}.\\
    Eine Komposition wird dargestellt durch eine gefüllte Raute an der Klassenbox, die das \textit{Ganze} repräsentiert.
\end{tcolorbox}