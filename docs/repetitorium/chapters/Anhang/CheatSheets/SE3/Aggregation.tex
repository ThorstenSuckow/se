\section{Aggregation}

\begin{tcolorbox}[title=Aggregation]
    Eine \textbf{Aggregation} ist in ihrer \semantischen Aussage unscharf (vgl.~\cite[40]{Buh09}).\\
    Sie beschreibt eine \textit{Ganzes}-\textit{Teile}-Beziehung, aber die \textit{Teile} können zu mehreren verschiedenen \textit{Ganzen} gehören.\\
    Außerdem kann ein Teil dem \textit{Ganzen} jederzeit wieder entnommen werden und das \textit{Ganze} ist nicht verantwortlich für das Erstellen der \textit{Teile}, weshalb Implementierungen beim Hinzufügen von Teilen oft schon fertige Objekte erwarten, die jeweils eines der \textit{Teile} repräsentieren.\\
    Aggregationsbeziehungen werden durch eine nicht-gefüllte Raute an der Seite der Klasse, die das Ganze repräsentiert, dargestellt.
\end{tcolorbox}