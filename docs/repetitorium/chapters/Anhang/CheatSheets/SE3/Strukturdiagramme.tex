\section{Strukturdiagramme}

\begin{tcolorbox}[title=Strukturdiagramme]
    In den \textbf{Strukturdiagrammen} werden \textbf{statische Strukturaspekte} eines Systems betrachtet.
    Für eine Systemmodellierung können alle dieser 6 Diagramme verwendet werden.

    \begin{itemize}
        \item \textbf{Klassendiagramm}: Das \textbf{Klassendiagramm} wird verwendet, um ein objektorientiertes System zu beschreiben.
        \item \textbf{Paketdiagramm}: Ein \textbf{Paketdiagramm} zeigt die Pakete eines Systems und deren Beziehungen.
        \item \textbf{Komponentendiagramm}: Stellt Komponenten oder auch \textit{Subsysteme} mit definierten Interfaces dar, um bspw. \textbf{Architekturdesign} darzustellen.
        \item \textbf{Verteilungsdiagramm}: Beschreibt eine Menge von Knoten, die die \textit{Ausführungsarchitektur} eines Systems definieren können, wobei Knoten i.d.R. \textit{Geräte} oder \textit{Softwareablaufumgebungen repräsentieren}.
        \item \textbf{Kompositionsstrukturdiagramm}: Stellt die \textbf{Komposition} von \textit{Systemstrukturen} (Klassen, Komponenten, Gesamtsystem) in einem bestimmten \textit{Kontext} mit einem bestimmten \textit{Ziel} dar (vgl. \cite[9]{Buh09}).
        \item \textbf{Objektdiagramm}: Momentaufnahme eines Systems zu genau einem \textit{Zeitpukt} während der Ausführung
    \end{itemize}

\end{tcolorbox}