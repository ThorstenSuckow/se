\section{Verhaltensdiagramme}

\begin{tcolorbox}[title=Verhaltensdiagramme]
    \textbf{Verhaltensdiagramme} stellen \textit{Dynamik}, \textit{interne Abläufe} und das \textit{Zusammenspiel} der Systemteile dar, um eine Spezifikation zu vervollständigen.

    \begin{itemize}
        \item \textbf{Anwendungsfalldiagramm}: Dienen zur \textit{Spezifizierung} und \textit{Formalisierung} von \textit{Systemanforderungen}, unter Berücksichtigung von \textbf{Akteuren}, \textbf{Systemen} und \textbf{Anwendungsfällen}
        \item \textbf{Zustandsdiagramm}: Auch \textbf{Zustandsautomaten}; zeigen für ein einzelnes Objekt Zustandsänderungen während der Lebenszeit
        \item \textbf{Aktivitätsdiagramme}: Werden zur Modellierung von \textit{Kontroll}- oder \textit{Objektflüssen} oder zur Darstellung von \textit{Programmlogik} genutzt.
        \item \textbf{Interaktionsdiagramme}: Stellen das Zusammenspiel mehrerer Kommunikationspartner dar.
        Es gibt unterschiedliche Typen von Interaktionsdiagrammen, die Interaktionen auf verschiedenen \textit{Abstraktionsebenen} modellieren:
        \begin{itemize}
            \item \textbf{Sequenzdiagramm}: Zeigt den Verlauf einer Interaktion in zwei Dimensionen, \textit{Kommunikationspartner} und die Nachrichten in ihrer \textit{zeitlichen Abfolge}
            \item \textbf{Kommunikationsdiagramm}: hebt die Kommunikationsbeziehungen zwischen den Partnern hervor
            \item \textbf{Timing Diagramm}: hebt die zeitlichen Aspekte einer Interaktion hervor
            \item \textbf{Interaktionsüberblickdiagramm}: Spezialfall des \textbf{Aktivitätsdiagramms} - statt Aktionen und Aktivitäten können das \textbf{Sequenz}-, \textbf{Kommunikations}- und das \textbf{Timing}-Diagramm als Knoten verwendet werden.
        \end{itemize}
        Alle verwenden dieselben Grundelemente: \textit{Lebenslinien} der Akteure und Nachrichten, die zwischen diesen ausgetauscht werden.
    \end{itemize}
\end{tcolorbox}
