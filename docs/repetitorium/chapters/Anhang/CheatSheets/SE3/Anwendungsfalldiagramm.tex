\section{Anwendungsfalldiagramm}

\begin{tcolorbox}[title=Anwendungsfalldiagramm]
    Die \textbf{Anwendungsfallanalyse} wird im Rahmen der \textbf{Anforderungsspezifikation} eingesetzt.\\

    \noindent
    Anwendungsfalldiagramme sind das am häufigsten eingesetzte Mittel zur Aufnahme und Darstellung von \textbf{Anforderungen}.\\
    Sie beschreiben selbst kein Verhalten und keine Abläufe, sondern zeigen nur die Zusammenhänge der an Anwendungsfällen beteiligten Modellelemente und sind somit ein Hilfsmittel zur Anforderungsermittlung und Verwaltung.\\

    \noindent
    Zu den wichtigsten Schritten einer Anwendungsfallanalyse gehören:
    \begin{enumerate}
        \item Akteure identifizieren
        \item Anwendungsfälle identifizieren
        \item Beschreiben der Akteure und Anwendungsfälle
        \item Identifizieren von \textit{Schlüsselobjekten}, die das System verwaltet
        \item Identifizieren der wichtigsten Anwendungsfälle (Priorisierung)
        \item detailliertere Beschreibung der Anwendungsfälle
        \item Strukturierung des Anwendungsfalldiagramms
    \end{enumerate}

    \noindent
    Anwendungsfälle beschreiben das \textbf{Szenario} der Nutzung, und nicht Features des Systems.
\end{tcolorbox}
