\section{Nicht-funktionale Anforderungen}

\noindent
\textbf{Nicht-funktionale Anforderungen} werden in \textbf{Qualitätsanforderungen} und \textbf{Randbedingungen} unterschieden:

\begin{itemize}
    \item \textbf{Qualitätsanforderungen} beschreiben die Qualität oder Eignung eines Systems (Performance, Benutzerfreundlichkeit, \ldots)
    \item \textbf{Randbedingungen} beschreiben technische Anforderungen (Programmiersprache, Framework, \ldots) oder organisatorische Anforderungen (Budget, Deadlines, \ldots)
\end{itemize}

\noindent
Auch nicht-funktionale Anforderungen verursachen Aufwand, da sie in allen Phasen mitberücksichtigt werden müssen.\\
Es ist deshalb nötig, alle nicht-funktionalen Anforderungen zu priorisieren, und sich gemeinsam mit Kunde und Anwendern auf die notwendigsten zu einigen.\\
Gute Praxis ist es, die Qualitätsmerkmale genau zu definieren.

\subsection*{Nicht-funktionale Anforderungen sind entscheidend}
Nicht-funktionale Anforderungen sind insb. entscheidend für die Systemarchitektur.\\
Es muss zumindest \textit{eine} Systemarchitektur, die die fundamentalen Anforderungen (Erreichbarkeit, Antwortzeiten, Funktionalität) abdeckt.

\subsection*{Systematik der Qualitätsanforderungen}
Vorgefertigte Kategorien und Typen von Qualitätsmerkmalen können bei der Erfassung von Qualitätsanforderungen helden, die Übersicht zu bewahren und keine wichtige Anforderung zu übersehen\footnote{man nennt dies \textit{systematische Aufstellungen} bzw. \textit{Qualitätssysteme}}.\\

\noindent
Die im Kurs verwendete Systematik folgt dem Standard \textbf{IEEE 1061-1998}:

\blockquote[{\url{https://standards.ieee.org/ieee/1061/1549/}\footnote{abgerufen 01.04.2024}}]{
    This IEEE Standards product is part of the family on Software Engineering. A methodology for establishing quality requirements and identifying, implementing, analyzing, and validating the process and product software quality metrics is defined. The methodology spans the entire software life cycle.
}

\noindent
Es existieren u.a. folgende Kategoerien:

\begin{itemize}
    \item \textbf{Verfügbarkeit (Availability)}: geplante Betriebszeit, während der das System vollständig nutzbar ist (formal beschrieben durch Verhältnis tatsächlicher Uptime / geplanter Uptime).
    Als Ausfallzeit werden i.d.R. nur ungeplante Ausfälle gezählt, aber nicht die Wartungsfenster.
    \item \textbf{Effizienz (Efficiency)}: Maß, wie gut ein System vorhandene Systemresourcen (Speicher, Prozessorleistung) nutzt.
    Nötig, wenn mehrere Programme auf dieselben Systemressourcen zugreifen muss, oder wenn Erweiterungen geplant sind
    \item \textbf{Erweiterbarkeit (Flexibility / Extensibility)}: beschreibt, wie leicht neue Funktionalität in das System eingebaut werden kann, um es zu erweitert.
    Bspw. können \textbf{Functional Points} verwendet werden, um die Komplexität eines Dialogs im UI zu messen\footnote{jedes \textit{control} erhält hier einen Wert, die Summe ergibt die ``Komplexität``}
    \item \textbf{Wartbarkeit}: Änderbarkeit des Systems aus technischer Sicht.
    Hiermit ist vor allem die\textit{Änderung existierender Funktionalität} gemeint, im Gegensatz zu \textbf{Erweiterbarkeit}, die die Erweiterung des Funktionsumfangs beschreibt
    \item \textbf{Integrität (Integrity)}: Sicherheit des Systems, bspw. gegen unberechtigte Zugriffe
    \item \textbf{Interoperabilität (Interoperability)}: wie leicht das System mit anderen Systemen kommunizieren oder Daten austauschen kann (bspw. Export in andere Formate, Schnittstellen zu externen Systemen, \ldots)
    \item \textbf{Zuverlässigkeit (Reliability)}: Durchschnittliche Zeit oder durchschnittliche Anzahl an Operationen, während der ein System fehlerfrei arbeitet
    \item \textbf{Robustheit (Robustness)}: Das Verhalten des Systems unter fehlerhaften (Rand-)Bedingungen (Falscheingaben, Hardwarefehler, \ldots)
    \item \textbf{Benutzbarkeit (Usability)}: Bedienbarkeit des Programms, bspw. Ergonomie, Barrierefreiheit, wie lange bestimmte Prozesschritte benötigen, bis sie von einem erfahrenen / unerfahrenen Anwender durchgeführt wurden, \ldots
    \item \textbf{Portabilität (Portability)}: Aufwand, eine Software auf ein anderes OS eine andere Hardware zu portieren (bspw. \textit{mobile})
    \item \textbf{Wiederverwendbarkeit (Reusability)}: Anforderung an Wiederverwendbarkeit einzelner Komponenten / Subsysteme in anderen Systemen; bedeutet i.d.R. zusätzlichen Aufwand durch Absprachen der involvierten Teams
    \item \textbf{Testbarkeit (Testability)}: Software muss so konstruiert sein, dass Testverfahren durchgeführt werden können (bspw. in der Industrie, wenn Testverfahren vorgeschrieben sind). \textbf{Testbarkeit} entspricht zumeist auch \textbf{Wartbarkeit}, da eine Software gut wartbar ist, wenn sie testbar ist, und umgekehrt gut testbar, wenn sie wartbar ist
    \item \textbf{Performance}: Antwortzeit bei Anfragen an das System.
    Faustregel: Systeme mit direkter Nutzeraktion sollten 1 sek als Antwortzeit nicht überschreiten; in manchen Umgebungen sind Antwortzeiten nicht garantierbar (Internet), im Embedded-Bereich müssen sie garantiert werden können (vgl.~\cite[62]{Wed09}).
    Vorhersehbare Kriterien wie Im-/Export großer Datenmengen über das Internet müssen berücksichtigt werden, da sich solche Gegebenheiten auf den Entwurf/die Architektur auswirken können (bspw. wegen Komprimierungsverfahren).
\end{itemize}


\noindent
Eine genaue Auswahl, Definition und Priorisierung hilft allen Stakeholdern, sich auf gemeinsame und realistische Ziele zu einigen.\\
Dabei sollte darauf geachtet werden, auch die Anforderungen mit zu erfassen, die nicht Teil eines Standards sind, weil alle geäußerten Anforderungen erheblichen Einfluss auf das Produkt haben können.


\subsection*{Technische Randbedingung}
\textbf{Randbedingungen} sind eine weitere wichtige Kategorie von nicht-funktionalen Anforderungen.\\

\vspace{5mm}
\begin{tcolorbox}
    \blockquote[{\cite[65, Hervorhebung eigene]{Wed09}}]{
        Bei \textbf{Randbedingungen} handelt es sich um Vorgaben vom Kunden, nicht um die Vorstellung der Softwareentwickler.
    }
\end{tcolorbox}
\vspace{5mm}

\noindent
Unterschieden wird hierbei zwischen \textbf{technischen} und \textbf{organisatorischen} Randbedingungen.\\

\noindent
Randbedingungen sind nicht priorisierbar, aber es kann angegeben werden, ob diese \textit{fest} oder \textit{anpassbar} sind.

\begin{itemize}
    \item \textbf{Technische Randbedingungen} definieren die einzusetzenden technischen Komponenten, wie bspw.
        \begin{itemize}
            \item Hardware-Infrastruktur
            \item Software-Infrastruktur
            \item Programmiersprachen
            \item Frameworks
        \end{itemize}
    \item \textbf{Organisatorische Randbedingungen} sind Vorgaben, die die Organisation des Projektes betreffen, wie bspw.
    \begin{itemize}
        \item Termine
        \item Dokumentationsvorgaben
        \item Vorgaben zur Vorgehensweise
        \item Vorgaben zu Tests oder Abnahmeprozessen
    \end{itemize}
\end{itemize}

\noindent
Organisatorische Randbedingungen können im einzelnen auf unterschiedliche Aspekte des Projektes Auswirkungen haben (QS, Implementierung, Projektorganisation).\\

\noindent
Es kann mitunter herausfordernd sein, zwischen eigenen Ideen und den Kundenanforderungen zu unterscheiden, was aber wichtig ist, da ``Randbedingungen Einschränkungen für mögliche Lösungen darstellen und damit entscheidende Auswirkungen auf die gesamte Tätigkeit haben`` (\cite[65]{Wed09}).