\section{Funktionale Anforderungen}

\vspace{5mm}
\begin{tcolorbox}
    \textbf{Funktionale Anforderungen} definieren, welche Funktionen eines zu entwickelnden Systems von Endanwendern oder anderen Systemen benutzt werden können (\cite[66]{Wed09}).
\end{tcolorbox}
\vspace{5mm}

\noindent
Ein weit verbreiteter Ansatz zur Erarbeitung von Anforderungen ist die Erfassung von \textbf{Anwendungsfällen} (\textit{Use Case}), oder auch von \textbf{User Storys}.\\

\noindent
Die Verfahren können gleichzeitig eingesetzt werden, da sind unterschiedlich detailliert sind und andere Perspektiven beleuchten.

\subsection{User Story}

\textbf{User Storys} beschreiben in ein bis zwei Sätzen, wie ein Endanwender mit dem System arbeiten möchte, bspw. ``Der Vertrieb bezieht aus dem System die Wiedervorlageliste und protokolliert die durchgeführten Gespräche``.\\

\noindent
Eine \textbf{User Story} ist aufgrund der fehlenden Details als eine Art \textbf{Merkhilfe} für Stakeholder und Entwickler zu verstehen, und nicht als formale Beschreibung eines Prozesses; idealerweise werden die Storys von den Stakeholdern erstellt.\\

\noindent
User Stories sind dazu geeignet, ``die Welt des Kunden zu verstehen``\footnote{
\textit{Wedemann} verweist hier auf \textit{Beck}, s.~\cite[66]{Wed09}
}.\\
Es kann durchaus passieren, dass sich User Storys widersprechen, wenn mehrere unterschiedliche Stakeholder die Storys erstellen.\\

\noindent
Auf eine \textit{exakte und detaillierte} Beschreibung der einzelnen Storys wird \textit{verzichtet}, da davon ausgegangen wird, dass die Stellvertreter der Endanwender jederzeit erreichbar sind.

\noindent
\textbf{Story Cards} (Karteikarten) werden verwendet, um die Storys zu notieren.\\
I.d.R. werden sie für Planungszwecke mit einer Nummer, sowie Kennungen für Priorität und Aufwand versehen.


\begin{itemize}
    \item \textbf{Vorteile}
        \begin{itemize}
            \item hilfreich für die direkte Arbeit mit den Endanwendern: Beschreibung, was mit dem System gemacht werden soll, hilft dem Endanwender, über seine Wünsche zu reflektieren, und dem Entwickler, die Bedürfnisse des Kunden zu verstehen
        \end{itemize}
    \item \textbf{Nachteile}
        \begin{itemize}
            \item schwierig, wenn Stakeholder nicht (dauernd) anwesend oder nicht erreichbar sind, da die knappe schriftl. Form zu vage ist, womit auch eine Dokumentation fehlt
            \item Formalisierung (wie in Anwendungsfällen) würde es den Stakeholdern und Endanwendern möglich machen, nichts zu vergessen und bei komplizierten Schaverhalten den Überblick zu behalten
        \end{itemize}
\end{itemize}

\subsection{Anwendungsfall}
\blockquote[{\cite[67, Hervorhebung eigene]{Wed09}}]{Ein \textbf{Anwendungsfall} beschreibt, wie Nutzer mit einem System in einer abgeschlossenen, ununterbrochenen Folge arbeiten, um ein fachliches Ziel zu verwirklichen.}

\noindent
Die Abläufe sollten für die Beschreibung als Anwendungsfälle nicht zu komplex sein.\\

\noindent
Man unterscheidet i.d.R. zwischen zwei Typen von Anwendungsfällen: \textbf{Grundlegende Anwendungsfälle} (\textit{Essential Use Case}) und \textbf{traditionelle Anwendungsfälle} (\textit{System Use Case}).

\subsubsection*{Grundlegender Anwendungsfall (Geschäftsvorfall)}
Ein \textbf{Grundlegender Anwendungsfall} beschreibt \textit{unter vollständigem Verzicht} auf Details der Technik oder der späteren Umsetzung, wie der Endnutzer mit dem System umgeht und wie das System darauf reagieren soll (vgl.~\cite[68]{Wed09}).\\

\noindent
Der Anwender ist hierbei in der Regel der \textbf{Akteur}, externe Systeme oder Hardware können als auch als \textit{Akteure} gesehen werden.\\
Das System selber ist nie ein Akteur.

\subsubsection*{Traditioneller Anwendungsfall (Systemanwendungsfall)}
\textbf{Traditionelle Anwenundgsfälle} können deutlich detaillierter als grundlegende Anwendungsfälle sein und \textit{Details der Implementierung berücksichtigen}.\\
Sie unterscheiden sich außerdem im Format zu den grundlegenden Anwendungsfällen und können unterschiedlich detailliert ausfallen.\\

\noindent
Ansatz zur Erstellung:

\begin{enumerate}
    \item Anwendungsfalls \texctit{high-level} erstellen (bloße Überschrift)
    \item Anwendungsfall \textit{informell} erstellen (grundlegender Ablauf)
    \item Anwendungsfall \textit{formell} ausprägen
\end{enumerate}


\subsubsection*{Probleme mit formellen Anwendungsfällen}
Für die detailierte bzw. formelle Ausprägung von Anwendungsfällen eignen sich eher Werkzeuge und Hilfsmittel, die während der Analsephase verwendet werden, wie \textbf{Aktivitätsdiagramme} der Anwendungsfälle, \textbf{Entwürfe} und \textbf{Ablaufdiagramme} der Benutzeroberfläche, oder das \textbf{Domänenmodell}.\\
Mit diesen Werkzeugen kann dann auch das Zusammenspiel mehrere Anwendungsfälle modelliert werden, wie \textit{Wedemann} feststellt (vgl.~\cite[71]{Wed09}).\\

\noindent
Ingesamt entstehen viele Dokumente.\\
Je umfangreicher das Projekt, desto häufiger müssen diese Dokumente geändert werden, was zusätzlichen Aufwand bedeutet.\\
Man sollte deshalb abwägen, welche Dokumente für den Einmalgebrauch bestimmt sind und welche auch für die zukünftige Dokumentation und Planung geeignet sind.

\subsection{UML Use Case-Diagramm}
Bei größeren Systemen, in denen Stakeholder und Entwickler gemeinsam viele Anwendungsfäll entwickeln, werden \textbf{Use Case-Diagramme} verwendet, die durch die \textit{Unifield Modeling Language} \textbf{UML} standardisiert sind.\\