\section{Regeln}
\textbf{Geschäftsregeln} (\textit{Business Rules}), die in der Software implementiert werden müssen, verstehen sich als Repräsentanten fundamentaler Regeln von (Geschäfts-)Prozessen.\\
Die Einhaltung dieser Regeln erhöht die Zuverlässigkeit der Anwendung und die Qualität der Daten.\\

\noindent
Geschäftsregeln können in folgende Typen eingeteilt werden:

\begin{itemize}
    \item \textbf{Randbedingungen (Constraint)} beschränken, was Nutzer tun dürfen (Stichwörter \textit{muss}, \textit{darf nicht}, \textit{nur})
    \item \textbf{Aktionsauslöser (Action Enabler)} beschrieben, was nach Auslösung eines Ereignisses passiert (\textit{wenn \ldots, dann} [Aktion])
    \item \textbf{Wissenserzeuger (Inference)} erzeugen neues Wissen, wenn eine Bedingung zutrifft (\textit{wenn \ldots, dann} [neues Wissen])
    \item \textbf{Berechnung (Computation)}: Vorschriften zur Berechnung mit Formeln oder Algorithmen
\end{itemize}

\noindent
Regeln sollten in einer Art \textit{Regelbuch} festgehalten werden.\\
Stakeholder und Entwickler haben dadurch die gleiche Sicht auf alle Regeln.\\
Folgende Spalten sind sinnvoll:

\begin{itemize}
    \item Nummer der Regel
    \item Beschreibung der Regel
    \item Typ der Regel
    \item Dynamik der Regel\footnote{\textit{statisch} oder \textit{dynamisch}, also ändert sich die Regel, oder bleibt sie unverändert?}
    \item Quelle\footnote{aus welchem Geschäftsbereich stammt die Regel?}
\end{itemize}