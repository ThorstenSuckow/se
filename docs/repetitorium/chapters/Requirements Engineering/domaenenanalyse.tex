\section{Domänenanalyse}

\noindent
Bei der \textbf{Domänenanalyse} eignen sich Entwickler Wissen über die umzusetzende Fachlichkeit an.\\

\noindent
Wissen über de Domände verpricht ein effektiveres und effizienteres Zusammenarbeiten mit Kunden und Anwendern, da auf einer gemeinsamen Basis kommuniziert werden kann, in der elementare Begriffe und Konzepte der Domände von allen Beteiligten gleich verstanden werden.\\
Dies erleichtert das Vorahnen von Anforderungen bzw. Anpassungen und das Treffen richtiger Entscheidschungen.

\noindent
Domänenwissen veraltet deutlich langsamer als technisches Wissen, weshalb Experten einer speziellen Domäne wertvolel Mitarbeiter sind.\\

\noindent
Domänenwissen kann aus Fachartikeln bzw. Fachbüchern gewonnen werden bzw. in Schulungen oder durch Gespräche mit fachkundigen Mitarbeitern des Kunden bzw. Kollegen.\\
Vevor das gespräch mit Fachexperten genutzt wird, um Detailfragen zu klären, sollte sich der Entwickler zunächst selber einen Überblick über die Fachlichkeit verschaffen, um den Kontext fachbezogener Fragen selber besser einschätzen zu können.\\

\noindent
Nach der Domänenanalyse ist ein Entwickler besser auf seine Aufgabe vorbereitet, und der Informationsaustausch gelingt besser.
Außerdem ist er in der Lage, neuen Mitarbeiter den Einstieg zu erleichtern, wozu solche Informationen auch schriftlich niedergelegt werden sollten.\\
\textit{Wedemann} nennt hierzu folgende Dokumentenarten (vgl.~\cite[42]{Wed09}):

\begin{itemize}
    \item Glossar der wichtigsten Fachbegriffe und deren Bedeutungen
    \item Literaturlisten mit kurzer Bewertung und Inhaltsangabe
    \item Zusammenfassungen von Texten bzw. Auszüge von Texten in Bezug auf das Projekt und das Produkt
    \item Erstellen von Kunden- und Anwenderstrukturen und -organisationen
    \item Auflistung konkurrierender Software
    \item Aufstellung derzeitiger, bereits bekannter Geschäftsprozesse
\end{itemize}