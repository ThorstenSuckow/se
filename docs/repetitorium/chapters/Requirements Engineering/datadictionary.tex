\section{Datadictionary und Mengengerüst}
In einem \textbf{Datadictionary} wird festgehalten, in welchem Format welche Daten verarbeitet werden.\\

\noindent
Über solch ein zentrales Dokument kann verhindert werden, dass unterschiedliche Anwendungsteile, die von unterschiedlichen Teams entwickelt werden, die Daten auch in unterschiedlichen Formaten verwalten.\\

\noindent
Zur Datendefinition gehören:

\begin{itemize}
    \item Typ (\textit{String}, \textit{Integer}, \ldots)
    \item Format (bspw. kalendarische Datumsformate)
    \item ggf. die Einheit
    \item andere \textit{Daten} bei zusammengesetzten Datentypen (bspw. bei Adressen)
\end{itemize}

\noindent
Zu beachten ist, dass es sich bei dem Datadictionary um ein \textbf{vorläufiges Dokument} handelt: Es dient dazu, Informationen, die während der Anforderungsphase gesammelt wurden, systematisch zu sammeln.\\
$\rightarrow$ die endgültige Datendefinition erfolgt erst in der Analysephase, weil dann erst genügend Informationen zusammengetragen wurden.

\vspace{5mm}
\begin{tcolorbox}
    Das Datadictionary ist als strukturierte Merkhilfe zu verstehen.
\end{tcolorbox}
\vspace{5mm}

\subsection{Mengengerüst}
Das \textbf{Mengengerüst} beschreibt die zu erwartende Anzahl der Daten\footnote{Kundendatensätze, Adressdatensätze, \ldots} als wichtiges Kriterium, da sich damit festgehaltenen Kennzahlen direkt auf die Architektur bzw. Datenhaltung (Datenbanksysteme) auswirken kann.\\

\noindent
\textit{Wedemann} merkt an, dass es sich hierbei streng genommen um eine \textit{nicht-funktionale} Anforderung handelt, aber nicht-funktionale Anforderungen meist globaler gefasst sind als ein detailliertes Mengengerüst (\cite[76]{Wed09}).