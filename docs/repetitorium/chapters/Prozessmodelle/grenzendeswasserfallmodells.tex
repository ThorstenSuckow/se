\section{Grenzen des Wasserfallmodells}\label{sec:grenzen-des-wasserfallmodells}

\noindent
Die Beliebtheit des Wasserfallmodells gründet u.a. auf:

\begin{itemize}
    \item logischer Aufbau der Abfolge der Aktivitäten (Theorie $\rightarrow$ Praxis)
    \item Ergebnisse einer einzelnen Phase ist Grundlage für die darauffolgende Phase (hohe Effektivität bei fehlerfreien Ergebnissen)
    \item einfache Verwaltung des Projektes durch das Management; Abschluss einer Phase durch Fertigstellung der Dokumente leicht als \textit{Meilensteine} definierbar
    \item Kunden wissen, welches Produkt sie erhalten; nach der Anforderungsphase sind die Anforderungen klar, weitere Kommunikation nach der Analyse ist nicht mehr notwendig\footnote{
    der Kunde wird in wenige weitere Phasen miteinbezogen, grundsätzlich läuft die Entwicklung dann aber ohne Kundenfeedback, im Gegensatz zu agilen Prozessen.
    }
\end{itemize}

\noindent
Oftmals sprechen (Projekt-)umstände aber gegen den Einsatz des Wasserfallmodells:

\begin{itemize}
    \item Anforderungen lassen sich zu Beginn nicht \textit{eindeutig} bestimmen: Kunden besitzen eine unklare Vorstellung von dem, was sie benötigen/wollen, was oft erst in Phasen erkannt wird, in dem der Kunde das erste mal mit dem System in Berührung kommt\footnote{
    was bei dem Wasserfallmodell halt recht spät ist
    }.
    Trotzdem soll wegen Zeitdruck bereits früh mit der Entwicklung begonnen werden.
    \item Kunden fällt die Abstraktion ihrer Anforderungen schwer.
    Oft lassen sich Anforderungen leichter anhand bereits lauffähiger Software klären, anstatt über Dokumente und Konzepte (\textit{Lasten-/Pflichtenheft})
    \item Kunden erhalten erst durch das lauffähige System Feedback darüber, wie ihre Anforderungen tatsächlich verstanden wurden, und zwar zu einem Zeitpunkt, an dem bereits ein wesentlicher Aufwand in Entwurf, Implementierung und Tests investiert wurde.
    \item Bei der Entwicklung der Architektur stellen sich viele Fragen über die Anforderungen mit dem Ergebnis, dass sich die Anforderungen ändern.
    Oft wird auch klar, dass die Anforderungen (technisch) nicht umsetzbar sind\footnote{
     das gleiche gilt für den Entwurf, wenn also bspw. das Fachkonzept in ein DV-Konzept gegossen werden soll
    }.
    $\rightarrow$ im Wasserfallmodell ist nicht geregelt, wie solche Erkenntnisse Einfluss auf die Anforderungen nehmen können
    \item Aufwand und Folgen durch den Einsatz neuer Technologien, die bspw. im Entwurf festgelegt werden, können nur schwer eingeschätzt werden, wenn den Entwicklern die Erfahrung damit fehlt.
    \item Bei dem Einsatz neuer Technologien stellt sich manchmal erst bei der Realisierung heraus, dass bestimmte Architekturen / Vorgehensweisen nicht möglich sind, und in früheren Phasen (Entwurf) andere Entscheidungen hätten getroffen werden sollen.
    Auch hierfür gibt es im Wasserfallmodell keinen Weg, der diese Erkenntnisse nachträglich in die ggw. Prozessphase (bspw. Entwicklung) einfliessen lassen kann.
\end{itemize}

\noindent
\textit{Alpar et al.} merken an, dass aus der umfangreichen Projektdokumentation eine Anzahl von Problemen resultieren kann, und stellen diesbzgl. ähnliche Probleme fest:

\blockquote[{\cite[323]{AABG14n}}]{
    \begin{itemize}
        \item Die Anforderungen müssen komplett vor Beginn der Entwicklung vorhanden
        sein.
        \item Der Auftraggeber hat Probleme, sämtliche Anforderungen konkret zu formulieren.
        \item Einige Wünsche der Anwender ergeben sich erst nach der ersten Nutzung.
        \item Von der Spezifikation der Anforderungen bis zum ersten Release vergehen oft
        Monate oder Jahre.
    \end{itemize}
}

\noindent
\textit{Petersen, Wohlin und Baca} untersuchen das Wasserfallmodell und seine Probleme in großen Projekten in~\cite{PWB09}.
Sie kommen zu folgendem Schluss:

\blockquote[{\cite[15]{PWB09}}]{
    The results are that the most critical issues in waterfall development
    are related to requirements and verification. In consequence, the waterfall model
    is not suitable to be used in large-scale development.
}