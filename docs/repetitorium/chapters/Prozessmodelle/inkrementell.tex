
\subsection{Inkrementell}
Aufgaben werden in fachlich abgeschlossene Teile eingeteilt.\\
Jedes Teilsystem (\textit{Inkrement}) wird so gewählt, dass es lauffähig und funktionstüchtig ist: Die Entwicklung der einzelnen Inkremente wiederum findet nach dem Phasenmodell statt,
wobei die Überführung in die letzte Phase \textit{Betrieb} dann wieder eine einzelne Phase ist, in der alle Teilsysteme zusammengeführt werden.\\
Tatsächlich findet die Auslieferung der einzelnen Inkremente an den Kunden aber bereits nach Fertigstellung dieser statt (vgl. Abbildung~\ref{fig:inkrementell}).\\

\begin{figure}
    \centering
    \includegraphics[scale=0.4]{chapters/Prozessmodelle/img/inkrementell}
    \caption{Inkrementelles Phasenmodell. (Quelle: in Anlehnung an \cite[322]{AABG14n})}
    \label{fig:inkrementell}
\end{figure}

\noindent
In Deutschland ist die bekannteste inkrementelle Vorgehensweise das \textbf{V-Modell XT}.\\
Das inkrementelle Vorgehen wird oft mit anderen Konzepten kombiniert.

\subsubsection*{Vorteile}

\begin{itemize}
    \item Kunde kann einzelne Teilsysteme abnehmen, ohne auf die Auslieferung des Gesamtsystems warten zu müssen.
    \item Durch Anwendung der Teilsysteme durch den Kunden kann dieser die Entwicklung der nächsten Inkremente beeinflussen.
    \item Entwickler können ihr Wissen aus den vorangegangenen Inkremente in die nächsten einfliessen lassen.
\end{itemize}

\subsubsection*{Nachteile}

\begin{itemize}
    \item Dauer der Entwicklung mit Inkrementen ist recht lange, da ein einzelnes Inkrement ein lieferfähiges Produkt darstellt, dass dieselben Phasen durchläuft wie ein einzelnes System, das über das Phasenmodell entwickelt wird
    \item Anforderungen des Kunden fließen nach der Anforderungsphase erst wieder in das nächste Inkrement ein
    \item Auch Entwickler, die während der Realisierung Schwächen im Entwurf entdecken, können Korrekturen erst im nächsten Inkrement einbringen.
\end{itemize}