\subsection{Agil}

\noindent
Unter \textbf{agilem Vorgehen} werden Vorgehensweisen zusammengefaßt, die dem \textbf{Agilen Manifest}\footnote{
    s. \url{https://agilemanifesto.org}, abgerufen 27.03.2024
} folgen:

\begin{tcolorbox}[title=Manifesto for Agile Software Development]
    We are uncovering better ways of developing software by doing it and helping others do it.\\
    Through this work we have come to value:
    \begin{itemize}
        \item Not only working software, but also well-crafted software.
        \item Not only responding to change, but also steadily adding value.
        \item Not only individuals and interactions, but also a community of professionals.
        \item Not only customer collaboration, but also productive partnerships.
        \end{itemize}
\end{tcolorbox}

\noindent
Die agilen Modelle befolgen ein \textbf{iteratives} Vorgehen, wobei eine Iteration i.d.R. $2$-$4$ Wochen dauert.\\
Statt dem Wasserfallmodell kommt Nebenläufigkeit zum Einsatz.\\
Steuerungsmodelle wie \textbf{Scrum} oder \textbf{Extreme Programming} sorgen dafür, dass die Abläufe gemanagt werden können.

\subsubsection*{Vorteile}
Agile Modelle sind gegenüber Änderungen der Anforderungen sehr flexibel und können die Stärken einzelner Mitarbeiter nutzen.