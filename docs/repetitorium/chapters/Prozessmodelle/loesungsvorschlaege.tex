\section{Lösungsvorschläge}

\subsection{Aufgabe 3.1}

Bei dem inkrementellen Vorgehen wird das Projekt in Inkremente aufgeteilt, deren Realisierung abgeschlossene Teilsysteme entsprechend.\\
Jedes Teilsystem repräsentiert einen abnahmefähigen, für den Wirkbetrieb bereiten Teil, der an den Kunden übergeben wird.\\
Erfahrungen und Ergebnisse solch eines Inkrements fließt in die Entwicklung der nachfolgenden Inkremente, die sequentiell umgesetzt werden.\\
Die Dauer eines Inkrements kann bis zu 9 Monate dauern (vgl.~\cite[84]{Wed09}).\\

\noindent
Beim iterativen Vorgehen wird das Projekt in Zeitabschnitte von (i.d.R.) 2-6 Wochen unterteilt.\\
Hierbei wird zunächst eine Zielbestimmung und dann eine Risikoanalyse durchgeführt, worauf die Ausführung folgt.\\
Im Anschluss daran wird dann die nächste Iteration geplant und die abgeschlossene bewertet.\\

\noindent
Im Unterschied zum inkrementellen Vorgehen erlauben Iterationen u.a. Prototypen, also nicht abnahmefähige Code-Teile. \\
Wird innerhalb einer Iteration nach einem \textbf{sequentiellen Modell} wie dem \textbf{Wasserfallmodell} gearbeitet, können dessen Phasen auch übersprungen werden (bspw. weil der Entwurf und die Analyse zu Beginn durchgeführt wurde, bzw. Tests \& Integration am Ende des Projektes in Iterationen durchgeführt werden).\\

\noindent
Des Weiteren können Ergebnisse einer Iteration in einer darauffolgenden modifiziert werden, wobei beim inkrementellen Vorgehen am Ende eines Inkrements ein lauffähiges, zunächst nicht änderbares Teilsystem steht.

\subsection{Aufgabe 3.2}

Beim agilen Vorgehen hält man sich an die Grundsätze des agilen Manifests:

\begin{itemize}
    \item Individuen und Interaktionen sind wichtiger als Prozesse und Werkzeuge
    \item Funktionale Software ist wichtiger als vollständige Dokumentation
    \item Zusammenarbeit mit den Kunden ist wichtiger als Vertragsverhandlungen
    \item Reaktionen auf Veränderungen ist wichtiger als die Verfolgung eines Plans
\end{itemize}

\noindent
Es wird nach wie vor geplant (durchaus auch im Laufe der Entwicklung), aber die Schwerpunkte der Methoden und Prinzipien verschieben sich von dem starren Konstrukt der sequentieller Phasenmodelle hin zu einem flexibleren Ansatz, bei dem der Kunde stärker und unmittelbarer in den Entwicklungsprozess involviert ist (Zwischenstände werden repräsentiert, Feedback fließt in die nächsten Iterationen ein\ldots).\\
Auch ist die Dokumentation nach wie vor ein wichtiges Mittel zur Kommunikation für die Entwickler, aber sie wird nicht zum Selbstzweck erstellt, wie bei dem Wasserfallmodell als notwendiges Ergebnis einer Phase.