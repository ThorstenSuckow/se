\section{Ingenieurmäßiges Vorgehen}

\begin{tcolorbox}
    \textbf{Software Engineering} ist dann erfolgreich eingesetzt worden, wenn der Nutzen der Software für den Kunden höher ist als die Kosten und dies in der Lebenszeit der Software so bleibt.
\end{tcolorbox}

\noindent
Der \textbf{Nutzen} von Software bestimmt sich bspw. aus Verkaufspreis und verkaufter Anzahl, Einsparungen, Einhaltung gesetzlicher Vorschriften usw.\\

\noindent
Die \textbf{Kosten} setzen sich i.d.R. aus Personalaufwand für die Entwicklung und Vermarktung , Schulung, Pflege der Software (auch: Dokumentation) usw. zusammen.\\

\subsection*{Kosten minimieren}
Die Aufgabe von SE ist auch, Kosten im Produktionsprozess sowie der späteren Wartung möglichst gering zu halten, durch
\begin{itemize}
    \item Einsatz bewährter Methoden und Werkzeuge
    \item Wiederverwendung von Lösungen
    \item Qualitätsarbeit, um Folgekosten für Korrekturarbeiten zu vermeiden
    \item gutes Teamklima (weniger Konflikte = höhere Produktivität, mehr Erfolg)
\end{itemize}

\subsection*{Nutzen maximieren}
Das \textbf{Paretoprinzip} besagt, dass in $20$\% der eingesetzten Zeit $80$\% der Ergebnisse geliefert werden\footnote{
s. \url{https://de.wikipedia.org/wiki/Paretoprinzip} - abgerufen 24.03.2024
}.\\
Umgekehrt bedeutet das, dass in $80$\% der Zeit nur $20$\% Ergebnisse erzielt werden.\\
Hier muss das Managemenet sicherstellen, dass nicht benötigte Funktionalität nicht (unnötigerweise) umgesetzt wird, und dass auf nicht benötigte Qualitätsziele verzichtet wird.\\

\subsection*{Machbarkeit}
Entwicklung von großen Systemen mit (sehr) vielen Mitarbeitern\footnote{
oder auch die Entwicklung von Systemen, an die spezielle Qualitätsanforderungen gestellt werden} ist ohne SE-Methoden nicht machbar; die Entwicklung in großen Teams ist bspw. nur durch ausgefeiltes Projektmanagement effektiv und effizient.