\chapter*{Teil 1 - Grundlagen der Softwaretechnik und Requirements Engineering}

\vspace{2cm}
\blockquote[{\url{https://fg-swt.gi.de}\footnote{abgerufen 04.04.2024}}]{``Unter \textbf{Softwaretechnik} (engl. \textit{Software Engineering}) versteht man allgemein die (Ingenieur-) Wissenschaft, die die kosteneffiziente Entwicklung von qualitativ hochwertiger Software behandelt.``}
\vspace{2cm}
\blockquote[{\cite[247]{BMEY07}}]{
    ``The amateur software engineer is always in search
    of magic, some sensational method or tool whose
    application promises to render software development trivial. It is the mark of the professional software engineer to know that no such panacea exists.
    Amateurs often want to follow cookbook steps; professionals know that right approaches to development usually lead to inept design products, born of a
    progression of lies and behind which developers can
    shield themselves from accepting responsibility for
    earlier misguided decisions. The amateur software
    engineer either ignores documentation all together,
    or follows a process that is documentation-driven,
    worrying more about how these paper products look
    to the customer than about the substance they contain. The professional acknowledges the importance
    of creating certain documents, but never does so at
    the expense of making sensible architectural innovations.``
}

\newpage
\section*{Lernziele Teil 1}

\begin{itemize}
    \item die Aufgaben und Ziele  des Software Engineerings kennen
    \item mit den Phasen des Entwicklungszyklus und ihren Inhalten vertraut sein
    \item einen Überblick über existierende Vorgehensmodelle besitzen und ihre Anwendbarkeit auf Problemstellungen beurteilen können
    \item in der Lage sein, Anforderungen geeignet aufzunehmen und schriftlich zu erfassen
\end{itemize}