\section{Zusammenfassung}

\begin{itemize}
    \item Unsystematisches Vorgehen bei der Softwareentwicklung verursacht Verzögerungen und Probleme
    \item Systematisches Vorgehen, wie es das \textit{Wasserfallmodell} in seinen 6 Phasen definiert, kann helfen, eine höhere Qualität
    durch Planbarkeit zu erreichen.\\
    Die einzelnen Phasen mit ihren jeweiligen Ergebnissen lauten:
    \begin{enumerate}
        \item \textbf{Anforderungen}: Es entsteht ein \textbf{Lastenheft}.
        \item \textbf{Analyse}: Das Domänen-Modell wird erarbeitet, es entsteht ein \textbf{Fachkonzept}.
        \item \textbf{Entwurf}: Das Fachkonzept wird zu einem \textbf{DV-Konzept} ausgearbeitet.
        \item \textbf{Implementierung}: Es entstehen \textbf{Programmcode} und \textbf{Datenbankschemen}.
        \item \textbf{Test}: Die in der vorherigen Phase erstellten Systemkomponenten werden in das System integriert, es finden Systemtests und Integrationstests statt, das Ergebnis sind \textbf{Testfälle} und \textbf{-protokolle}.
        \item \textbf{Betrieb}: Das System geht in den Betrieb über und wird über Aktualisierungen und Fehlerbehebeungen gepflegt.

    \end{enumerate}

\end{itemize}