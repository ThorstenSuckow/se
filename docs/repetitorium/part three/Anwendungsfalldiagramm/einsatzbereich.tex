\section{Einsatzbereich}

\begin{tcolorbox}
    Die \textbf{Anwendungsfallanalyse} wird im Rahmen der \textbf{Anforderungsspezifikation} eingesetzt.
\end{tcolorbox}

\noindent
Sie ist das am häufigsten eingesetzte Mittel zur Aufnahme und Darstellung von Anforderungen.\\

\noindent
\textbf{Anwendungsfalldiagramme} und Beschreibungen bilden die Grundlage für weitere Prozessschritte.

\vspace{2mm}
\begin{tcolorbox}
Der größere Informationsgehalt liegt bei den textlichen Beschreibungen der Anwendungsfälle.\\

\noindent
Anwendungsfalldiagramme beschreiben selbst kein Verhalten und keine Abläufe, sondern zeigen nur die Zusammenhänge der an Anwendungsfällen beteiligten Modellelemente und sind somit ein Hilfsmittel zur Anforderungsermittlung und Anforderungsverwaltung (vgl. \cite[213]{Oes05}).
\end{tcolorbox}
\vspace{2mm}

\noindent
Die \textbf{textliche Beschreibung} eines Anwendungsfalls sollte folgende Informationen beinhalten (vgl.~\cite[51]{Buh09}):

\begin{itemize}
    \item Beschreibung des Anwendungsfalls
    \item auslösender, primärer Akteur
    \item sonstige beteiligte Akteure
    \item Vor-/Nachbedingungen
    \item den essentiellen Ablauf
    \item ggf. Beschreibung alternativer Abläufe
    \item Priorität des Anwendungsfalls
\end{itemize}

\begin{tcolorbox}
    Zu den wichtigsten Schritten einer Anwendungsfallanalyse gehören (vgl.~\cite[Aufgabe 5.2, 88]{Buh09}):
    \begin{enumerate}
        \item Akteure identifizieren
        \item Anwendungsfälle identifizieren
        \item Beschreiben der Akteure und Anwendungsfälle
        \item Identifizieren von \textit{Schlüsselobjekten}, die das System verwaltet
        \item Identifizieren der wichtigsten Anwendungsfälle (Priorisierung)
        \item detailliertere Beschreibung der Anwendungsfälle
        \item Strukturierung des Anwendungsfalldiagramms
    \end{enumerate}
\end{tcolorbox}
