\section{Zusammenfassung}


\begin{itemize}
    \item Die vorgestellten erweiterten Konzepte werden vor allem im \textbf{Feinentwurf} von Klassenstrukturen eingesetzt.
    \item Generell gilt, dass die Spezifikationen von Objekten und Properties nur so detailliert ausgeführt werden sollen, wie nötig: Sind Details für das Verständnis unnötig, sollte darauf verzichtet werden.
    \item \textbf{Generalisierungsbeziehungen} sollten grundsätzlich modelliert werden.
    \item Ist eine \textbf{Komposition} statt einer Generalisierung für den Sachverhalt möglich, sollte diese verwendet werden: ``Favor object composition over inheritance.`` (\cite[19 f.]{GHJV94})
    \item \textbf{Kompositionsbeziehungen} geben eindeutige Impulse für eine Implementierung, \textbf{Aggregationsbeziehungen} sind in ihrer semantischen Aussage nicht sehr eindeutig
    \item \textbf{Assoziationsklassen} sind in ihrer frühen Designphase empfehlenswert, die Realisierung (mit {bspw.} Java) erfordert aber eine Transformation in eine volle Klasse, wobei die Multiplizitäten angepasst werden müssen
    \item Elemente können in \textbf{Paketen} zusamengefasst und unter einem gemeinsamen Namensraum gruppiert werden
\end{itemize}