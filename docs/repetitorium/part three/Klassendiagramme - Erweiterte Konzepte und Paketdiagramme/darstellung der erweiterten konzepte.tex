\section{Darstellung der erweiterten Konzepte}\label{sec:darstellung-der-erweiterten-konzepte}

\textbf{Erweiterte Konzepte} werden vor allem im \textbf{Detailentwurf} verwendet.\\

\noindent
Dabei werden \textit{nicht} alle Konzepte durch eigene \textbf{Symbolik} dargestellt: Mitunter werden gleiche Symbole mit unterschiedlichen \textbf{Keywords} (\textit{Schlüsselwörter}) versehen.\\

\noindent
Keywords werden in französischen Anführungszeichen (\textit{guillemets}: \guillemotleft Keyword\guillemotright) geschrieben und sind durch die Spezifikation vorgegeben (s.~\cite[746 ff.]{OMG17}).

\noindent
Durch Keywords werden auch \textbf{Standardstereotypen} ausgewiesen.\\
\textbf{Stereotypen} sind Erweiterungen von vorhandenen Modellierungkonzepten auf der Ebene des UML Metamodells und spezifizieren diese näher für einen vorgesehen Verwendungszeck.\\

\noindent
Stereotypen, die einer Anwendungsdomäne zugeordnet werden können, können in \textbf{Profile}s zusammengefasst werden\footnote{
Darstellung etwa: Package-Symbol mit Schlüsselwort \guillemotleft profile\guillemotright. Die UML definiert bspw. das \textit{UML Testing Profile}: \url{https://www.omg.org/spec/UTP2/}, abgerufen 05.05.2024
}

