\section{Einsatzbereich}

Mit \textbf{Zustandsautomaten} kann das \textit{Verhalten} von Systemen modelliert werden, wobei die \textit{Reaktionen} des Systems im Mittelpunkt stehen, und nicht die \textit{Aktionen}, wie bspw. bei den \textbf{Aktivitätsdiagrammen}.

\begin{tcolorbox}
    Ein \textbf{Zustandsautomat} kann den \textit{Lebensweg} eines Objektes modellieren, weshalb Zustandsautomaten oft im \textbf{Entwurf} als Ergänzung zu den \textbf{Klassendiagrammen} eingesetzt werden.
\end{tcolorbox}

\noindent
So wird der \textbf{Zustand} eines Objektes durch die Werte seiner \textit{Attribute} bestimmt, und in Zustandsdiagrammen können sinnvolle Kombinationen der Attributwerte modelliert werden.\\
Ein Objekt kann auf gleiche Anfragen in unterschiedlichen Zuständen unterschiedlich reagieren.\\

\subsection*{Eigenschaften}

Folgende Eigenschaften liegen Zustandsdiagrammen zugrunde:

\begin{itemize}
    \item endliche Anzahl von Zuständen\footnote{$S$, s. \textit{endlicher Automat}}
    \item eine endliche Anzahl möglicher Eingaben\footnote{$w \in L \subseteq \Sigma^*$}
    \item eine Übergangsfunktion (\textit{Transition}\footnote{$\delta$}) mit zwei Argumenten
    \begin{itemize}
        \item einem Anfangszustand
        \item einer Eingabe
    \end{itemize}
    \noindent
    und einem neuen Zustand als Rückgabewert
    \item einem Startzustand\footnote{$s_0 \in S$}
    \item einer Menge finaler Zustände\footnote{$F \subseteq S$}
\end{itemize}