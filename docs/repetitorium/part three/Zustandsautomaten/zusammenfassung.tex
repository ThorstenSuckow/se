\section{Zusammenfassung}

\begin{itemize}
    \item Das \textbf{Zustandsdiagramm} gehört wie das \textbf{Aktivitätsdiagramm} zu den \textbf{Verhaltensdiagrammen}.
    \item Ein Zustandsdiagramm zeigt den \textbf{Lebensweg} eines \textbf{Objektes}.
    \item \textbf{Attributwerte} bestimmen den Zustand eines Objektes.
    \item \textbf{Transitionen} zwischen Zuständen werden durch \textbf{Guards} und \textbf{Events} gesteuert.
    \item Während der Transitionen können \textbf{Effekte} (\textit{Aktivitäten}) realisiert werden.
    \item Die Modellierung \textbf{zusammengesetzter Zustände} ist ebenfalls möglich.
    \item Für die Darstellung \textbf{paralleler} oder  \textbf{konkurrierender} Abläufe können \textbf{Regionen} modelliert werden.
\end{itemize}