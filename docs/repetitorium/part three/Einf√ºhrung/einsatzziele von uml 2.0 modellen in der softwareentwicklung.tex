\section{Einsatzziele von UML 2.0 Modellen in der Softwareentwicklung}

\textbf{UML} 2.0 (\textit{Unified Modeling Language}) ist eine \textbf{Modellierungssprache}, um objektorientierte Softwaresysteme zu modellieren.\\

\noindent
UML Modelle können in einem gegebenen \textbf{Entwicklungsprozess} wie folgt eingesetzt werden:

\begin{itemize}
    \item \textbf{Sketching Modus} (Skizzen Modus)
    \item \textbf{Blueprint Modus} (Vorlagen Modus)
    \item \textbf{Vorlage zur Codegenerierung}
\end{itemize}

\subsection*{Sketching}
\textbf{Sketching} wird verwendet, um (vor allem in der \textbf{Anforderungsphase}) skizzenhaft Diagramme zu entwerfen.\\
Die Entwürfe dienen als Grundlage zur Diskusison eines Designs, oder um Architekturdokumente näher zu erläutern.\\
Die Diagramme enthalten nur wenige (technische) Details - nur die, die für die \textit{Entscheidungsaspekte} relevant sein können, sind enthalten (vgl.\cite[2]{Buh09}).

\subsection*{Blueprints}
\textbf{Blueprints} enthalten eine möglichst genaue Spezifikation eines (Sub-)Systems: Es müssen so viele Details auf den Diagrammen untergebracht werden, dass eine Implementierung zweifelsfrei vorgenommen werden kann.\\
\textbf{Blueprints} enthalten i.d.R. mehrere Diagrammtypen, um möglichst viele Aspekte der Implementierung zu beleuchten.

\subsection*{Grundlage für die Codegenerierung}
\textit{Codegeneratoren} können Quellcode auf Basis von UML Modellen generieren.\\
Der Code wird in der \textbf{Implementierungsphase} weiter ausgestaltet.
Man spricht in diesem Zusammenhang von der \textbf{modellgetriebenen Softwareentwicklung} (\textit{MDSD - model driven software development}) und von der \textbf{modellgetriebenen Architektur} (\textit{MDA - model driven architecture}).

\begin{itemize}
    \item \textbf{MDSD}: betrachtet den Einsatz von Modellen im Entwicklungsprozess, gibt Hinweise auf Prozessaktivitäten und (bewährte) Einsatzpraktiken für Modelle
    \item \textbf{MDA}: Grundgedanke: Geschäftsprozesse sind stabiler als {techn.} Plattformen, gleiche Geschäftsprozesse müssen auf unterschiedlichen Plattformen verteilt werden.\\
    $\rightarrow$ verschiedene Modellierungsebenen: Geschäftslogik wird plattformunabhängig modelliert, Transformation auf ein plattformabhängiges Modell (bzw. Codegenerierung aus plattformabhängigen Modell) erfolgt automatisch
\end{itemize}
