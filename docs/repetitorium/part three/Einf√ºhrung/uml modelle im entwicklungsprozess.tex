\section{UML 2.0 Modelle im Entwicklungsprozess}
Die \textbf{UML} ist nicht an einen bestimmten Entwicklungsprozess gebunden:
UML Modelle können sowohl in agilen Prozessen als auch im Wasserfallmodell eingesetzt werden.\\

 \begin{tcolorbox}[colback=white]
     Bei iterativen Prozessen empfiehlt es sich nicht, sehr detaillierte Diagramme anzufertigen, da Änderungen und Anpassungen zu erwarten sind.\\
     Stattdessen sollte \textbf{Sketching} verwendet werden, um ein Design für eine folgende Iteration zu skizzieren (vgl.~\cite[14]{Buh09}).
 \end{tcolorbox}