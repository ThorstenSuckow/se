\section{Einsatzbereich}

Das \textbf{Aktivitätsdiagramm} gehört zu den \textbf{Verhaltensdiagrammen} und stellt dar, wie \textit{Verhalten} realisiert wird.\\

\begin{tcolorbox}[title=Einsatzbereich]
Aktivitätsdiagramme können in \textit{allen} Phasen des Softwareentwicklungsprozesses eingesetzt werden:

\begin{itemize}
    \item \textbf{Analyse}: Modellierung von Geschäftsprozessen
    \item \textbf{Anforderung}: Verhalten von Anwendungsfällen
    \item \textbf{Entwurf}: Darstellung von Systemverhalten, Vorlage zur Umsetzung in Sourcecode
    \item \textbf{Implementierung}: Darstellung komplexer Algorithmen
    \item \textbf{Testphase}: Ableitung von Testfällen
\end{itemize}
\end{tcolorbox}

\noindent
\textit{Buhl} hebt als Vorteil die sehr übersichtliche Darstellung paralleler Abläufe hervor (\cite[57]{Buh09}).