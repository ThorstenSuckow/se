\section{Einsatzbereich}

Das \textbf{Klassendiagramm} ist eins der ältesten und wichtigsten Bestandteile der UML.

\vspace{2mm}
\begin{tcolorbox}[title=Klassendiagramm]
    Das \textbf{Klassendiagramm} beschreibt die Klassen eines Systems und die statischen Beziehungen zwischen ihnen (\cite[17]{Buh09}).\\
    Sie werden in Form von Domainklassen bei der Analyse und als detaillierte Entwürfe von Systemklassen als \textit{Blueprint} im gesamten Entwicklungsprozess eingesetzt.
\end{tcolorbox}
\vspace{2mm}

\noindent
Es gehört zu den \textbf{Strukturdiagrammen} (s. Abbildung~\ref{fig:diagrammuebersicht}) und stellt von allen Diagrammtypen die größte Anzahl von \tetxtbf{Modellierungskonzepten}\footnote{s. Abschnitt~\ref{sec:sprachaufbau-der-uml-2.0}} zur Verfügung.\\

\noindent
Klassendiagramme zeigen charakteristische Merkmale (\textit{Features}) der Klasse, also inhaltliche Eigenschaften wie

\begin{itemize}
    \item Attribute
    \item Operationen
    \item Einschränkungen für deren Nutzung
\end{itemize}