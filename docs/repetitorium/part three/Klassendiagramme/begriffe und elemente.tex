\section{Begriffe und Elemente}

\subsection*{Properties}
Es gibt zwei unterschiedliche Notationen für Properties: Die \textbf{Attribut}- und die \textbf{Assoziationsnotation}.

\begin{itemize}
    \item \textbf{Attributnotation}: Textzeile innerhalb des zweiten \textit{Compartments} der Klassenbox.\\
    Neben \textbf{Instanzattributen}, die die Eigenschaften eines Objektes charakterisieren, können auch \textbf{Klassenattribute} angegeben werden: Der Bezeichner wird dann \underline{unterstrichen}.\\
    Die formale Notation für ein Attribut lautet:
    \begin{equation}\notag
        \text{[Sichtbarkeit][/] name [:Typ][[Multiplizität]][=Vorgabewert][\{eigenschaftswert\}*]}
    \end{equation}
    Eckige Klammern stehen hierbei für \textit{optionale} Angaben\footnote{
        die \textbf{Multipliziät} wird immer in eckigen Klammern angegeben, das zweite Klammernpaar steht hier für die optionale Angabe
    }.
    \begin{itemize}
        \item \textbf{Sichtbarkeit}:
            \begin{itemize}
                \item \code{+} \textit{public}
                \item \code{#} \textit{protected}
                \item \code{-} \textit{private}
                \item \code{~} \textit{package private}
            \end{itemize}
        \item \textbf{/}: es handelt sich um ein \textbf{abgeleitetes Attribut}\footnote{
            \textit{abgeleitet}: der Wert wird berechnet und nicht durch den Anwender angegeben oder aus einer externen Quelle geladen
        }
        \item \textbf{Multiplizität}:
        \begin{itemize}
            \item \code{[0..*]}: optionales Attribut, gekennzeichnet durch $0$ als untere Grenze.
            Der Stern $*$ zeigt an, dass bel. viele Ausprägungen vorhanden sein dürfen.
            Statt \code{[0..*]} kann in dem Fall auch \code{[*]} gnotiert werden.
            \item \code{[1..n]}: obligatorische Property, gekennzeichnet durch $1$ als untere Grenze ($n >= 1 \land n=*$):
            \item \code{[1]}: Standardwert
        \end{itemize}
        \item \textbf{Vorgabewert}: Initialisierungswert
        \item \textbf{Eigenschaftswert}: \textit{Tagged Values}.
        Stellen besondere Eigenschaftswerte bezogen auf das \textbf{Modellelement} (hier: \textit{Attribut}) dar, {bspw.} \code{{readOnly}}, \code{{ordered}}, \code{{frozen}} usw. Werden keine Vorgaben der UML genutzt, hat der Eigenschaftswert die Form \code{{Eigenschaft=Wert}} (so ist bspw. \code{{readOnly}} eine Kurzschreibweise für  \code{{readOnly=true}})
        \item \textbf{Einschränkungen}: Zusätzliche Einschränkungen, bspw. in der Notation der \textbf{OCL}, werden ebenfalls in geschweiften Klammern angegeben.
        Diese Anforderungen, die das Attribut erfüllen muss, müssen als boolescher Ausdruck formuliert werden, so dass die Anforderung zu \code{true} oder \code{false} ausgewertet werden kann.
        Einschränkungen können direkt hinter das Attribut, als Notiz oder ausgelagert in einer Textdatei angegeben werden.
    \end{itemize}

\end{itemize}

\vspace{2mm}
\begin{tcolorbox}[title=Fehlende Angaben zur Multiplizität]
    \textit{Fowler} empfiehlt die Angabe von \textbf{Multiplizitäten}, um Unklarheiten zu vermeiden: Wird eine Multiplizität nicht angegeben, kann der Autor damit ausdrücken wollen, dass die Standard-Multiplizität $[1]$ gemeint ist - es kann aber auch sein, dass einfach keine Angaben dazu im Diagramm gemacht worden sind, weshalb man per se nicht von einer Multiplizität von $[1]$ ausgehen darf.
    \blockquote[{\cite[39]{Fow03b}}]{
        The default implicity of an attribute is [1]. Although this is true in the meta-model, you can't assume that an attribute in a diagram that's missing a multiplicity has a value of [1], as the diagram may be suppressing the multiplicity information. As a result, I prefer to explicitly state a [1] multiplicity if it's important.
    }
\end{tcolorbox}
\vspace{2mm}

\begin{tcolorbox}[title=Nur ein Wert $n > 1$ bei der Multiplizität,colback=red!20]
    In~\cite[19]{Buh09} vermerkt \textit{Buhl}:
    \blockquote{
    [1] - untere Grenze 1 und obere Grenze 1.[\ldots] Ist nur ein Wert angegeben, so wird die untere Grenze grundsätzlich mit 1 angenommen.
    }\\

    \noindent
    Im Gegensatz hierzu findet sich bei \textit{Oestereich}:\\
    \blockquote[{\cite[274, Hervorhebung eigene]{Oes05}}]{
        $0..3,\quad 7,\quad 9..*\quad$ von null bis drei oder \textbf{genau sieben} oder größer oder gleich neun
    }\\
    sowie in der Spezifikation:\\
    \blockquote[{\cite[35]{UML17}}]{
        If the lower bound is equal to the upper bound, then an alternate notation is to use a string containing just the upper
        bound.
    }\\
    \textit{Balzert} argumentiert in \cite[43]{Bal05} gleich.
\end{tcolorbox}

\subsection*{Assoziationsnotation}

\begin{tcolorbox}
    Eine Assoziation repräsentiert die semantische Beziehung zwischen Klassen.
\end{tcolorbox}

Die \textbf{Assoziationsnotation} verwendet durchgehende Linien zwischen Klassen.\\

\noindent
Assoziationen können Namen besitzen, die mit einer Leserichtung ausgestattet sind (bspw. durch eine Pfeilspitze).\\

\noindent
\textbf{Assoziationsenden} können darüberhinaus mit folgenden Informationen versehen werden:

\begin{itemize}
    \item \textbf{Rollennamen}, zusätzlich mit \textit{Sichtbarkeiten}
    \item \textbf{Eigenschaftswerte}\footnote{
    ``Die bedeutung der von der UML vorgegebenen Eigenschaftswerte hat gegenüber den Attributen eine leicht abweichende Semantik.`` (\cite[20]{Buh09})
    }
    \item \textbf{Einschränkungen}
    \item \textbf{Multiplizitäten}
\end{itemize}

\noindent
\textbf{Pfeilspitzen} an Assoziationsenden geben die \textbf{Navigierbarkeit} an:

\begin{itemize}
    \item \textbf{Bidirektional}: Pfeilspitzen an beiden Enden
    \item \textopf{Unidirektional}: Pfeilspitze an nur einem Ende
    \item \textbf{keine Angabe}: bidirektionale Assoziation, oder Navigierbarkeit wurde noch nicht modelliert\footnote{
     zur Vermeidung von Unklarheiten empfiehlt es sich, die Navigierbarkeit anzugeben. S. auch die obige Anmerkung zur \textit{Multiplizität}
    }
\end{itemize}

\subsection*{Anmerkungen}
\textit{Buhl} weist darauf hin, dass

\begin{itemize}
    \item Eigenschaftswerte von Properties in einer Implementierung (Java) umgesetzt werden müssen
    \item es günstig ist, Klassen \textit{restriktiv} zu entwerfen: Rechte können in Unterklassen erweitert werden.\\
    Rechte können aber nicht in Unterklassen eingeschränkt werden.
\end{itemize}