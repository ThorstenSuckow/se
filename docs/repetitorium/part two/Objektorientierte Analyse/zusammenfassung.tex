\section{Zusammenfassung}


\begin{itemize}
    \item in der \textbf{Analyse} werden \textbf{Modelle} genutzt, um Anforderungen zu verstehen und Lösungen aus fachlicher Sicht zu entwickeln
    \item hierzu werden Modelle des Umfelds und der Lösung auf formalisierte Art erstellt
    \item diese dienen den Modellen des \textbf{Entwurfs}
    \item Modelle bestehen aus \textbf{statischen} und \textbf{dynamischen} Modellteilen, wobei statische Modellteile die Teile und ihre Beziehungen untereinander modellieren, und dynamische das Zusammenspiel der Teile beschreiben
    \item In der OO-Analyse wird statische Modellierung mit \textbf{Klassendiagrammen} realisiert.
    \item[] Dynamische Modelle, die Zustände und Übergänge von Zuständen beschreiben, werden mittels \textbf{Zustandsdiagrammen} modelliert.
    \item[] Beschreiben dynamische Modelle Abläufe von Aktivitäten, nutzt man \textbf{Aktivitätsdiagramme}.
    \item Bei der Analyse kommen außerdem Analysemuster zum Einsatz: Muster beschreiben Lösungskizzen verallgemeinerter Probleme
    \item[] Sie helfen, indem sie \textbf{bewährte Lösungen} beschreiben und ein \textbf{Kommunikationsmittel} darstellen
    \item Die wichtigsten Analysemuster sind \textbf{Exemplartyp}, \textbf{Wechselnde Rollen} und \textbf{Allgemeine Hierarchie}.
    \item Die Gestaltung der Benutzeroberfläche ist Bestandteil der Analyse, wobei logische Struktur, die Informationsarchitektur und die physische Gestalt entworfen wird.
    \item[] Ergonomie und Barrierefreiheit sind wichtige Punkte, die dabei beachtet werden müssen.
\end{itemize}