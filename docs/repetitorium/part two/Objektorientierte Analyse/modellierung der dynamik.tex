\section{Modellierung der Dynamik}

\noindent
In der \textbf{statischen Modellierung} im Domänenmodell werden Klassen in ihrem \textit{Zusammenhang} modelliert.\\

\noindent
Das statische Modell gibt keine Informationen über den Zustand eines Objektes oder wie Botschaften untereinander ausgetauscht werden, und wie sich Objekte / KLassen daraufhin verhalten.\\

\noindent
Um darzustellen, wie sich Klassen oder Objekte verhalten, nutzt man das \textbf{dynamische Modell}.\\

\noindent
Hierzu wird in der \textbf{Analyse} Zustände und Zustandsübergänge eines Systems sowie Aktivitäten modelliert.\\
Grundlage sind Dokumente der Anforderungen, die Verhalten beschreiben, also User Storys, Anwendungsfälle und Szenarien.\\
Die dort beschriebenen Abläufe werden bei der dynamischen Modellierung konkretisiert und genauer definiert.

\subsection*{Zustände und Zustandsänderungen}
\textbf{Systeme} (\textit{Klassen}, \textit{Subsysteme}, \textit{Anwendungen}) können \textbf{Zustände} besitzen, die sich verändern können.\\

\noindent
Oft sind nur Übergänge zwischen bestimmten Zuständen möglich.\\

\noindent
Die grafische Darstellung dieses Verhaltens erreicht man über \textbf{UML-Zustandsdiagramme}.

\subsection*{Aktivitätsdiagramme}
Bei Zustandsdiagrammen erschließt sich der Ablauf von Aktivitäten meist nur indirekt, da bspw. keine Auskunft über die beteiligten Akteure modelliert werden können.\\

\noindent
Aus diesem Grund verwendet man \textbf{UML-Aktivitätsdiagramm}, in denen Aktivitäten in ihrer möglichen Abfolge und ihrer Zuordnung zu Akteuren modelliert werden.\\

\noindent
Darüberhinaus kann mit Aktivitätsdiagrammen  auch der Fluss von Informationen un die Auswirkung auf Objekte dargestellt werden.\\

\noindent
Aktivitätsdiagramme sind zur Modellierung von User Storys, Szenarien oder Anwendungsfällen bestens geeignet.