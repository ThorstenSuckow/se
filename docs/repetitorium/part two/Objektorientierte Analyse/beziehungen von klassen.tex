\section{Beziehungen von Klassen}\label{sec:beziehungen-von-klassen}

Klassen können auf drei unterschiedliche Arten miteinander in Beziehung stehen:

\begin{itemize}
    \item \textbf{Abhängigkeit} \textit{Dependency}
    \item \textbf{Assoziation} \textit{Association}
    \item \textbf{Vererbung} \textit{Inheritance}
\end{itemize}

\noindent
Bei der Analyse müssen diese Arten von Beziehungen ermittelt werden, damit entsprechend modelliert werden kann.\\

\subsection*{Abhängigkeiten}
\textbf{Abhängikeiten} nutzten eine gestrichelte Linie.\\
Abhängigkeiten werden \textit{in der Analyse} i.d.R. selten modelliert.\\
Als Beispiel für eine Abhängigkeit sei eine \textit{Benutzt-Beziehungen} in Form eines \textit{formalen Parameter} gegeben: Der aktuelle Parameter überdauert den Aufruf einer Methode in einer Klasse, für die diese Abhängigkeit modelliert werden soll, nicht.\\
Gleicherweise kann auch eine \textit{lokale Variable} eines bestimmten Typs eine Abhängigkeit bedeuten.

\subsection*{Assoziation}
Unter einer \textbf{Assoziation} versteht man, dass ein Objekt ein anderes Objekt derselben oder anderen Klasse \textit{dauerhaft} kennt.\\
Assoziationen werden anhand einer durchgezogenen Linie dargestellt.\\
Wird in einem Objekt einer bestimmten Klasse in einem Attribute eine Referenz auf ein Objekt einer anderen oder derselben Klasse gespeichert, existiert eine Assoziation zwischen diesen beiden Klassen.

\subsection*{Vererbung}
Eine \textbf{Vererbung} bedeutet in Vererbungsrichtung eine \textbf{Spezialisierung}, entgegengesetzt eine \textbf{Generalisierung}\footnote{
bzw. \textit{Verallgemeinerung} im Skript (vgl.~\cite[7]{Wed09b}).
}.

\subsection*{Polymorphie}
Eine weitere Bedeutung der Vererbungsbeziehung, ist, dass eine Instanz einer Klasse genauso verwendet werden kann wie eine Instanz seiner übergeordneten Klasse. Allgemein faßt man dies unter \textbf{Polymorphie} zusammen.\\
Als Konsequenz folgt, dass immer dann, wenn ein bestimmter Typ gefordert wird, auch sein Untertyp erlaubt ist (vgl.~\cite[466]{Ull23}), was der Kern des \textit{Liskovschen Substitutionsprinzips}\footnote{
    ``Wikipedia - Liskov substitution principle``: \url{https://en.wikipedia.org/wiki/Liskov_substitution_principle}, abgerufen 11.04.2024
} ist\footnote{
    s.a. ``Wikipedia - Covariance and Contravariance``: \url{https://en.wikipedia.org/wiki/Covariance_and_contravariance_(computer_science)}, , abgerufen 11.04.2024
}.\\
