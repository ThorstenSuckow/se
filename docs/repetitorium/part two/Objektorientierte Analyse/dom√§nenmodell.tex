\section{Domänenmodell}

\noindent
Der Begriff \textbf{Domäne} bezeichnet das fachliche Umfeld der Aufgabe.\\

\noindent
Das \textbf{Domänenmodell} ist das \textit{statische Modell} der Domäne.\\

\noindent
In der \textbf{Analysephase} wird das Domänenmodell in Form von Klassendiagrammen erstellt.
Es beschreibt damit


\begin{itemize}
    \item welche Klassen es in einer Domäne gibt
    \item wie diese Klassen ausgeprägt sind
    \item welche Beziehungen sie untereinander haben
\end{itemize}

\noindent
Als \textbf{Artefakt der Analyse} gilt für das Domänenmodell, dass auf eine Darstellung technischer Gesichtspunkte verzichtet wird.\\

\noindent
Für die weitere Arbeit ist das Domänenmodell

\begin{itemize}
    \item Grundlage für die Entwicklung der grafischen Benutzeroberfläche
    \item sowie in der \texbf{Realisierungsphase} Basis für
        \begin{itemize}
            \item die Implementierung der Datenhaltung der Software
            \item Datenbankmodelle
            \item Formate von Daten
            \item Schnittstellen zwischen Anwendungen
        \end{itemize}
\end{itemize}

\subsection{Erstellung eines Domänenmodells}
Das \textbf{Domänenmodell} basiert auf den Anforderungen und den in der Anforderungsphase gesammelten Informationen.\\

\noindent
Die einzelnen Teile des Modells können systematisch ermittelt werden, bspw. in folgender Reihenfolge:

\begin{enumerate}
    \item Klassen
    \item Assoziationen
    \item Attribute
    \item Vererbungsbeziehungen
    \item Verantwortlichkeiten oder Operationen
\end{enumerate}

\subsection*{Klassen ermitteln}
Werden \textbf{User Stories} oder \textbf{Anwendungsfälle} erfasst, stehen die \textit{Hauptwörter} in den Anforderungen meistens für \textbf{Klassen}, \textbf{Objekte} oder \textbf{Attribute}.

\subsection*{Assoziationen}
\textbf{Assoziationen} erkennt man an Beschreibungen wie
\begin{itemize}
    \item \textit{hat}
    \item \textit{besitzt}
    \item \textit{kontrolliert}
    \item \textit{ist verbunden mit}
    \item \textit{ist Teil von}
    \item \textit{hat Teil}
    \item \textit{ist Mitglied von}
    \item \textit{besitzt Mitglied}
\end{itemize}

\subsection*{Attribute ermitteln}

\noindent
Hinter den Formulierungen \textit{hat} und \textit{besitzt} verbergen sich auch häufig \textbf{Attribute}.\\

\noindent
\textbf{Attribute} sind Informationen, die eine Klasse besitzt und pflegt.

\subsection*{Vererbungsbeziehungen}
\textbf{Vererbungsbeziehungen} erkennt man daran, dass Daten oder Verhalten mehrerer Klassen identisch sind.

\vspace{2mm}
\begin{tcolorbox}
    Für die Beziehung zwischen einer abgeleiteten Klasse und ihrer Superklasse muss gelten
    \textit{kann verwendet werden als} und \textit{ist ein}\footnote{
    s. Abschnitt~\ref{sec:beziehungen-von-klassen}
    } (vgl.\cite[16]{Wed09b}).
\end{tcolorbox}
\vspace{2mm}


\subsection*{Verantwortlichkeit}
Unter dem Begriff \textbf{Verantwortlichkeit} fasst man die Summe des Verhaltens einer Klasse zusammen.\\
Zum typischen Verhalten gehören bspw.

\begin{itemize}
    \item \textit{lesen}
    \item \textit{schreiben}
    \item \textit{suchen}
    \item \textit{berechnen}
    \item \textit{entscheiden}
\end{itemize}

\subsection*{Anzahl der Verantwortlichkeiten einer Klasse}
Wenn eine Verantwortlichkeit nicht eindeutig einer Klasse zugeordnet werden kann, sollte man überprüfen, ob ein Fehler in der Modellierung vorliegt oder ob explizit eine Klasse modelliert werden muss, die diese Verantwortlichkeit trägt.\\

\noindent
Klassen sollten max. ein bis zwei Verantwortlichkeiten haben.\\
Sollten mehrere Verantwortlichkeiten vorliegen, sollte man überlegen, die Klasse aufzuteilen.