\section{Beispiel}

Bei \cite[43 ff.]{Wed09b} findet sich der beispielhafte Weg zu der Entwicklung einer Architektur für das in dem Skript vorgestellte \textit{Bibliothekssystem}.\\

\noindent
Die Systemidee schreibt den Einsatz einer Datenbank zur Speicherung und zum Durchsuchen der Daten vor.\\
Weitere Anforderungen, aus denen auch Anforderungen an die Architektur entstehen, lauten wie folgt:

\begin{itemize}
    \item \textbf{N1: Mehrbenutzerbetrieb}
    \item[] Alle Daten liegen zentral, was u.a. für eine \textbf{Client-Server}-Architektur spricht
    \item \textbf{N2: Robustheit}
    \item[] Daten der Bibliothek dürfen nicht verloren gehen, also muss ein Backup aller Daten möglich sein
    \item \textbf{N3: Verfügbarkeit}
    \item[] Ausfälle müssen vermieden werden, die Komponenten auf Hardwareseite müssen redundant sein.
    \item[] Für die Datenbank wird \textbf{aktive Redundanz}\footnote{s. Glossar} als Taktik gewählt
    \item \textbf{N4: Benutzerfreundlichkeit}
    \item[] mit \textbf{N6} und \textit{Verfügbarkeit als Internetanwendung} folgt diese Anforderung
    \item \textbf{N5: Geringe Antwortzeiten}
    \item[] Das System muss skalierbar sein und leicht mit zusätzlicher Hardware erweitert werden können.
    \item[] Es wird mit \textit{Java Enterprise Edition (JSS)} eine Referenzarchitektur gewählt, mit der das Entwicklerteam bereits gute Erfahrungen bei skalierbaren Webanwendungen gemacht hat.
    \item[] Ein Lastverteiler verteilt die Anfragen auf die verfügbaren Systeme.
    \item \textbf{N6: Verteilte Anwendung}
    \item[] Die Benutzer sollen über das Internet an die Daten der Bibliothek kommen, weshalb die Anwendung als Webanwendung geplant wird.
    \item \textbf{N7: Integrität}
    \item[] Da die Anwendung über das Internet erreichbar ist, muss sichergestellt werden, dass ausreichende Sicherheitsmechanismen den Nutzerrollen entsprechend Zugriff auf Daten gewährt oder verweigert
\end{itemize}