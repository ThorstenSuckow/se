\section{Strukturen und Verhalten}
Wie bei der Analyse bestehen auch Klassen des Entwurfs aus \textbf{statischen} und \textbf{dynamischen Modellen}.\\

\noindent
Statische Modelle werden durch \textbf{Paket-} und \textbf{Klassendiagramme} beschrieben\footnote{
\textit{Paketdiagramme} werden auch zur Dokumentation der Architektur genutzt.
}.\\

Dynamische Modelle werden durch \textbf{Zustands-} und \textbf{Sequenzdiagramme} beschrieben\footnote{
\textit{Wedemann} weist darauf hin, dass auch \textit{Aktivitätsdiagramme} eingesetzt werden, Praktiker die \textit{Sequenzdiagramme} im Entwurf aber für sinnvoller halten (vgl.\cite[49]{Wed09b}).
}.\\

\noindent
\textbf{Klassendiagramme} werden wie in der Analyse eingesetzt, allerdings kommen im Entwurf technische Details dazu, wie die \textit{Sichtbarkeiten}, die \textit{Typen} von Attributen, Rückgabewerten und formaler Parameter.\\
Es empfiehlt sich, Gerüstmethoden wie \textit{Getter}/\textit{Setter} sowie Konstruktoren der Übersicht halber wegzulassen (s. \cite[50]{Wed09b}).\\

\noindent
\textbf{Sequenzdiagramme} modellieren den zeitlichen Ablauf von Nachrichten zwischen Objekten.\\
Für Notation und Beispiele s. Abschnitt~\ref{sec:sequenzdiagramme}.