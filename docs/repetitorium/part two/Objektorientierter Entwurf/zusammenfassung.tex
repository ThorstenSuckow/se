\section{Zusammenfassung}


\begin{itemize}
    \item im \textbf{Entwurf} werden Klassen und Dateien eines zu entwerfenden Systems \textbf{geplant}
    \item dabei beruht der Entwurf auf den \textbf{Anforderungen}, den \textbf{Ergebnissen der Analyse} und der \textbf{Architektur}
    \item die Klassen des Entwurfs entsprechen den Klassen der verwendeten objektorientierten Sprache
    \item die \textbf{Struktur der Klassen} wird mit \textbf{UML-Klassendiagrammen} definiert
    \item \textbf{Abläufe} und das \textbf{Zusammenspiel} von Klassen wird mit \textbf{UML-Sequenzdiagrammen} modelliert
    \item der Entwurf sollte sinnvollerweise \textbf{schrittweise} (\texit{iterativ}) entwickelt werden
    \item hierbei helfen grundlegende Prinzipien wie ``\textit{Reise mit leichtem Gepäck}``
    \item Entwurfsmuster wie \textbf{Beobachter}, \textbf{MVC} und \textbf{Fassade} sind wichtige Hilfsmittel beim Entwurf
    \item anhand von \textbf{Prinzipien guten Entwurfs} kann die \textbf{Qualität eines Entwurfs} beurteilt werden
    \item das wichtigste Prinzip hierbei ist die Aufteilung der Gesamtaufgabe in \textbf{kleine Teile}, wobei die Teile lose untereinander gekoppelt sein sollen, was durch eine \textbf{hohe Kohäsion} (\textit{Zusammenhalt}) sowie \textbf{Vermeidung starker Kopplung} (Globale Variablen, lange formale Parameterlisten in den Methoden-Signaturen usw.) erreicht werden kann, außerdem durch die Beachtung guten Entwurfs wie Vermeidung zirkulärer Abhängigkeiten
    \item die Qualität eines Entwurfs bleibt gut, indem der Code durch \textbf{Refactorings} stetig verbessert wird
\end{itemize}