\section{Zusammenfassung}


\begin{itemize}
    \item im \textbf{Entwurf} werden Klassen und Dateien eines zu entwerfenden Systems \textbf{geplant}
    \item dabei beruht der Entwurf auf den \textbf{Anforderungen}, den \textbf{Ergebnissen der Analyse} und der \textbf{Architektur}
    \item die Klassen des Entwurfs entsprechen den Klassen der verwendeten OO Sprache
    \item die Struktur der Klassen wird mit UML-Klassendiagrammen definiert
    \item Abläufe und das Zusammenspiel von Klassen wird mit UML-Sequenzdiagrammen modelliert
    \item der Entwurf sollte sinnvollerweise schrittweise entwickelt werden
    \item hierbei helfen grundlegende Prinzipien wie ``\textit{Reise mit leichtem Gepäck}``
    \item Entwurfsmuster wie Beobachter, MVC und Fassade sind hierbei wichtige Hilfsmittel beim Entwurf
    \item anhand von Prinzipien guten Entwurfs kann die Qualität eines Entwurfs beurteilt werden
    \item das wichtigste Prinzip hierbei ist die Aufteilung der Gesamtaufgabe in kleine Teile, wobei die Teile lose untereinander gekoppelt sein sollen, was durch eine hohe Kohäsion (\textit{Zusammenhalt}) sowie Vermeidung starker Kopplung (Globale Variablen, lange formale Parameterlisten in den Methoden-Signaturen usw.) erreicht werden kann, außerdem durch die Beachtung guten Entwurfs wie Vermeidung zirkuläre Abhängigkeiten
    \item die Qualität eines Entwurfs kann verbessert werden, indem der Code durch Refactorings verbessert wird
\end{itemize}