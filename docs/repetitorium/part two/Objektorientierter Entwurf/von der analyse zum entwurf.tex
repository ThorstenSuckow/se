\section{Von der Analyse zum Entwurf}

\noindent
Im \textbf{Entwurf} beschreiben Entwickler


\begin{itemize}
    \item aus welchen \textbf{Klassen}, \textbf{Methoden} oder \textbf{Dateien} das zu entwerfende System bestehen soll
    \item wie diese Bauteile aussehen sollen
    \item in welcher Beziehung sie zueinander stehen sollen
    \item wie diese Bauteile zusammenarbeiten
\end{itemize}

\noindent
Im \textbf{objektorientierten Entwurf} bedeutet dies

\begin{itemize}
    \item das Erstellen von Klassen samt deren Attribute und Methode
    \item das Festlegen der Assoziationen und Vererbungsbeziehungen
    \item das Beschreiben des Zusammenspiels der Instanzen
\end{itemize}

\subsection*{Basis, Anforderungen, Analyse und Architektur}
Der Entwurf geschieht auf Grundlage der Anforderungen und der Ergebnisse der Analyse und innerhalb der gewählten Architektur.\\

\noindent
Aus den Anforderungen sind für den Entwurf die \textbf{funktionalen Anforderungen} wichtig.\\
Die \textbf{nicht-funktionalen Anforderungen} sind bereits in der Architektur berücksichtigt.\\

\noindent
Die Architektur liefert die Bauteile für die Software, der Entwurf konkretisiert diese dann anhand der funktionalen Anforderungen.\\

\noindent
Die Ergebnisse der \textbf{Analyse} in Form des \textbf{Domänenmodells} ist Grundlage für die \textit{Datenhaltung} in der Anwendung und für eventuelle \textit{Datenbankschemas}.\\

\noindent
Die \textbf{Definition} der Schnittstellen, insb. der GUI, bilden die Basis für den Entwurf der technischen Umsetzung der Schnittstelle.

\subsection*{Vor- und Nachteile}
Probleme werden meist erst beim Modellieren klar.
Es bietet sich deshalb an, erst einen groben Entwurf zu skizzieren.
Der Entwurf kann (eindeutige) Absprachen unter den Projektbeteiligten erleichtern.\\

\noindent
Größere Entwürfe bzw. sehr detaillierte sollten im ersten Anlauf vermieden werden - stattdessen sollte iterativ gearbeitet werden: Es werden erst Teile entworfen und umgesetzt, und basierend auf den gemachten Erfahrungen werden Entwürfe angepasst.


