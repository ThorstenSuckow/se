\section{Klassen im Entwurf}

Im Unterschied zu den Entwürfen aus der Analyse  beschreiben Klassen des Entwurfs nicht fachliche Zusammenhänge, sondern Klassen, die in der gewählten objektorientierten Sprache umgesetzt werden.\\

\noindent
Damit besitzen die Klassen des Entwurfs \textbf{technische Verantwortichkeiten}, wie

\begin{itemize}
    \item Datenhaltung (\textit{Entity})
    \item Durchführung von Berechnungen
    \item Überprüfung von Regeln
    \item Interaktion mit dem Anwender (\textit{User Interface})
    \item Ablaufsteuerung (\textit{Control})
    \item Schnittstellen zu anderen Systemen (\textit{Boundary})
    \item Services für andere Klassen (\textit{Utility})
\end{itemize}

\subsection*{Eine Verantwortlichkeit pro Klasse}
Wie in der Analyse gilt: jede klasse sollte eine, max. zwei Verantwortlichkeiten besitzen.\\
$\rightarrow$ die Klassen werden dadurch leichter verständlich und änderbar.\\

\noindent
Klassen mit ähnlichen Verantwortlichkeiten werden gemeinsam in \textbf{Schichten} oder \textbf{Module}\footnote{
bspw. \textit{Pakete} bei \textit{Wedemann} (vgl.\cite[49]{Wed09b}). In einem System mit hoher \textit{Kohäsion} liegt eine starke funktionale Abgrenzung der Module, Klassen und Methoden vor (vgl.\cite[310 ff.]{Dem79}).
} verteilt.\\
Viele Klassen, die jeweils wenig Funktionalität besitzen, sind einfacher zu verstehen, als wenig Klassen, die unübersichtlich viel Funktionalität besitzen (vgl. \cite[49]{Wed09b})}\footnote{
s. a. \textit{Martin} in \cite[95, The Single-Responsibility Principle]{Mar03}: ``A class should have only one reason to change ``
}. \\



\noindent
Die \textbf{Übersicht} über die Vielzahl der Klassen wird durch einen \textbf{sauberen Entwurf} gewährleistet.