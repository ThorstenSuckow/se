\section{Verbesserung des Entwurfs durch Refactoring}
Nur selten kann ein neues System von Anfang an so entworfen werden, dass der Entwurf für das gesamte System  durchgängig geeignet und zweckmäßig ist.
Wie in Abschnitt~\ref{sec:vorgehen-beim-entwerfen} diskutiert ist es i.d.R. notwendig, \textbf{iterativ} vorzugehen.
Der Entwurf wird so immer wieder angepaßt.\\
Somit können Schwächen früherer Entwürfe entdeckt und entfernt werden.\\
Sollten sich ansonsten kleine und große Fehler im Entwurf anhäufen, können die dadurch entstehenden starken Kopplungen den Code wenig erweiter- und wartbar machen.\\

\noindent
Es ist allerdings nicht einfach, bereits kleine Mengen existierenden Codes zu ändern.\\
Den Entwurf bestehender Software zu verbessern, ohne dabei die Funktionalität zu ändern, beschreibt \textit{Fowler} in \cite{Fow99}.

\subsubsection*{Vorgehen}
Refactoring besteht aus mehreren Schritten:

\begin{enumerate}
    \item Problematische Stellen identifizieren (sog. \textit{Code Smells})
    \item Refactorings anwenden, um die problematischen Stellen zu verbessern (\textit{Martin}: ``\textit{Heuristics}`` (vgl.~\cite[285 ff.]{Mar08}))
    \item Tests ausführen, um sicherzustellen, dass das Refactoring nicht die Funktionalität geändert hat
\end{enumerate}

\subsubsection*{Refactoring verursacht Aufwand}
Mit Refactoring kann man die Qualität von Code verbessern.
Allerdings zeigt die Praxis, dass fast jeder code kleinere oder größere Probleme hat, die immer weiter verbessert werden können.
Teilweise sind Refactorings mit hohem Aufwand verbunden, wobei nur Qualität, aber nicht Funktionalität verbessert wird.\\
Es sollte deshalb stehts zwischen Dringlichkeit, Aufwand und Auswirkungen abgewogen werden.